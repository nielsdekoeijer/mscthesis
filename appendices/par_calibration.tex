Calibration is necessary for the Par detectability to provide the correct output.
In the previous section, it was shown that the constants $C_a$ and $C_s$ are used to this end.
Correct calibration of the Par detectability must satisfy the following:
\begin{enumerate}
    \item The just noticeable disturbance signal must result in a detectability of $1$.
    \item The threshold of hearing takes effect appropriately.
\end{enumerate}
Both the concepts of just noticeable distortion and the threshold of hearing require knowledge of the sound pressure level of the stimuli.
Thus, before determining the calibration coefficients, it is important to first discuss the 
relationship between the input signals $x[n]$ and $\varepsilon[n]$ and reproduced sound pressure level.
This is the topic of \autoref{ch:perceptual:implementation:calibration:sound_pressure_level}.
Afterwards, \autoref{ch:perceptual:implementation:calibration:coefficients} discusses the determination of 
the coefficients $C_a$ and $C_s$. 

\section{Relating Digital Representation and Sound Pressure Level}
\label{ch:perceptual:implementation:calibration:sound_pressure_level}
One difficulty of taking the threshold of hearing into account is that it is typically given in terms of sound pressure level (SPL), measured in dB. 
The one-sided spectrum of the threshold of hearing in dB SPL can be approximated by the following function~\cite{painter2000perceptual}:
\begin{equation}
    T_q(f) = 3.64\left(\frac{f}{1000}\right)^{-0.8} + 0.001\left(\frac{f}{1000}\right)^4 - 6.5\exp\left[-0.6\left(\frac{f}{1000}-3.3\right)^2\right]
\end{equation}
The signals $x[n]$ and $\varepsilon[n]$ are however given digital representation of audio. 

For example, they might be given in a pulse code modulated (PCM) format within which they attain integer values between -32768 and 32767.

As such, to meaningfully integrate the threshold of quiet, the digital representation and the sound pressure levels must be related.
This relationship can be modeled as follows:
\begin{equation}
    X_\text{dB}(f) = 10\log_{10}(\left|X(f)\right|^2) + O_\text{dB}
    \label{eq:perceptual:implementation:calibration:sound_pressure_level:dB_representation}
\end{equation}
Here, $X_\text{dB}(f)$ is the dB SPL representation of a given spectrum $X(f)$.
Furthermore, $O_\text{dB}$ is an offset to ensure the digital representation corresponds to the correct sound pressure level. 
In order to use this relationship to determine the appropriate digital equivalent of the threshold in quiet, a definition of the offset $O_\text{dB}$
must be determined.

One way of determining the offset $O_\text{dB}$ is by relating the sound pressure level and the digital representation of a full-scale sinusoid.
A full-scale sinusoid is a sinusoid that has an amplitude of the maximum value that can be attained in the digital representation.

In our previous example, one way of doing so would be to state that a full-scale sinusoid with amplitude 32767 corresponds to 
e.g., a sound pressure level of 100 dB SPL.
The interpretation of this is that playing a full-scale sinusoid will result in a sound pressure of 100 dB SPL when played from the sound system.

To do so, let the digital representation of the full-scale sinusoid be modeled by a sinusoid with amplitude $A$ and frequency $f_0$.
Consider the one-sided fourier representation of the digital representation of this full-scale sinusoid: 
\begin{equation}
    \mathcal{F}\left\{A\cos\left(2\pi f_0 t\right)\right\} = A\delta\left(f - f_0\right)
\end{equation}
It is assumed that playing the digital representation of this sinusoid results in a sound pressure level of $A_\text{dB}$ dB SPL. 
Substituting these definitions into \autoref{eq:perceptual:implementation:calibration:sound_pressure_level:dB_representation} 
results in the following definition for $O_\text{dB}$: 
\begin{align}
    O_\text{dB} &= 10\log_{10}\left(\left|A\right|^2\right) - A_\text{dB} 
\end{align}
The offset fully defines the relationship between digital representation and sound pressure level, and allows for the conversion of the threshold of hearing
to digital representation.

\section{Determining Calibration Constants}
\label{ch:perceptual:implementation:calibration:coefficients}
There are various ways of calibrating this model, but this section will discuss the method of calibrating 
that is given in the original paper~\cite{van2005perceptual}.
The given approach is to find the two unknowns $C_a$ and $C_s$ by solving a system of two equations that model the previously stated calibration requirements.

The first requirement is that a just noticeable disturbance signal must result in a detectability of 1.
From perceptual literature, it is known that a sinusoidal disturbance signal at a given frequency $f_0$ is just noticeable in the presence of an in-phase sinusoidal masking signal that is 18 dB SPL louder~\cite{van2005perceptual}.
To model this, consider the following masking and disturbance signals. 
\begin{align}
    x_\text{JND}[n] &= A_{70}\cos\left(2\pi f_0 n / f_s\right) \\
    \varepsilon_\text{JND}[n] &= A_{52}\cos\left(2\pi f_0 n / f_s\right)
\end{align}
Here, $x_\text{JND}[n]$ is a sinusoid with an amplitude $A_{70}$, which corresponds to 70 dB SPL.
Furthermore, $\varepsilon_\text{JND}[n]$ is a sinusoid with an amplitude $A_{52}$, which is 18 dB SPL less.
Note that the amplitudes are both given in digital representation, not sound pressure level representation.
The digital representation amplitudes are found through \autoref{eq:perceptual:implementation:calibration:sound_pressure_level:dB_representation}.

Thus, $\varepsilon_\text{JND}[n]$ must be just noticeable in presence of $x_\text{JND}[n]$.
This can be expressed as follows:
\begin{equation}
    D(x_\text{JND}[n],\varepsilon_\text{JND}[n]) = 1
    \label{eq:perceptual:implementation:calibration:coefficients:first}
\end{equation}
This expression forms the first equation in the system of equations that can be solved to calibrate the Par detectability.

The second requirement is that the threshold of hearing must be included correctly.
The threshold of hearing defines the sound pressure levels that are the verge between audible and inaudible sound as a function of frequency.
To this end, consider the following masking and disturbance signals:
\begin{align}
    x_\text{THR}[n] &= 0 \\
    \varepsilon_\text{THR}[n] &= A_{\text{tq}}\cos\left(2\pi f_0 n / f_s\right)
\end{align}
Here the masking signal $x_\text{THR}[n]$ is zero. 
The disturbance signal is a sinusoid of frequency $f_0$ with amplitude $A_{\text{tq}}$, which is chosen such that it attains 
the threshold of quiet at $f_0$, i.e. $T_q(f_0)$. 

As the threshold of quiet is the verge between audible and inaudible sound, it is assumed that a disturbance signal in the presence of no masking signal that has
an amplitude equal to the threshold of quiet is just noticeable.
Recall that for just noticeable stimuli, the detectability must be equal to 1.
This allows us to specify the second equation in the system of equations:
\begin{equation}
    D(0,\varepsilon_\text{THR}[n]) = 1
    \label{eq:perceptual:implementation:calibration:coefficients:second}
\end{equation}

The system of equations defined by \autoref{eq:perceptual:implementation:calibration:coefficients:first} and 
\autoref{eq:perceptual:implementation:calibration:coefficients:second} can be solved through the bisection method.
To see how this is done, the reader is referred to the original paper~\cite{van2005perceptual}.
