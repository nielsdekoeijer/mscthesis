In \autoref{ch:perceptual_sound_zone:perceptual_minimization} two perceptual sound zone algorithms were introduced. 
First, an unconstrained perceptual pressure matching approach where the detectability of the sound pressure errors is minimized.
Secondly, a constrained perceptual pressure matching approach which leverages the fact that the detectability has a consistent 
perceptual interpretation to constrain the detectability of the reproduction error.

This section will explore the results of both approaches.

\subsection{Evaluating Unconstrained Perceptual Pressure Matching}
In this section, the unconstrained perceptual pressure matching algorithm will be evaluated.
This algorithm minimized the total detectability of the reproduction and the leakage error.
As mentioned in \autoref{ch:results:methodology}, the perceptual pressure matching approaches will be evaluated with respect to the 
reference pressure matching algorithm introduced in \autoref{ch:sound_zone:approaches}.
The unconstrained reference algorithm instead minimizes the total reproduction and leakage error.

The experiment consists of 12 simulations which use speech signals for both zones.
For evaluation, 3 metrics are used as introduced by \autoref{ch:results:methodology}.
First, Overall PESQ, which quantifies the quality of the total experience: achieved intended audio and interference.
Secondly, Target PESQ, which quantifies the quality of just the achieved intended audio (excluding interference).
Finally, Distraction, which determines how distracting the interference is.
These three quantities were found to be representative for the sound zone experience.

\begin{table}[]
    \centering
\begin{tabular}{|l|l|c|c|}
\hline
\multicolumn{2}{|l|}{Statistic}                                                  
& \multicolumn{1}{l|}{\textbf{Reference}} & \multicolumn{1}{l|}{\textbf{Perceptual}} \\ \hline
\multicolumn{1}{|c|}{\multirow{2}{*}{\textbf{Overall PESQ}}} 
& Mean               & 2.61                                    & 3.15                                     \\ \cline{2-4} 
\multicolumn{1}{|c|}{}                                      
& Standard Deviation & 0.247                                   & 0.240                                    \\ \hline
\multirow{2}{*}{\textbf{Target PESQ}}                        
& Mean               & 4.11                                    & 3.35                                     \\ \cline{2-4} 
& Standard Deviation & 0.065                                   & 0.254                                    \\ \hline
\multirow{2}{*}{\textbf{Distraction}}                       
& Mean               & 12.7                                    & 7.8                                      \\ \cline{2-4} 
& Standard Deviation & 7.46                                    & 5.89                                     \\ \hline
\end{tabular}
\caption{
    Summary of results for the evaluation of the unconstrained perceptual pressure matching approach,
    using the evaluation metrics defined in \autoref{ch:results:methodology}.
}
\label{ch:results:results:unconstrained_results}
\end{table}

The results of the experiment are summarized in \autoref{ch:results:results:unconstrained_results}.
This table depicts the mean and standard deviation of the metrics taken over all 4 control points and all 12 experiments.
From this, the following conclusions are drawn:
\begin{itemize}
    \item The perceptual sound zone algorithm outperforms the reference sound zone algorithm in Overall PESQ.
        Thus, this implies that perceptual sound zone algorithm delivers greater quality of speech.
    \item The reference sound zone algorithm outperforms the perceptual sound zone algorithm in Target PESQ.
        This implies that the reference algorithm, when disregarding interference, delivers greater quality for the intended speech signal.
    \item The perceptual sound zone algorithm outperforms the reference with regards to Distraction.
        This can be understood as the reference results in more distracting sound zones than the perceptual sound zone algorithm.
\end{itemize}
From the results above, the Overall PESQ and the Distraction indicate that the perceptual sound zone algorithm seems to 
achieve an better overall experience.
The reference sound zone algorithm achieves a greater Target PESQ, however after adding the interference, the Overall PESQ is lower.

This implies that the the perceptual sound zone algorithm makes a better overall perceptual trade-off between producing the target content 
and minimizing the leakage.

\begin{tikzpicture}
    \begin{groupplot}[group style={group size=1 by 3, vertical sep=2cm}]
        \nextgroupplot[xlabel={Time [samples]}, ylabel={$t^{(m)}[n]$ [SPL]}, ylabel near ticks, 
        title={Target Sound Pressure for Control Point $m\in A$}, scale only axis, 
        height=3cm, width=9cm, xmin=5000, xmax=6400, xtick={5000, 5200, 5400, 5600, 5800, 6000, 6200, 6400},
        ymin=-0.5, ymax=0.5]
        \addplot [color=red, error bars/.cd, y dir=both, y explicit]
        table [x expr=\thisrow{samples}, y expr=(\thisrow{audio} / 32787), col sep=comma] 
            {data/target.csv};
       \addplot [samples=50, smooth, name path=H11] coordinates {(5500,-0.5) (5500,0.5)};
       \addplot [samples=50, smooth, name path=H12] coordinates {(6200,-0.5) (6200,0.5)};
       \addplot [fill opacity=0.05] fill between [of=H11 and H12];

        \nextgroupplot[xlabel={time [samples]}, ylabel={$p_\zb^{(m)}[n]$}, ylabel near ticks, 
        title={Perceptual Achieved Dark Zone Sound Pressure for Control Point $m\in A$}, scale only axis, 
        height=3cm, width=9cm, xmin=5000, xmax=6400, xtick={5000, 5200, 5400, 5600, 5800, 6000, 6200, 6400},
        ymin=-0.5, ymax=0.5]
        \addplot [color=red, error bars/.cd, y dir=both, y explicit]
        table [x expr=\thisrow{samples}, y expr=(\thisrow{audio} / 32787), col sep=comma] 
            {data/per_leakage.csv};
       \addplot [samples=50, smooth, name path=H11] coordinates {(5500,-0.5) (5500,0.5)};
       \addplot [samples=50, smooth, name path=H12] coordinates {(6200,-0.5) (6200,0.5)};
       \addplot [fill opacity=0.05] fill between [of=H11 and H12];

        \nextgroupplot[xlabel={time [samples]}, ylabel={$p_\zb^{(m)}[n]$}, ylabel near ticks, 
        title={Reference Achieved Bright Zone Sound Pressure in Control Point $m\in A$}, scale only axis, 
        height=3cm, width=9cm, xmin=5000, xmax=6400, xtick={5000, 5200, 5400, 5600, 5800, 6000, 6200, 6400},
        ymin=-0.5, ymax=0.5]
        \addplot [color=red, error bars/.cd, y dir=both, y explicit]
        table [x expr=\thisrow{samples}, y expr=(\thisrow{audio} / 32787), col sep=comma] 
            {data/ref_leakage.csv};
       \addplot [samples=50, smooth, name path=H11] coordinates {(5500,-0.5) (5500,0.5)};
       \addplot [samples=50, smooth, name path=H12] coordinates {(6200,-0.5) (6200,0.5)};
       \addplot [fill opacity=0.05] fill between [of=H11 and H12];
    \end{groupplot}
\end{tikzpicture}

\begin{figure}[p]
    \centering
    \makebox[\linewidth]{
    \begin{tikzpicture}
        \begin{groupplot}[group style={group size=2 by 4, vertical sep=2.25cm, horizontal sep=2.5cm}]

            \nextgroupplot[xlabel={time [samples]}, ylabel={$t^{(m)}[n]$}, ylabel near ticks, 
            title={Target Sound Pressure $m\in A$ }, scale only axis, 
            height=3cm, width=6cm, xmin=29000, xmax=31000, xtick={29000, 30000, 31000},
            ymin=-0.5, ymax=0.5]
            \addplot [color=red, error bars/.cd, y dir=both, y explicit]
            table [x expr=\thisrow{samples}, y expr=(\thisrow{audio} / 32787), col sep=comma] 
                {data/target_Q.csv};
            \addplot [samples=50, smooth, name path=H11] coordinates {(29300,-0.5) (29300,0.5)};
            \addplot [samples=50, smooth, name path=H12] coordinates {(30250,-0.5) (30250,0.5)};
            \addplot [fill opacity=0.05] fill between [of=H11 and H12];

            \nextgroupplot[xlabel={time [samples]}, ylabel={$t^{(m)}[n]$}, ylabel near ticks, 
            title={Target Sound Pressure $m\in B$}, scale only axis, 
            height=3cm, width=6cm, xmin=29000, xmax=31000, xtick={29000, 30000, 31000},
            ymin=-0.5, ymax=0.5]
            \addplot [color=red, error bars/.cd, y dir=both, y explicit]
            table [x expr=\thisrow{samples}, y expr=(\thisrow{audio} / 32787), col sep=comma] 
                {data/target_Q_offzone.csv};
            \addplot [samples=50, smooth, name path=H11] coordinates {(29750,-0.5) (29750,0.5)};
            \addplot [samples=50, smooth, name path=H12] coordinates {(30850,-0.5) (30850,0.5)};
            \addplot [fill opacity=0.05] fill between [of=H11 and H12];

            \nextgroupplot[xlabel={time [samples]}, ylabel={$p_\za^{(m)}[n]$}, ylabel near ticks, 
            title={Reconstruction for $Q=1$}, scale only axis, 
            height=3cm, width=6cm, xmin=29000, xmax=31000, xtick={29000, 30000, 31000},
            ymin=-0.5, ymax=0.5]
            \addplot [color=red, error bars/.cd, y dir=both, y explicit]
            table [x expr=\thisrow{samples}, y expr=(\thisrow{audio} / 32787), col sep=comma] 
                {data/reconstruction_Q_1.csv};
            \addplot [samples=50, smooth, name path=H11] coordinates {(29300,-0.5) (29300,0.5)};
            \addplot [samples=50, smooth, name path=H12] coordinates {(30250,-0.5) (30250,0.5)};
            \addplot [fill opacity=0.05] fill between [of=H11 and H12];

            \nextgroupplot[xlabel={time [samples]}, ylabel={$p_\zb^{(m)}[n]$}, ylabel near ticks, 
            title={Leakage for $Q=1$}, scale only axis, 
            height=3cm, width=6cm, xmin=29000, xmax=31000, xtick={29000, 30000, 31000},
            ymin=-0.5, ymax=0.5]
            \addplot [color=red, error bars/.cd, y dir=both, y explicit]
            table [x expr=\thisrow{samples}, y expr=(\thisrow{audio} / 32787), col sep=comma] 
                {data/leakage_Q_1.csv};
            \addplot [samples=50, smooth, name path=H11] coordinates {(29750,-0.5) (29750,0.5)};
            \addplot [samples=50, smooth, name path=H12] coordinates {(30850,-0.5) (30850,0.5)};
            \addplot [fill opacity=0.05] fill between [of=H11 and H12];

            \nextgroupplot[xlabel={time [samples]}, ylabel={$p_\za^{(m)}[n]$}, ylabel near ticks, 
            title={Reconstruction for $Q=7$}, scale only axis, 
            height=3cm, width=6cm, xmin=29000, xmax=31000, xtick={29000, 30000, 31000},
            ymin=-0.5, ymax=0.5]
            \addplot [color=red, error bars/.cd, y dir=both, y explicit]
            table [x expr=\thisrow{samples}, y expr=(\thisrow{audio} / 32787), col sep=comma] 
                {data/reconstruction_Q_7.csv};
            \addplot [samples=50, smooth, name path=H11] coordinates {(29300,-0.5) (29300,0.5)};
            \addplot [samples=50, smooth, name path=H12] coordinates {(30250,-0.5) (30250,0.5)};
            \addplot [fill opacity=0.05] fill between [of=H11 and H12];

            \nextgroupplot[xlabel={time [samples]}, ylabel={$p_\zb^{(m)}[n]$}, ylabel near ticks, 
            title={Leakage for $Q=7$}, scale only axis, 
            height=3cm, width=6cm, xmin=29000, xmax=31000, xtick={29000, 30000, 31000},
            ymin=-0.5, ymax=0.5]
            \addplot [color=red, error bars/.cd, y dir=both, y explicit]
            table [x expr=\thisrow{samples}, y expr=(\thisrow{audio} / 32787), col sep=comma] 
                {data/leakage_Q_7.csv};
            \addplot [samples=50, smooth, name path=H11] coordinates {(29750,-0.5) (29750,0.5)};
            \addplot [samples=50, smooth, name path=H12] coordinates {(30850,-0.5) (30850,0.5)};
            \addplot [fill opacity=0.05] fill between [of=H11 and H12];

            \nextgroupplot[xlabel={time [samples]}, ylabel={$p_\za^{(m)}[n]$}, ylabel near ticks, 
            title={Reconstruction for $Q=21$}, scale only axis, 
            height=3cm, width=6cm, xmin=29000, xmax=31000, xtick={29000, 30000, 31000},
            ymin=-0.5, ymax=0.5]
            \addplot [color=red, error bars/.cd, y dir=both, y explicit]
            table [x expr=\thisrow{samples}, y expr=(\thisrow{audio} / 32787), col sep=comma] 
                {data/reconstruction_Q_21.csv};
            \addplot [samples=50, smooth, name path=H11] coordinates {(29300,-0.5) (29300,0.5)};
            \addplot [samples=50, smooth, name path=H12] coordinates {(30250,-0.5) (30250,0.5)};
            \addplot [fill opacity=0.05] fill between [of=H11 and H12];

            \nextgroupplot[xlabel={time [samples]}, ylabel={$p_\zb^{(m)}[n]$}, ylabel near ticks, 
            title={Leakage for $Q=21$}, scale only axis, 
            height=3cm, width=6cm, xmin=29000, xmax=31000, xtick={29000, 30000, 31000},
            ymin=-0.5, ymax=0.5]
            \addplot [color=red, error bars/.cd, y dir=both, y explicit]
            table [x expr=\thisrow{samples}, y expr=(\thisrow{audio} / 32787), col sep=comma] 
                {data/leakage_Q_21.csv};
            \addplot [samples=50, smooth, name path=H11] coordinates {(29750,-0.5) (29750,0.5)};
            \addplot [samples=50, smooth, name path=H12] coordinates {(30850,-0.5) (30850,0.5)};
            \addplot [fill opacity=0.05] fill between [of=H11 and H12];
        \end{groupplot}
\end{tikzpicture}}
\caption{Plots depicting stuff.}
\end{figure}


\subsection{Evaluating Constrained Perceptual Pressure Matching}

\constrainingplot
{A__0.wav__target_reconstruction_pesq_mean}
{A__0.wav__target_reconstruction_pesq_std}
{A__0.wav__target_result_pesq_mean}
{A__0.wav__target_result_pesq_std}
{A__0.wav__reconstruction_leakage_distraction_mean}
{A__0.wav__reconstruction_leakage_distraction_std}
