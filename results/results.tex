In \autoref{ch:perceptual_sound_zone:perceptual_minimization} two perceptual sound zone algorithms are proposed. 
First, an unconstrained perceptual pressure matching approach where the detectability of the sound pressure 
errors is minimized.
Secondly, a constrained perceptual pressure matching approach which leverages the fact that the 
detectability has a consistent 
perceptual interpretation to constrain the detectability of the reproduction error.

This section will evaluate the results of the performed experiments for both proposed approaches.
To this end, \autoref{ch:results:results:unconstrained_results} the unconstrained perceptual pressure 
matching approach is evaluated, and in  \autoref{ch:results:results:constrained_results} the constrained perceptual pressure 
matching approach is evaluated.

In order to effectively describe various points in the room, a version of the room 
given by \autoref{fig:results:methodology:room}
with the various control points numbered is given in \autoref{fig:results:methodology:room_numbered}.

\begin{figure}[]
    \centering
    \scalebox{1.0}{\begin{tikzpicture}
    \draw [draw=black] (0,0) rectangle (5,5);

    % Speakers on the top wall
    \pic[scale=0.7] at (1.00, 4.25) {Speaker};
    \pic[scale=0.7] at (4.00, 4.25) {Speaker};

    % Speakers on the bottom wall
    \pic[rotate=180, scale=0.7] at (1.00, 0.75) {Speaker};
    \pic[rotate=180, scale=0.7] at (4.00, 0.75) {Speaker};

    \draw[opacity=0.4, fill=blue, draw=white] (1.25, 2.00) circle[radius=0.15];
    \draw[thick] (1.25,2.00) circle (.15) node[align=center] {\scriptsize \textbf{1}};
    \draw[opacity=0.4, fill=blue, draw=white] (2.25, 2.00) circle[radius=0.15];
    \draw[thick] (2.25,2.00) circle (.15) node[align=center] {\scriptsize \textbf{2}};
    \draw[opacity=0.4, fill=red, draw=white] (2.75, 2.00) circle[radius=0.15];
    \draw[thick] (2.75,2.00) circle (.15) node[align=center] {\scriptsize \textbf{3}};
    \draw[opacity=0.4, fill=red, draw=white] (3.75, 2.00) circle[radius=0.15];
    \draw[thick] (3.75,2.00) circle (.15) node[align=center] {\scriptsize \textbf{4}};
\end{tikzpicture}
}
    \caption{The square room used in \autoref{fig:results:methodology:room} labeled with numbers corresponding to 
    the various control points.}
    \label{fig:results:methodology:room_numbered}
\end{figure}


\subsection{Evaluating Unconstrained Perceptual Pressure Matching}
\label{ch:results:results:unconstrained_results}
In this section, the unconstrained perceptual pressure matching algorithm will be evaluated in accordance to the 
approach discussed in \autoref{ch:results:methodology}.
This is done by first evaluating the results of the simulations of the various setups 
quantitatively through the proposed measures.
From this, a conclusions are drawn which are then motivated qualitatively by reasoning about algorithm behavior through
waveforms.

\subsubsection*{Quantitative Analysis of Simulation Results}

\begin{table}[]
\begin{tabular}{|l|l|l|l|}
\hline
\multicolumn{2}{|c|}{\textbf{Measure}}                           & \textbf{Unconstrained Perceptual} & \textbf{Reference} \\ \hline
\multirow{2}{*}{\textbf{Distraction}}       & Mean               & 7.83                              & 12.69              \\
                                            & Standard Deviation & 6.19                              & 11.29              \\ \hline
\multirow{2}{*}{\textbf{Acoustic Contrast}} & Mean               & 22.03                             & 50.41              \\
                                            & Standard Deviation & 5.96                              & 29.02              \\ \hline
\multirow{2}{*}{\textbf{Total PESQ}}        & Mean               & 3.15                              & 2.61               \\
                                            & Standard Deviation & 0.27                              & 0.28               \\ \hline
\multirow{2}{*}{\textbf{Bright Zone PESQ}}  & Mean               & 3.35                              & 4.11               \\
                                            & Standard Deviation & 0.29                              & 0.17               \\ \hline
\multirow{2}{*}{\textbf{Total STOI}}        & Mean               & 0.94                              & 0.94               \\
                                            & Standard Deviation & 0.01                              & 0.02               \\ \hline
\multirow{2}{*}{\textbf{Bright Zone STOI}}  & Mean               & 0.95                              & 0.99               \\
                                            & Standard Deviation & 0.01                              & 0.01               \\ \hline
\multirow{2}{*}{\textbf{Total SIIB}}        & Mean               & 978.21                            & 784.96             \\
                                            & Standard Deviation & 69.20                             & 164.45             \\ \hline
\multirow{2}{*}{\textbf{Bright Zone SIIB}}  & Mean               & 1125.50                           & 1201.76            \\
                                            & Standard Deviation & 66.18                             & 90.99              \\ \hline
\multirow{2}{*}{\textbf{Total MSE}}         & Mean               & 0.115\%                           & 0.016\%            \\
                                            & Standard Deviation & 0.037\%                           & 0.007\%            \\ \hline
\multirow{2}{*}{\textbf{Bright Zone MSE}}   & Mean               & 0.107\%                           & 0.008\%            \\
                                            & Standard Deviation & 0.036\%                           & 0.004\%            \\ \hline
\end{tabular}
\caption{
    Summary of results for the evaluation of the unconstrained perceptual pressure matching approach,
    using the evaluation metrics defined in \autoref{ch:results:methodology}.
}
\label{ch:results:results:unconstrained_results}
\end{table}

In order to quantify the performance of the unconstrained perceptual pressure matching approach, the various 
measures introduced in \autoref{ch:results:methodology} are determined for all 12 simulations.
The measures are averaged over all simulations and over all control points.
For comparison purposes, the reference algorithm given in \autoref{ch:perceptual_sound_zone:stft} is 
simulated in an identical fashion, and the measures are also evaluated and averaged for all simulations.

The results of this experiment is summarized in \autoref{ch:results:results:unconstrained_results}.
This table depicts the mean and standard deviation of the measures taken over all 4 control points and all 12 experiments.
As discussed in \autoref{ch:perceptual:review}, it is important to note that the perceptual measures can only
be used as an indication, and further listening tests must be performed to draw any real conclusions.

From the results in the table, the following observations are made and conclusions are drawn:
\begin{enumerate}
    \item The perceptual approach outperforms the reference in all perceptual measures evaluating the total experience:
        total PESQ, STOI, and SIIB all attain higher values for the perceptual approach.
        Note that these measures evaluate the total experience, taking in consideration all sound pressure per control 
        point including interference.

        This implies that the perceptual approach may result in an overall better perceptual experience.
        \label{obs:total_perceptual}

    \item The perceptual approach outperforms the reference in terms of the perceptual distraction measure,
        Implying that the interference in the perceptual approach may be less distracting 
        \label{obs:distraction}

    \item The reference approach outperforms the perceptual approach in perceptual measures evaluating the 
        bright-zone quantities: bright zone PESQ, STOI and SIIB are all higher for the reference approach.
        Note that these measures are sans interference: they only evaluate how well the achieved sound pressure
        attains the target, ignoring interference.

        This implies that, disregarding interference, the reference approximates the target more effectively 
        perceptually speaking.
        However, in \autoref{obs:total_perceptual} it is concluded that adding the interferer results in a 
        lower overall experience.

        This implies that, although the reference algorithm approximates the target better perceptually, 
        the interference that it introduces a sufficient disturbance to be outperformed by the reference.
        \label{obs:bright_zone_perceptual}

    \item The reference approach outperforms the perceptual approach for all physical measures: 
        total and bright zone MSE and acoustic contrast.
        This is to be expected, as the reference approach optimizes the MSE directly.

        Interestingly, although the total MSE is lower, the reference is outperformed in terms of all total perceptual 
        measures, as discussed in \autoref{obs:total_perceptual}.

        In addition to this, the acoustic contrast between intended and interfering sound pressure for 
        the reference approach is over twice as large than the perceptual approach.
        Nevertheless, the perceptual approach less distracting according to the distraction model as 
        discussed in \autoref{obs:distraction}.

        These results imply that MSE or AC may not be optimal measures to use in the evaluation of the  
        perceptual experience of sound zones.
        \label{obs:physical}
\end{enumerate}

In summary, from the observations above it is concluded that the perceptual sound zone algorithm may
outperform the reference sound zone algorithm in terms of perceptual experience.
This seems to be due to the perceptually disturbing interference introduced by the reference algorithm,
as, when disregarding noise, the reference algorithm has a better reproduction of the target perceptually speaking.
The distraction ratings also indicate that the noise introduced by the reference algorithm is more distracting.

In addition to this, these results also imply that the physical metrics MSE and AC do not correlate well 
with the perceptual measures, as the physical metrics predict the reference to outperform the perceptual approach.

\subsubsection*{Analyzing Algorithm Behavior}
\begin{figure}[]
    \centering
    \begin{tikzpicture}
    \begin{groupplot}[group style={group size=1 by 3, vertical sep=2cm}]
        \nextgroupplot[xlabel={Time [samples]}, ylabel={$t^{(m)}[n]$ [SPL]}, ylabel near ticks, 
        title={Target Sound Pressure for Control Point $m\in A$}, scale only axis, 
        height=3cm, width=9cm, xmin=5000, xmax=6400, xtick={5000, 5200, 5400, 5600, 5800, 6000, 6200, 6400},
        ymin=-0.5, ymax=0.5]
        \addplot [color=red, error bars/.cd, y dir=both, y explicit]
        table [x expr=\thisrow{samples}, y expr=(\thisrow{audio} / 32787), col sep=comma] 
            {data/target.csv};
       \addplot [samples=50, smooth, name path=H11] coordinates {(5500,-0.5) (5500,0.5)};
       \addplot [samples=50, smooth, name path=H12] coordinates {(6200,-0.5) (6200,0.5)};
       \addplot [fill opacity=0.05] fill between [of=H11 and H12];

        \nextgroupplot[xlabel={time [samples]}, ylabel={$p_\zb^{(m)}[n]$}, ylabel near ticks, 
        title={Perceptual Achieved Dark Zone Sound Pressure for Control Point $m\in A$}, scale only axis, 
        height=3cm, width=9cm, xmin=5000, xmax=6400, xtick={5000, 5200, 5400, 5600, 5800, 6000, 6200, 6400},
        ymin=-0.5, ymax=0.5]
        \addplot [color=red, error bars/.cd, y dir=both, y explicit]
        table [x expr=\thisrow{samples}, y expr=(\thisrow{audio} / 32787), col sep=comma] 
            {data/per_leakage.csv};
       \addplot [samples=50, smooth, name path=H11] coordinates {(5500,-0.5) (5500,0.5)};
       \addplot [samples=50, smooth, name path=H12] coordinates {(6200,-0.5) (6200,0.5)};
       \addplot [fill opacity=0.05] fill between [of=H11 and H12];

        \nextgroupplot[xlabel={time [samples]}, ylabel={$p_\zb^{(m)}[n]$}, ylabel near ticks, 
        title={Reference Achieved Bright Zone Sound Pressure in Control Point $m\in A$}, scale only axis, 
        height=3cm, width=9cm, xmin=5000, xmax=6400, xtick={5000, 5200, 5400, 5600, 5800, 6000, 6200, 6400},
        ymin=-0.5, ymax=0.5]
        \addplot [color=red, error bars/.cd, y dir=both, y explicit]
        table [x expr=\thisrow{samples}, y expr=(\thisrow{audio} / 32787), col sep=comma] 
            {data/ref_leakage.csv};
       \addplot [samples=50, smooth, name path=H11] coordinates {(5500,-0.5) (5500,0.5)};
       \addplot [samples=50, smooth, name path=H12] coordinates {(6200,-0.5) (6200,0.5)};
       \addplot [fill opacity=0.05] fill between [of=H11 and H12];
    \end{groupplot}
\end{tikzpicture}

    \caption{Depiction of the wave forms of the target sound pressure and achieved dark zone sound pressure for the 
        unconstrained perceptual pressure matching approach and the reference pressure matching approach.}
        \label{fig:results:evaluation:unconstrained_results:behavior}
\end{figure}

In the preceding section the reference and perceptual algorithms are compared quantitatively.
Results indicate that the perceptual algorithm outperforms the reference in terms of the total experience.
This section considers the behavior of the algorithm in an attempt to explain these results.

To this end, consider \autoref{fig:results:evaluation:unconstrained_results:behavior}.
This figure contains a plot of the wave forms of the target sound pressure and achieved dark-zone sound pressure 
for both the perceptual and non-perceptual variants of pressure matching for control point $m=2$ 
(see: \autoref{fig:results:methodology:room_numbered}) from the experiments.

The selected control point is in zone $\za$. 
As explained in \autoref{ch:results:methodology}, the achieved dark-zone sound pressure can be understood 
as the interference due to the other zone, zone $\zb$.

Consider the highlighted region for the highlighted region for the perceptual algorithm.
From this it can be seen that the 
magnitude of the interference is correlated to the magnitude of the target sound pressure for zone $\za$.
Contrast this to the highlighted region for the reference algorithm, where the interference is a 
at a relatively constant level.

This may explain why, while having lower overall contrast, the perceptual approach outperforms the 
reference approach in terms of distraction and overall perceptual experience.
When determining the interference for control point $m=2$, the perceptual algorithm takes the 
target sound pressure for $\za$ into account. 
Effectively, when the target sound pressure is relatively loud, more interference is allowed 
as it is masked to a degree by the target sound pressure.

In doing so, the interference is less perceptually disturbing, which serves as a possible explanation to 
the results given in \autoref{ch:results:results:unconstrained_results}

\subsection{Evaluating Constrained Perceptual Pressure Matching}
\label{ch:results:results:constrained_results}

\subsubsection*{Quantifying Algorithm Performance}
\begin{figure}[p]
    \centering
    \noindent
    \makebox[\linewidth]{
    \begin{tikzpicture}
        % PESQ SIIB STOI MSE CONTRAST DISTRACTION
        \begin{groupplot}[group style={group size=2 by 3, vertical sep=2.25cm, horizontal sep=2.5cm}]
            \nextgroupplot[xlabel={$D_0$}, ylabel={PESQ}, ylabel near ticks, 
            title={Average PESQ per Constraint $D_0$}, scale only axis, 
            height=3cm, width=6cm, xmin=1, xmax=25, legend style={nodes={scale=0.5, transform shape}}] 
            \addplot [color=blue, error bars/.cd, y dir=both, y explicit]
            table [x expr=\thisrow{Q}, 
                    y expr=(\thisrow{target_reconstruction_pesq_mean}), 
                    y error expr=(\thisrow{target_reconstruction_pesq_std}), 
                    col sep=comma] 
                {data/constrained_evaluation_per_Q.csv};
            \addplot [color=red, error bars/.cd, y dir=both, y explicit]
            table [x expr=\thisrow{Q}, 
                    y expr=(\thisrow{target_result_pesq_mean}), 
                    y error expr=(\thisrow{target_result_pesq_std}), 
                    col sep=comma] 
                {data/constrained_evaluation_per_Q.csv};
            \legend{Bright Zone PESQ, Total PESQ};

            \nextgroupplot[xlabel={$D_0$}, ylabel={MSE}, ylabel near ticks, 
            title={Average MSE per Constraint $D_0$}, scale only axis, 
            height=3cm, width=6cm, xmin=1, xmax=25, legend style={nodes={scale=0.5, transform shape}}] 
            \addplot [color=blue, error bars/.cd, y dir=both, y explicit]
            table [x expr=\thisrow{Q}, 
                    y expr=(\thisrow{target_reconstruction_mse_mean}), 
                    y error expr=(\thisrow{target_reconstruction_mse_std}), 
                    col sep=comma] 
                {data/constrained_evaluation_per_Q.csv};
            \addplot [color=red, error bars/.cd, y dir=both, y explicit]
            table [x expr=\thisrow{Q}, 
                    y expr=(\thisrow{target_result_mse_mean}), 
                    y error expr=(\thisrow{target_result_mse_std}), 
                    col sep=comma] 
                {data/constrained_evaluation_per_Q.csv};
            \legend{Bright Zone MSE, Total MSE};

            \nextgroupplot[xlabel={$D_0$}, ylabel={STOI}, ylabel near ticks, 
            title={Average STOI per Constraint $D_0$}, scale only axis, 
            height=3cm, width=6cm, xmin=1, xmax=25, legend style={nodes={scale=0.5, transform shape}}] 
            \addplot [color=blue, error bars/.cd, y dir=both, y explicit]
            table [x expr=\thisrow{Q}, 
                    y expr=(\thisrow{target_reconstruction_stoi_mean}), 
                    y error expr=(\thisrow{target_reconstruction_stoi_std}), 
                    col sep=comma] 
                {data/constrained_evaluation_per_Q.csv};
            \addplot [color=red, error bars/.cd, y dir=both, y explicit]
            table [x expr=\thisrow{Q}, 
                    y expr=(\thisrow{target_result_stoi_mean}), 
                    y error expr=(\thisrow{target_result_stoi_std}), 
                    col sep=comma] 
                {data/constrained_evaluation_per_Q.csv};
            \legend{Bright Zone STOI, Total STOI};

            \nextgroupplot[xlabel={$D_0$}, ylabel={SIIB}, ylabel near ticks, 
            title={Average SIIB per Constraint $D_0$}, scale only axis, 
            height=3cm, width=6cm, xmin=1, xmax=25, legend style={nodes={scale=0.5, transform shape}}] 
            \addplot [color=blue, error bars/.cd, y dir=both, y explicit]
            table [x expr=\thisrow{Q}, 
                    y expr=(\thisrow{target_reconstruction_siib_mean}), 
                    y error expr=(\thisrow{target_reconstruction_siib_std}), 
                    col sep=comma] 
                {data/constrained_evaluation_per_Q.csv};
            \addplot [color=red, error bars/.cd, y dir=both, y explicit]
            table [x expr=\thisrow{Q}, 
                    y expr=(\thisrow{target_result_siib_mean}), 
                    y error expr=(\thisrow{target_result_siib_std}), 
                    col sep=comma] 
                {data/constrained_evaluation_per_Q.csv};
            \legend{Bright Zone SIIB, Total SIIB};

            \nextgroupplot[xlabel={$D_0$}, ylabel={Distraction}, ylabel near ticks, 
            title={Average Distraction per Constraint $D_0$}, scale only axis, 
            height=3cm, width=6cm, xmin=1, xmax=25, legend style={nodes={scale=0.5, transform shape}}] 
            \addplot [color=red, error bars/.cd, y dir=both, y explicit]
            table [x expr=\thisrow{Q}, 
                    y expr=(\thisrow{reconstruction_leakage_distraction_mean}), 
                    y error expr=(\thisrow{reconstruction_leakage_distraction_std}), 
                    col sep=comma] 
                {data/constrained_evaluation_per_Q.csv};
            \legend{Distraction};

            \nextgroupplot[xlabel={$D_0$}, ylabel={Acoustic Contrast}, ylabel near ticks, 
            title={Average Acoustic Contrast per Constraint $D_0$}, scale only axis, 
            height=3cm, width=6cm, xmin=1, xmax=25, legend style={nodes={scale=0.5, transform shape}}] 
            \addplot [color=red, error bars/.cd, y dir=both, y explicit]
            table [x expr=\thisrow{Q}, 
                    y expr=(\thisrow{reconstruction_leakage_contrast_mean}), 
                    y error expr=(\thisrow{reconstruction_leakage_contrast_std}), 
                    col sep=comma] 
                {data/constrained_evaluation_per_Q.csv};
            \legend{Acoustic Contrast};
        \end{groupplot}
\end{tikzpicture}}
\caption{Plots depicting stuff.}
\end{figure}


\subsubsection*{Analyzing Algorithm Behavior}
\begin{figure}[p]
    \centering
    \makebox[\linewidth]{
    \begin{tikzpicture}
        \begin{groupplot}[group style={group size=2 by 4, vertical sep=2.25cm, horizontal sep=2.5cm}]

            \nextgroupplot[xlabel={time [samples]}, ylabel={$t^{(m)}[n]$}, ylabel near ticks, 
            title={Target Sound Pressure $m\in A$ }, scale only axis, 
            height=3cm, width=6cm, xmin=29000, xmax=31000, xtick={29000, 30000, 31000},
            ymin=-0.5, ymax=0.5]
            \addplot [color=red, error bars/.cd, y dir=both, y explicit]
            table [x expr=\thisrow{samples}, y expr=(\thisrow{audio} / 32787), col sep=comma] 
                {data/target_Q.csv};
            \addplot [samples=50, smooth, name path=H11] coordinates {(29300,-0.5) (29300,0.5)};
            \addplot [samples=50, smooth, name path=H12] coordinates {(30250,-0.5) (30250,0.5)};
            \addplot [fill opacity=0.05] fill between [of=H11 and H12];

            \nextgroupplot[xlabel={time [samples]}, ylabel={$t^{(m)}[n]$}, ylabel near ticks, 
            title={Target Sound Pressure $m\in B$}, scale only axis, 
            height=3cm, width=6cm, xmin=29000, xmax=31000, xtick={29000, 30000, 31000},
            ymin=-0.5, ymax=0.5]
            \addplot [color=red, error bars/.cd, y dir=both, y explicit]
            table [x expr=\thisrow{samples}, y expr=(\thisrow{audio} / 32787), col sep=comma] 
                {data/target_Q_offzone.csv};
            \addplot [samples=50, smooth, name path=H11] coordinates {(29750,-0.5) (29750,0.5)};
            \addplot [samples=50, smooth, name path=H12] coordinates {(30850,-0.5) (30850,0.5)};
            \addplot [fill opacity=0.05] fill between [of=H11 and H12];

            \nextgroupplot[xlabel={time [samples]}, ylabel={$p_\za^{(m)}[n]$}, ylabel near ticks, 
            title={Reconstruction for $Q=1$}, scale only axis, 
            height=3cm, width=6cm, xmin=29000, xmax=31000, xtick={29000, 30000, 31000},
            ymin=-0.5, ymax=0.5]
            \addplot [color=red, error bars/.cd, y dir=both, y explicit]
            table [x expr=\thisrow{samples}, y expr=(\thisrow{audio} / 32787), col sep=comma] 
                {data/reconstruction_Q_1.csv};
            \addplot [samples=50, smooth, name path=H11] coordinates {(29300,-0.5) (29300,0.5)};
            \addplot [samples=50, smooth, name path=H12] coordinates {(30250,-0.5) (30250,0.5)};
            \addplot [fill opacity=0.05] fill between [of=H11 and H12];

            \nextgroupplot[xlabel={time [samples]}, ylabel={$p_\zb^{(m)}[n]$}, ylabel near ticks, 
            title={Leakage for $Q=1$}, scale only axis, 
            height=3cm, width=6cm, xmin=29000, xmax=31000, xtick={29000, 30000, 31000},
            ymin=-0.5, ymax=0.5]
            \addplot [color=red, error bars/.cd, y dir=both, y explicit]
            table [x expr=\thisrow{samples}, y expr=(\thisrow{audio} / 32787), col sep=comma] 
                {data/leakage_Q_1.csv};
            \addplot [samples=50, smooth, name path=H11] coordinates {(29750,-0.5) (29750,0.5)};
            \addplot [samples=50, smooth, name path=H12] coordinates {(30850,-0.5) (30850,0.5)};
            \addplot [fill opacity=0.05] fill between [of=H11 and H12];

            \nextgroupplot[xlabel={time [samples]}, ylabel={$p_\za^{(m)}[n]$}, ylabel near ticks, 
            title={Reconstruction for $Q=7$}, scale only axis, 
            height=3cm, width=6cm, xmin=29000, xmax=31000, xtick={29000, 30000, 31000},
            ymin=-0.5, ymax=0.5]
            \addplot [color=red, error bars/.cd, y dir=both, y explicit]
            table [x expr=\thisrow{samples}, y expr=(\thisrow{audio} / 32787), col sep=comma] 
                {data/reconstruction_Q_7.csv};
            \addplot [samples=50, smooth, name path=H11] coordinates {(29300,-0.5) (29300,0.5)};
            \addplot [samples=50, smooth, name path=H12] coordinates {(30250,-0.5) (30250,0.5)};
            \addplot [fill opacity=0.05] fill between [of=H11 and H12];

            \nextgroupplot[xlabel={time [samples]}, ylabel={$p_\zb^{(m)}[n]$}, ylabel near ticks, 
            title={Leakage for $Q=7$}, scale only axis, 
            height=3cm, width=6cm, xmin=29000, xmax=31000, xtick={29000, 30000, 31000},
            ymin=-0.5, ymax=0.5]
            \addplot [color=red, error bars/.cd, y dir=both, y explicit]
            table [x expr=\thisrow{samples}, y expr=(\thisrow{audio} / 32787), col sep=comma] 
                {data/leakage_Q_7.csv};
            \addplot [samples=50, smooth, name path=H11] coordinates {(29750,-0.5) (29750,0.5)};
            \addplot [samples=50, smooth, name path=H12] coordinates {(30850,-0.5) (30850,0.5)};
            \addplot [fill opacity=0.05] fill between [of=H11 and H12];

            \nextgroupplot[xlabel={time [samples]}, ylabel={$p_\za^{(m)}[n]$}, ylabel near ticks, 
            title={Reconstruction for $Q=21$}, scale only axis, 
            height=3cm, width=6cm, xmin=29000, xmax=31000, xtick={29000, 30000, 31000},
            ymin=-0.5, ymax=0.5]
            \addplot [color=red, error bars/.cd, y dir=both, y explicit]
            table [x expr=\thisrow{samples}, y expr=(\thisrow{audio} / 32787), col sep=comma] 
                {data/reconstruction_Q_21.csv};
            \addplot [samples=50, smooth, name path=H11] coordinates {(29300,-0.5) (29300,0.5)};
            \addplot [samples=50, smooth, name path=H12] coordinates {(30250,-0.5) (30250,0.5)};
            \addplot [fill opacity=0.05] fill between [of=H11 and H12];

            \nextgroupplot[xlabel={time [samples]}, ylabel={$p_\zb^{(m)}[n]$}, ylabel near ticks, 
            title={Leakage for $Q=21$}, scale only axis, 
            height=3cm, width=6cm, xmin=29000, xmax=31000, xtick={29000, 30000, 31000},
            ymin=-0.5, ymax=0.5]
            \addplot [color=red, error bars/.cd, y dir=both, y explicit]
            table [x expr=\thisrow{samples}, y expr=(\thisrow{audio} / 32787), col sep=comma] 
                {data/leakage_Q_21.csv};
            \addplot [samples=50, smooth, name path=H11] coordinates {(29750,-0.5) (29750,0.5)};
            \addplot [samples=50, smooth, name path=H12] coordinates {(30850,-0.5) (30850,0.5)};
            \addplot [fill opacity=0.05] fill between [of=H11 and H12];
        \end{groupplot}
\end{tikzpicture}}
\caption{Plots depicting stuff.}
\end{figure}

