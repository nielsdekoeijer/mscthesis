In \autoref{ch:perceptual_sound_zone:perceptual_minimization} two perceptual sound zone algorithms are proposed. 
First, an unconstrained perceptual pressure matching approach where the detectability of the sound pressure 
errors is minimized.
Secondly, a constrained perceptual pressure matching approach which leverages the fact that the 
detectability has a consistent 
perceptual interpretation to constrain the detectability of the reproduction error.

This section will evaluate the results of the performed experiments for both proposed approaches.
To this end, \autoref{ch:results:results:unconstrained_results} the unconstrained perceptual pressure 
matching approach is evaluated, and in  \autoref{ch:results:results:constrained_results} the constrained perceptual pressure 
matching approach is evaluated.

In order to effectively describe various points in the room from the simulations, the 
control points numbering given by \autoref{fig:results:methodology:room} is used.

\subsection{Evaluating Unconstrained Perceptual Pressure Matching}
\label{ch:results:results:unconstrained_results}
In this section, the unconstrained perceptual pressure matching algorithm will be evaluated in accordance to the 
approach discussed in \autoref{ch:results:methodology}.
This is done by first evaluating the results of the simulations of the various setups 
quantitatively through the proposed measures.
From this, a conclusions is drawn which are then motivated qualitatively by reasoning about algorithm behavior
by considering waveforms generated by the investigated algorithms.

\subsubsection*{Quantitative Analysis of Simulation Results}
\begin{table}[]
    \sisetup{table-format=1.3(7), round-precision=3, table-comparator=true, round-integer-to-decimal=false, separate-uncertainty=true, table-align-text-post = true}
    \centering
    \begin{tabular}{lS[round-mode=places]S[round-mode=places]}
    \toprule
     \multicolumn{1}{c}{\textbf{Measure}} & 
     \textbf{\begin{tabular}[c]{@{}c@{}}Unconstrained Perceptual PM\\ Mean ($\pm$ 95\% CI)\end{tabular}} & 
     \textbf{\begin{tabular}[c]{@{}c@{}}Reference PM\\ Mean ($\pm$ 95\% CI)\end{tabular}} \\ 
    \midrule
    Total PESQ          & 3,154                  \pm 0,081              & 2,609                 \pm 0,084                \\ 
    Bright Zone PESQ    & 3,345                  \pm 0,087              & 4,107                 \pm 0,051                \\ 
    Total STOI          & 0,943                  \pm 0,003              & 0,940                 \pm 0,006                \\ 
    Bright Zone STOI    & 0,950                  \pm 0,003              & 0,989                 \pm 0,001                \\ 
    Total SIIB          & 1114,306               \pm 23,762             & 893,225               \pm 63,815               \\ 
    Bright Zone SIIB    & 1260,117               \pm 14,290             & 1311,041              \pm 12,333               \\ 
Total NMSE              & -4,929 \pm 0,235 $\mathrm{\hspace{-12pt}dB}$  & -13,529 \pm 0,856 $\mathrm{\hspace{-12pt}dB}$  \\ 
    Bright Zone NMSE    & -5,241 \pm 0,248 $\mathrm{\hspace{-12pt}dB}$  & -16,600 \pm 0,875 $\mathrm{\hspace{-12pt}dB}$  \\ 
    Distraction         & 7,828                  \pm 1,868              & 12,693  \pm 3,405                              \\ 
    Acoustic Contrast   & 13,258 \pm 0,379 $\mathrm{\hspace{-12pt}dB}$  & 16,075  \pm 0,936 $\mathrm{\hspace{-12pt}dB}$  \\ 
    \bottomrule
\end{tabular}
    \caption{Summary of the results of the evaluation of the unconstrained perceptual pressure matching approach 
        and the reference pressure matching approach using the evaluation metrics defined in 
        \autoref{ch:results:methodology}.}
    \label{tb:results:results:unconstrained_results}
\end{table}
In order to quantify the performance of the unconstrained perceptual pressure matching approach, the various 
measures introduced in \autoref{ch:results:methodology} are determined for all 12 simulations.
The measures are averaged over all simulations and over all control points.
For comparison purposes, the reference pressure matching approach is 
simulated and evaluated in an identical fashion.

The results of this experiment is summarized in \autoref{tb:results:results:unconstrained_results}.
This table depicts the mean and 95\% confidence interval of the measures taken 
over all 4 control points and all 6 unique experiments.
As discussed in \autoref{ch:perceptual:review}, it is important to note that the perceptual measures can only
be used as an indication, and further listening tests must be performed to draw any real conclusions.

From the results in the table, the following observations are made:
\begin{enumerate}
    \item The perceptual approach outperforms the reference in two perceptual measures evaluating the total experience:
        total PESQ and SIIB attain higher values for the perceptual approach.
        The perceptual approach also attains a higher value for STOI, but the confidence intervals indicate that 
        the values are too close to drawn any real conclusions. 

        Note that these measures evaluate the total experience, taking into consideration the total 
        sound pressure per control point including interference.
        This implies that the perceptual approach may result in an overall better perceptual experience.
        \label{obs:total_perceptual}

    \item The perceptual approach outperforms the reference in terms of the perceptual distraction measure,
        implying that the interference in the perceptual approach may be less distracting 
        \label{obs:distraction}

    \item The reference approach outperforms the perceptual approach in perceptual measures evaluating the 
        bright-zone quantities: bright zone PESQ, STOI and SIIB are all higher for the reference approach.
        Note that these measures are sans interference: they only evaluate how well the achieved sound pressure
        attains the target, ignoring interference.

        This implies that, disregarding interference, the reference approximates the target more effectively 
        perceptually.
        However, from \autoref{obs:total_perceptual} it is known that the total experience results in a 
        better quality of experience. 

        This implies that, although the reference algorithm approximates the target better perceptually, 
        the interference that it introduces a sufficient disturbance to be outperformed by the reference.
        \label{obs:bright_zone_perceptual}

    \item The reference approach outperforms the perceptual approach for all physical measures: 
        total and bright zone NMSE and acoustic contrast.
        This is to be expected, as the reference approach optimizes the NMSE directly.

        Interestingly, although the total NMSE is lower, the reference is outperformed in terms of all total perceptual 
        measures, as discussed in \autoref{obs:total_perceptual}.

        In addition to this, the acoustic contrast between intended and interfering sound pressure for 
        the reference approach is over twice as large than the perceptual approach.
        Nevertheless, the perceptual approach less distracting according to the distraction model as 
        discussed in \autoref{obs:distraction}.

        These results imply that NMSE or AC may not be optimal measures for the evaluation of the  
        perceptual experience of sound zones.
        \label{obs:physical}
\end{enumerate}

In summary, from the observations above it is concluded that the perceptual sound zone algorithm may
outperform the reference sound zone algorithm in terms of perceptual experience.
This seems to be due to the perceptually disturbing interference introduced by the reference algorithm.
As such, the perceptual algorithm seems to make a better perceptual trade-off between reproduction of the 
target sound pressure and suppression of the interference than the reference.

This is because, when disregarding noise, the reference algorithm has a better reproduction of the target perceptually.
However, when the noise is added, the reference algorithm gets outperformed by the perceptual approach.
The distraction ratings also indicate that the noise introduced by the reference algorithm is more distracting.

In addition to this, these results also imply that the physical metrics NMSE and AC do not correlate well 
with the perceptual measures, as the physical metrics predict the reference to outperform the perceptual approach.

\subsubsection*{Analyzing Algorithm Behavior}
\begin{figure}[]
    \centering
    \begin{figure}
    \centering
    \begin{tikzpicture}
        \begin{groupplot}[group style={group size=1 by 3, vertical sep=2cm}]
            \nextgroupplot[xlabel={$n$}, ylabel={$t^{(m)}[n]$}, ylabel near ticks, 
            title={Target}, scale only axis, 
            height=3cm, width=9cm, xmin=5000, xmax=6400, xtick={5000, 5200, 5400, 5600, 5800, 6000, 6200, 6400},
            ymin=-0.4, ymax=0.4]
            \addplot [color=red, error bars/.cd, y dir=both, y explicit]
            table [x expr=\thisrow{samples}, y expr=(\thisrow{audio} / 32787), col sep=comma] 
                {data/target.csv};

            \nextgroupplot[xlabel={$n$}, ylabel={$p_\zz^{(m)}[n]$}, ylabel near ticks, 
            title={Perceptual Leakage}, scale only axis, 
            height=3cm, width=9cm, xmin=5000, xmax=6400, xtick={5000, 5200, 5400, 5600, 5800, 6000, 6200, 6400},
            ymin=-0.4, ymax=0.4]
            \addplot [color=red, error bars/.cd, y dir=both, y explicit]
            table [x expr=\thisrow{samples}, y expr=(\thisrow{audio} / 32787), col sep=comma] 
                {data/per_leakage.csv};

            \nextgroupplot[xlabel={$n$}, ylabel={$p_\zz^{(m)}[n]$}, ylabel near ticks, 
            title={Reference Leakage}, scale only axis, 
            height=3cm, width=9cm, xmin=5000, xmax=6400, xtick={5000, 5200, 5400, 5600, 5800, 6000, 6200, 6400},
            ymin=-0.4, ymax=0.4]
            \addplot [color=red, error bars/.cd, y dir=both, y explicit]
            table [x expr=\thisrow{samples}, y expr=(\thisrow{audio} / 32787), col sep=comma] 
                {data/ref_leakage.csv};
        \end{groupplot}
        % \begin{axis}[xlabel=$n$, ylabel=$t$, ylabel near ticks, title={Target}, scale only axis, 
        %     height=5cm, width=14cm, xmin=5000, xmax=6400, xtick={5000, 5200, 5400, 5600, 5800, 6000, 6200, 6400},
        %     ymin=-0.4, ymax=0.4]
        %     \addplot [color=red, error bars/.cd, y dir=both, y explicit]
        %     table [x expr=\thisrow{samples}, y expr=(\thisrow{audio} / 32787), col sep=comma] 
        %         {data/target.csv};
        % \end{axis}
        % \begin{axis}[xlabel=$n$, ylabel=$t$, ylabel near ticks, title={Reference Leakage}, scale only axis, 
        %     height=5cm, width=14cm, xmin=5000, xmax=6400, xtick={5000, 5200, 5400, 5600, 5800, 6000, 6200, 6400},
        %     ymin=-0.4, ymax=0.4]
        %     \addplot [color=red, error bars/.cd, y dir=both, y explicit]
        %     table [x expr=\thisrow{samples}, y expr=(\thisrow{audio} / 32787), col sep=comma] 
        %         {data/ref_leakage.csv};
        % \end{axis}
        % \begin{axis}[xlabel=$n$, ylabel=$t$, ylabel near ticks, title={Perceptual Leakage}, scale only axis, 
        %     height=5cm, width=14cm, xmin=5000, xmax=6400, xtick={5000, 5200, 5400, 5600, 5800, 6000, 6200, 6400},
        %     ymin=-0.4, ymax=0.4]
        %     \addplot [color=red, error bars/.cd, y dir=both, y explicit]
        %     table [x expr=\thisrow{samples}, y expr=(\thisrow{audio} / 32787), col sep=comma] 
        %         {data/per_leakage.csv};
        % \end{axis}
    \end{tikzpicture}
\end{figure}

    \caption{Depiction of the wave forms of the target sound pressure and achieved dark zone sound pressure for the 
        unconstrained perceptual pressure matching approach and the reference pressure matching approach.}
        \label{fig:results:evaluation:unconstrained_results:behavior}
\end{figure}

In the preceding section the reference and perceptual algorithms are compared quantitatively.
Results indicate that the perceptual algorithm outperforms the reference in terms of the total experience.
This section considers the behavior of the algorithm in an attempt to explain these results.

To this end, consider \autoref{fig:results:evaluation:unconstrained_results:behavior}.
This figure contains a plot of the wave forms of the target sound pressure and achieved dark-zone sound pressure 
for both the perceptual and non-perceptual variants of pressure matching for control point $m=2$ 
(see: \autoref{fig:results:methodology:room}) from the experiments.

The selected control point is in zone $\za$. 
As explained in \autoref{ch:results:methodology}, the achieved dark-zone sound pressure can be understood 
as the interference due to another zone, in this case zone $\zb$.

Consider the highlighted region for the highlighted region for the perceptual algorithm.
From this it can be seen that the 
magnitude of the interference is correlated to the magnitude of the target sound pressure for zone $\za$.
Contrast this to the highlighted region for the reference algorithm, where the interference is a 
at a relatively constant level.

This may explain why, while having lower overall contrast, the perceptual approach outperforms the 
reference approach in terms of distraction and overall perceptual experience.
When determining the interference for control point $m=2$, the perceptual algorithm takes the 
target sound pressure for $\za$ into account. 
Effectively, when the target sound pressure is relatively loud, more interference is allowed 
as it is masked to a degree by the target sound pressure.

In doing so, the interference is less detectable and thus perceptually less disturbing, 
which serves as a possible explanation to the results given in \autoref{ch:results:results:unconstrained_results}

\subsection{Evaluating Constrained Perceptual Pressure Matching}
\label{ch:results:results:constrained_results}
This section details the evaluation of the constrained perceptual pressure matching algorithm discussed in 
\autoref{ch:perceptual_sound_zone:perceptual_minimization:constrained}.
As discussed, this algorithm minimizes the detectability of the leakage error whilst constraining the detectability of the 
reproduction error.

This leverages the fact that the Par detectability has a perceptual interpretation.
That is to say, if two disturbances result in the same detectability, this should mean that they are equally detectable 
perceptually.
As such, using the detectability of the reproduction errors in constraints could allow for more precise control over the 
quality of the reproduced audio.

Traditional pressure matching approaches such as the reference pressure matching algorithm can also attempt to constrain the 
reproduction error.
However, the reference pressure matching algorithm uses the mean square sound pressure error rather than detectability,
which does not always correlate well with perception.
That is to say, two mean square pressure errors can vary widely perceptually.

This section seeks to explore the degree of control that the reproduction error detectability constraints provide.
In this case, no comparison is made to the reference.
Instead, the measures defined in \autoref{ch:results:methodology} are used to determine the performance of the 
constrained perceptual pressure matching approach for varying constraint values.
Afterwards, a qualitative analysis of the algorithm is given through waveforms in order to motivate the quantitative results.

\subsubsection*{Quantifying Algorithm Performance}
\begin{figure}[p]
    \centering
    \noindent
    \makebox[\linewidth]{
    \begin{tikzpicture}
        % PESQ SIIB STOI MSE CONTRAST DISTRACTION
        \begin{groupplot}[group style={group size=2 by 3, vertical sep=2.25cm, horizontal sep=2.5cm}]
            \nextgroupplot[xlabel={$D_0$}, ylabel={PESQ}, ylabel near ticks, 
            title={Average PESQ per Constraint $D_0$}, scale only axis, 
            height=3cm, width=6cm, xmin=1, xmax=25, legend style={nodes={scale=0.5, transform shape}}] 
            \addplot [color=blue, error bars/.cd, y dir=both, y explicit]
            table [x expr=\thisrow{Q}, 
                    y expr=(\thisrow{target_reconstruction_pesq_mean}), 
                    y error expr=(\thisrow{target_reconstruction_pesq_std}), 
                    col sep=comma] 
                {data/constrained_evaluation_per_Q.csv};
            \addplot [color=red, error bars/.cd, y dir=both, y explicit]
            table [x expr=\thisrow{Q}, 
                    y expr=(\thisrow{target_result_pesq_mean}), 
                    y error expr=(\thisrow{target_result_pesq_std}), 
                    col sep=comma] 
                {data/constrained_evaluation_per_Q.csv};
            \legend{Bright Zone PESQ, Total PESQ};

            \nextgroupplot[xlabel={$D_0$}, ylabel={MSE}, ylabel near ticks, 
            title={Average MSE per Constraint $D_0$}, scale only axis, 
            height=3cm, width=6cm, xmin=1, xmax=25, legend style={nodes={scale=0.5, transform shape}}] 
            \addplot [color=blue, error bars/.cd, y dir=both, y explicit]
            table [x expr=\thisrow{Q}, 
                    y expr=(\thisrow{target_reconstruction_mse_mean}), 
                    y error expr=(\thisrow{target_reconstruction_mse_std}), 
                    col sep=comma] 
                {data/constrained_evaluation_per_Q.csv};
            \addplot [color=red, error bars/.cd, y dir=both, y explicit]
            table [x expr=\thisrow{Q}, 
                    y expr=(\thisrow{target_result_mse_mean}), 
                    y error expr=(\thisrow{target_result_mse_std}), 
                    col sep=comma] 
                {data/constrained_evaluation_per_Q.csv};
            \legend{Bright Zone MSE, Total MSE};

            \nextgroupplot[xlabel={$D_0$}, ylabel={STOI}, ylabel near ticks, 
            title={Average STOI per Constraint $D_0$}, scale only axis, 
            height=3cm, width=6cm, xmin=1, xmax=25, legend style={nodes={scale=0.5, transform shape}}] 
            \addplot [color=blue, error bars/.cd, y dir=both, y explicit]
            table [x expr=\thisrow{Q}, 
                    y expr=(\thisrow{target_reconstruction_stoi_mean}), 
                    y error expr=(\thisrow{target_reconstruction_stoi_std}), 
                    col sep=comma] 
                {data/constrained_evaluation_per_Q.csv};
            \addplot [color=red, error bars/.cd, y dir=both, y explicit]
            table [x expr=\thisrow{Q}, 
                    y expr=(\thisrow{target_result_stoi_mean}), 
                    y error expr=(\thisrow{target_result_stoi_std}), 
                    col sep=comma] 
                {data/constrained_evaluation_per_Q.csv};
            \legend{Bright Zone STOI, Total STOI};

            \nextgroupplot[xlabel={$D_0$}, ylabel={SIIB}, ylabel near ticks, 
            title={Average SIIB per Constraint $D_0$}, scale only axis, 
            height=3cm, width=6cm, xmin=1, xmax=25, legend style={nodes={scale=0.5, transform shape}}] 
            \addplot [color=blue, error bars/.cd, y dir=both, y explicit]
            table [x expr=\thisrow{Q}, 
                    y expr=(\thisrow{target_reconstruction_siib_mean}), 
                    y error expr=(\thisrow{target_reconstruction_siib_std}), 
                    col sep=comma] 
                {data/constrained_evaluation_per_Q.csv};
            \addplot [color=red, error bars/.cd, y dir=both, y explicit]
            table [x expr=\thisrow{Q}, 
                    y expr=(\thisrow{target_result_siib_mean}), 
                    y error expr=(\thisrow{target_result_siib_std}), 
                    col sep=comma] 
                {data/constrained_evaluation_per_Q.csv};
            \legend{Bright Zone SIIB, Total SIIB};

            \nextgroupplot[xlabel={$D_0$}, ylabel={Distraction}, ylabel near ticks, 
            title={Average Distraction per Constraint $D_0$}, scale only axis, 
            height=3cm, width=6cm, xmin=1, xmax=25, legend style={nodes={scale=0.5, transform shape}}] 
            \addplot [color=red, error bars/.cd, y dir=both, y explicit]
            table [x expr=\thisrow{Q}, 
                    y expr=(\thisrow{reconstruction_leakage_distraction_mean}), 
                    y error expr=(\thisrow{reconstruction_leakage_distraction_std}), 
                    col sep=comma] 
                {data/constrained_evaluation_per_Q.csv};
            \legend{Distraction};

            \nextgroupplot[xlabel={$D_0$}, ylabel={Acoustic Contrast}, ylabel near ticks, 
            title={Average Acoustic Contrast per Constraint $D_0$}, scale only axis, 
            height=3cm, width=6cm, xmin=1, xmax=25, legend style={nodes={scale=0.5, transform shape}}] 
            \addplot [color=red, error bars/.cd, y dir=both, y explicit]
            table [x expr=\thisrow{Q}, 
                    y expr=(\thisrow{reconstruction_leakage_contrast_mean}), 
                    y error expr=(\thisrow{reconstruction_leakage_contrast_std}), 
                    col sep=comma] 
                {data/constrained_evaluation_per_Q.csv};
            \legend{Acoustic Contrast};
        \end{groupplot}
\end{tikzpicture}}
\caption{Plots depicting stuff.}
\end{figure}

As mentioned, the goal is to analyse the behavior the constrained perceptual pressure matching algorithm
introduced in \autoref{ch:perceptual_sound_zone:perceptual_minimization:constrained} for various values of the 
constraint $D_0$ shown in \autoref{eq:perceptual_constrained}. 
To quantify the performance of the algorithm, the measures introduced in \autoref{ch:results:methodology} are used.

Consider \autoref{fig:results:results:constrained_results}, where the results of the 
experiments for the constrained perceptual pressure matching algorithm are depicted for different values of $D_0$. 
The measures as depicted are averaged over all 6 unique experiments and over all 4 points in the room. 
The error bars indicate the 95\% confidence intervals.

The following observations are made:
\begin{enumerate}
    \item It can be seen that the bright zone PESQ, NMSE, STOI and SIIB are all strictly decreasing or increasing as a 
        function of $D_0$.
        Recall that the bright zone quantities refer to the reproduction of the intended target sound pressures sans 
        interference.
        Thus, increasing the constraints correlates with a lowering of the quality of the reproduced target
        perceptually.
        \label{obs:constrained:1}

    \item It can be seen that the total PESQ, STOI and SIIB are not strictly decreasing functions of $D_0$.
        The total quantities refer to the perceptual quality all the sound pressure in the control points,
        including interference.
        Interestingly, the measures peak at a constraint value $D_0$ of about 3 or 5. 
        This effect is likely due to the interference, as it is shown in \autoref{obs:constrained:1} that 
        the achieved sound pressure sans interference is strictly decreasing.

        One explanation for this is as follows.
        Increasing the constraint value $D_0$ relaxes the optimization problem, 
        which should allow for further minimization of the leakage error detectability.
        For high values of $D_0$, the total and bright zone quantities converge to one another.
        This corresponds to the interference being so small that it is perceptually irrelevant.
        \label{obs:constrained:2}

    \item From the distraction plot, it can be seen that for lower constraint values the distraction decreases as 
        a function of $D_0$.
        At a constraint value of about 15, the distraction starts increasing again.
        % This implies that decreasing the amount of interference any further than this point does not result in 
        % further decrease in distraction.

        The rise in distraction can potentially be explained by a decrease in achieved bright zone sound pressure energy.
        One way of reducing the amount of interference is decreasing the bright zone sound energy.
        As such, the algorithm has an incentive to do so.
        Increasing constraint values $D_0$ allows for larger reproduction errors, which allow for a lower-energy
        representation of the achieved bright zone sound pressure.
        Thus, the increase in distraction may be due to a decrease in achieved bright zone sound pressure energy without
        a meaningful decrease of interference.

        This effect can also be observed through the mean acoustic contrast, however due to the size of the error bars are 
        it is difficult to draw conclusion based on these results.
        \label{obs:constrained:3}
\end{enumerate}
From the observations above, it is concluded that perceptually constrained pressure matching allows
for a degree of control of the perceived quality of the bright zone sound pressure.
This can be seen the small confidence intervals and 
from the observation in \autoref{obs:constrained:1} where it is shown that the bright zone measures are 
all a strictly decreasing functions of the constraint value $D_0$. 
However, as given by \autoref{obs:constrained:2}, the total perceived quality (including interference) is 
not a strictly decreasing function of $D_0$, so the constraint must be chosen carefully.

Finally, it is shown in \autoref{obs:constrained:3} that the distraction can be controlled to a degree through 
the constraint values $D_0$, as the distraction is a strictly decreasing function of $D_0$ for low constraint values. 
Higher constraint values seem to increase the distraction.
It is theorized that this is due to there being diminishing returns in increasing the constraint $D_0$, as the 
interference is not decreased in a perceptually meaningful way.
This can be seen from the convergence of the bright zone and the total perceptual measures as discussed in 
\autoref{obs:constrained:2}
Furthermore, the bright zone sound pressure energy is theorized to be decreased due to the further 
relaxation of the constraints.

\subsubsection*{Analyzing Algorithm Behavior}
\begin{figure}[p]
    \noindent
    \makebox[\linewidth]{
    \begin{tikzpicture}
        \begin{groupplot}[group style={group size=2 by 4, vertical sep=2.25cm, horizontal sep=2.5cm}]

            \nextgroupplot[xlabel={time [samples]}, ylabel={$t^{(m)}[n]$}, ylabel near ticks, 
            title={Target Sound Pressure $m\in A$ }, scale only axis, 
            height=3cm, width=6cm, xmin=29000, xmax=31000, xtick={29000, 30000, 31000},
            ymin=-0.5, ymax=0.5]
            \addplot [color=red, error bars/.cd, y dir=both, y explicit]
            table [x expr=\thisrow{samples}, y expr=(\thisrow{audio} / 32787), col sep=comma] 
                {data/target_Q.csv};
            \addplot [samples=50, smooth, name path=H11] coordinates {(29300,-0.5) (29300,0.5)};
            \addplot [samples=50, smooth, name path=H12] coordinates {(30250,-0.5) (30250,0.5)};
            \addplot [fill opacity=0.05] fill between [of=H11 and H12];

            \nextgroupplot[xlabel={time [samples]}, ylabel={$t^{(m)}[n]$}, ylabel near ticks, 
            title={Target Sound Pressure $m\in B$}, scale only axis, 
            height=3cm, width=6cm, xmin=29000, xmax=31000, xtick={29000, 30000, 31000},
            ymin=-0.5, ymax=0.5]
            \addplot [color=red, error bars/.cd, y dir=both, y explicit]
            table [x expr=\thisrow{samples}, y expr=(\thisrow{audio} / 32787), col sep=comma] 
                {data/target_Q_offzone.csv};
            \addplot [samples=50, smooth, name path=H11] coordinates {(29750,-0.5) (29750,0.5)};
            \addplot [samples=50, smooth, name path=H12] coordinates {(30850,-0.5) (30850,0.5)};
            \addplot [fill opacity=0.05] fill between [of=H11 and H12];

            \nextgroupplot[xlabel={time [samples]}, ylabel={$p_\za^{(m)}[n]$}, ylabel near ticks, 
            title={Reconstruction for $Q=1$}, scale only axis, 
            height=3cm, width=6cm, xmin=29000, xmax=31000, xtick={29000, 30000, 31000},
            ymin=-0.5, ymax=0.5]
            \addplot [color=red, error bars/.cd, y dir=both, y explicit]
            table [x expr=\thisrow{samples}, y expr=(\thisrow{audio} / 32787), col sep=comma] 
                {data/reconstruction_Q_1.csv};
            \addplot [samples=50, smooth, name path=H11] coordinates {(29300,-0.5) (29300,0.5)};
            \addplot [samples=50, smooth, name path=H12] coordinates {(30250,-0.5) (30250,0.5)};
            \addplot [fill opacity=0.05] fill between [of=H11 and H12];

            \nextgroupplot[xlabel={time [samples]}, ylabel={$p_\zb^{(m)}[n]$}, ylabel near ticks, 
            title={Leakage for $Q=1$}, scale only axis, 
            height=3cm, width=6cm, xmin=29000, xmax=31000, xtick={29000, 30000, 31000},
            ymin=-0.5, ymax=0.5]
            \addplot [color=red, error bars/.cd, y dir=both, y explicit]
            table [x expr=\thisrow{samples}, y expr=(\thisrow{audio} / 32787), col sep=comma] 
                {data/leakage_Q_1.csv};
            \addplot [samples=50, smooth, name path=H11] coordinates {(29750,-0.5) (29750,0.5)};
            \addplot [samples=50, smooth, name path=H12] coordinates {(30850,-0.5) (30850,0.5)};
            \addplot [fill opacity=0.05] fill between [of=H11 and H12];

            \nextgroupplot[xlabel={time [samples]}, ylabel={$p_\za^{(m)}[n]$}, ylabel near ticks, 
            title={Reconstruction for $Q=7$}, scale only axis, 
            height=3cm, width=6cm, xmin=29000, xmax=31000, xtick={29000, 30000, 31000},
            ymin=-0.5, ymax=0.5]
            \addplot [color=red, error bars/.cd, y dir=both, y explicit]
            table [x expr=\thisrow{samples}, y expr=(\thisrow{audio} / 32787), col sep=comma] 
                {data/reconstruction_Q_7.csv};
            \addplot [samples=50, smooth, name path=H11] coordinates {(29300,-0.5) (29300,0.5)};
            \addplot [samples=50, smooth, name path=H12] coordinates {(30250,-0.5) (30250,0.5)};
            \addplot [fill opacity=0.05] fill between [of=H11 and H12];

            \nextgroupplot[xlabel={time [samples]}, ylabel={$p_\zb^{(m)}[n]$}, ylabel near ticks, 
            title={Leakage for $Q=7$}, scale only axis, 
            height=3cm, width=6cm, xmin=29000, xmax=31000, xtick={29000, 30000, 31000},
            ymin=-0.5, ymax=0.5]
            \addplot [color=red, error bars/.cd, y dir=both, y explicit]
            table [x expr=\thisrow{samples}, y expr=(\thisrow{audio} / 32787), col sep=comma] 
                {data/leakage_Q_7.csv};
            \addplot [samples=50, smooth, name path=H11] coordinates {(29750,-0.5) (29750,0.5)};
            \addplot [samples=50, smooth, name path=H12] coordinates {(30850,-0.5) (30850,0.5)};
            \addplot [fill opacity=0.05] fill between [of=H11 and H12];

            \nextgroupplot[xlabel={time [samples]}, ylabel={$p_\za^{(m)}[n]$}, ylabel near ticks, 
            title={Reconstruction for $Q=21$}, scale only axis, 
            height=3cm, width=6cm, xmin=29000, xmax=31000, xtick={29000, 30000, 31000},
            ymin=-0.5, ymax=0.5]
            \addplot [color=red, error bars/.cd, y dir=both, y explicit]
            table [x expr=\thisrow{samples}, y expr=(\thisrow{audio} / 32787), col sep=comma] 
                {data/reconstruction_Q_21.csv};
            \addplot [samples=50, smooth, name path=H11] coordinates {(29300,-0.5) (29300,0.5)};
            \addplot [samples=50, smooth, name path=H12] coordinates {(30250,-0.5) (30250,0.5)};
            \addplot [fill opacity=0.05] fill between [of=H11 and H12];

            \nextgroupplot[xlabel={time [samples]}, ylabel={$p_\zb^{(m)}[n]$}, ylabel near ticks, 
            title={Leakage for $Q=21$}, scale only axis, 
            height=3cm, width=6cm, xmin=29000, xmax=31000, xtick={29000, 30000, 31000},
            ymin=-0.5, ymax=0.5]
            \addplot [color=red, error bars/.cd, y dir=both, y explicit]
            table [x expr=\thisrow{samples}, y expr=(\thisrow{audio} / 32787), col sep=comma] 
                {data/leakage_Q_21.csv};
            \addplot [samples=50, smooth, name path=H11] coordinates {(29750,-0.5) (29750,0.5)};
            \addplot [samples=50, smooth, name path=H12] coordinates {(30850,-0.5) (30850,0.5)};
            \addplot [fill opacity=0.05] fill between [of=H11 and H12];
        \end{groupplot}
\end{tikzpicture}}
\caption{Plots depicting stuff.}
\end{figure}

In the previous section, it is hypothesized that the perceptually constrained pressure matching approach does 
allow for accurate control of the perceived sound pressure.
This section explores the effects that increasing the value of $D_0$ have on the wave forms of the 
bright zone sound pressure and the dark zone sound pressure.

To this end, consider \autoref{fig:results:constrained_results:behavior}.
This plot depicts wave forms for control point $m=2$ in zone $\za$ for one of the experiments for different values 
of the constraint $D_0$.

As can be seen from the plots on the right hand side,
increasing the constraint value $D_0$ seems to reduce the interference.
Interestingly, one can see the frequency-weighting that occurs in Par detectability.
As discussed in \autoref{ch:perceptual:implementation:least_squares}, the Par detectability has a low perceptual weighting 
for lower frequencies due to the threshold of hearing.
As can be seen, for $D_0 = 1$ many high frequencies are still present in the achieved dark zone sound pressure.
For $D_0=21$, only lower frequencies remain. 

The achieved bright zone sound pressures on the left hand side provide evidence for the claim that the 
achieved bright zone sound pressure decreases with increasing constraints.
When compared to the target for that zone, the total energy present seems to decrease greatly between 
constraint values $D_0=1$ and $D_0=21$.  
