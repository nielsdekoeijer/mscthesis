In this section, the results obtained in this work are contrasted with other perceptual sound zone
approaches from literature.

In prior work by Lee et al. the signal-adaptive perceptual variable span trade-off (AP-VAST)
perceptual sound zone approach was proposed~\cite{lee2019towards, lee2020signal}.
Here, the existing variable span trade-off (VAST) sound zone framework is extended with a 
time-domain perceptual weighting filter.
The perceptual weighting is determined by the reciprocal of the masking curves of the 
target sound pressure~\cite{lee2020signal}.

As such, the approach by Lee et al. and the proposed approaches are similar.
The approach by Lee et al. however does not directly optimize over a perceptual model as is done in 
the proposed approach.
Therefore, the perceptual interpretation of the cost function, which enables the perceptually-motivated constraints
proposed in this work, 
may not be preserved when using AP-VAST. 

Similarly to this work, the AP-VAST framework was shown to outperform a reference pressure matching
and acoustic contrast control approach in terms of PESQ and STOI~\cite{lee2019towards}.
In addition to this, Lee et al. showed through a MUSHRA listening test that the perceptual approach had a 20\%
better performance than existing non-perceptual approaches~\cite{lee2020signal}.
Due to difference in setups, the approach by Lee et al. and the approaches proposed in this paper cannot be 
directly compared.
One way to effectively compare the two approaches is through listening tests.

In other work, Donley et al. showed how sound zones can be constructed by optimizing over the speech
intelligibility contrast (SIC) between zones to improve the speech 
privacy~\cite{donley2015multizone, donley2016improving, donley2018multizone}.
This is similar to the approach done in this work, as the sound zones are constructed by direct optimization
of a perceptual model.

Optimizing for speech privacy however also allows for the addition of white noise in order to increase the privacy.
As such, while the proposed approach and the approach by Donley et al. are similar mathematically, the outcomes are quite 
different and therefore difficult to compare.
