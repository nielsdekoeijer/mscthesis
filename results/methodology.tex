In \autoref{ch:sound_zone} and \autoref{ch:perceptual_sound_zone} various sound zone algorithms were defined.
First, in \autoref{ch:sound_zone} a non-perceptual pressure matching approach was introduced.
In the following chapter two perceptual sound zone algorithms were introduced, namely 
unconstrained perceptual pressure matching and perceptually constrained pressure matching.

This section will discuss the general approach for evaluation of the previously derived algorithms.
All algorithms will be evaluated by means of simulation. 
As all derived algorithms were found to be computationally intensive, many of the design consideration for the simulations
were chosen such to keep the computational complexity low.

\subsection{Room Setup}
All sound zone algorithms were evaluated in a simulated square room of 5 by 5 meters, with a ceiling height of 3.4 meters.
The room contained 4 loudspeakers placed in the corners of the room at a height of 1.2 meters.

Two zones each consisting of two control points are in the middle of the room.
In order to obtain a challenging problem, the zones are placed in close proximity, which the two closest points 0.5 meters apart.

\begin{figure}[]
    \centering
    \scalebox{1.0}{\begin{tikzpicture}
    \draw [draw=black] (0,0) rectangle (5,5);

    % Speakers on the top wall
    \pic[scale=0.7] at (1.00, 4.25) {Speaker};
    \pic[scale=0.7] at (4.00, 4.25) {Speaker};

    % Speakers on the bottom wall
    \pic[rotate=180, scale=0.7] at (1.00, 0.75) {Speaker};
    \pic[rotate=180, scale=0.7] at (4.00, 0.75) {Speaker};

    \draw[opacity=0.4, fill=blue, draw=white] (1.25, 2.00) circle[radius=0.15];
    \draw[opacity=0.4, fill=blue, draw=white] (2.25, 2.00) circle[radius=0.15];
    \draw[opacity=0.4, fill=red, draw=white] (3.75, 2.00) circle[radius=0.15];
    \draw[opacity=0.4, fill=red, draw=white] (2.75, 2.00) circle[radius=0.15];
\end{tikzpicture}
}
    \caption{The room setup used in the simulations for the evaluation of the algorithms.}
    \label{fig:results:methodology:room}
\end{figure}
An image depicting the entire setup is given in \autoref{fig:results:methodology:room}.

\subsection{Content Selection}
For the evaluation speech audio content was used.
The motivation for this is that it is computationally intensive to run the algorithms at higher sampling rate such as 48 kHz, and speech signals 
can be represented well at lower sampling rates.
In addition to this, the objective speech quality measures described in \autoref{ch:perceptual} were found to be more robust.
Many general audio quality measures were found to have little to no official openly available implementations.

A set of 4 speech signals is used for evaluation.
For each experiment, a speech signal was assigned to each zone.
All possible combinations of the speech signals were made, resulting in 12 total experiments.

\subsection{Simulation Outputs}
After running the simulations, all output will be written to file.
For ease of evaluation, the following sound pressures is available:
\begin{itemize}
    \item \textbf{Target Sound Pressure per Control Point $m$:}\\
        This is the desired sound pressure per control point $m$.
        This can be understood as the sound pressure that the algorithm attempts to attain.
        In the derivations of \autoref{ch:sound_zone}, this was denoted by $t^{(m)}[n]$.
    \item \textbf{Achieved Target Sound Pressure per Control Point $m$:}\\
        This is the target sound pressure achieved by the algorithm per control point $m$, excluding the interference coming from other zones.
        This quantity allows for the evaluation of the quality of intended sound pressure per control point, isolated from all interference.
        In the two zone case for any $m\in A$, this corresponds to $p_\za^{(m)}[n]$ in previous derivations.  
    \item \textbf{Achieved Interfering Sound Pressure per Control Point $m$:}\\
        This is the interference sound pressure achieved by the algorithm per control point $m$, isolated the target sound pressure that was intended for that control point.
        This allows for analysis of the interference in isolation from the intended sound pressure.
        Again for the two zone case for any $m\in A$, this corresponds to $p_\zb^{(m)}[n]$.  
    \item \textbf{Total Achieved Sound Pressure per Control Point $m$:}\\
        This is the total sound pressure per control point $m$ achieved by the algorithm.
        This can be understood as the sound pressure that would be experienced by a listener standing in the position of point $m$, 
        and is equal to the sum over the achieved target sound pressure and the achieved interfering sound pressure previously discussed.
        In the derivations for the two zone case, this corresponds to to $p_\za^{(m)}[n] + p_\zb^{(m)}[n]$ for any $m$. 
\end{itemize}
The achieved target sound pressure and the achieved interfering sound pressure can thought of as decompositions of the total achieved sound pressure.
This decomposition was found to be useful during evaluation.

\subsection{Evaluation Criteria}
In \autoref{ch:perceptual:review} a literature review into various perceptual grading models from literature was discussed.
As mentioned before, speech audio is used as the content for the sound zones.
As such, it makes sense to use metrics that specialize in speech for evaluation.

For this reason, one of the metrics that will be used is the Perceptual Evaluation of Speech Quality (PESQ).
As described in \autoref{ch:perceptual:review:objective}, PESQ is a metric which grades the quality of a degraded speech signal with respect to a reference speech signal.
The resulting quality grade will be between 0 and 5, where 5 is the highest obtainable grade.

PESQ will be used to evaluate two aspects of the result.
\begin{itemize}
    \item \textbf{PESQ of the Achieved Target Sound Pressure with respect to the Target Sound Pressure per Control Point $m$:}\\
        As described, the achieved target sound pressure is the sound pressure per control point $m$, without the interference due to the other zones. 

        This will henceforth referred to as the \textbf{``Target PESQ''}.
        This quantity thus describes the quality of the intended sound pressure in isolation from interference.
    \item \textbf{PESQ of the Total Achieved Sound Pressure with respect to the Target Sound Pressure per Control Point $m$:}\\
        As previously described, the total achieved sound pressure is the sound pressure that would be audible to an individual present at control point $m$.  
        This includes the intended sound pressure, but also the interference coming from other zones.

        This quantity will be referred to as the \textbf{``Total PESQ''} and represents the quality of the total experience per control point $m$. 
\end{itemize}

Another metric that will be used for evaluation is the Distraction model also introduced in \autoref{ch:perceptual:review:objective}.
This model grades how distracting an interferer is in presence of target audio.
The grade uses a scale from 0 to 100, where 100 is considered maximally distracting.

The distraction will be used as follows:
\begin{itemize}
    \item \textbf{Distraction of the Achieved Interfering Sound Pressure with respect to the Achieved Target Sound Pressure per Control Point $m$:}\\
        This quantifies how distracting the interfering sound pressure is when listening to the indented sound pressure per control point $m$. 
        This is a way of quantifying how disturbing the interference of the resulting algorithms is.

        This will simply be referred to as the \textbf{``Distraction''} per control point from now on.
\end{itemize}
