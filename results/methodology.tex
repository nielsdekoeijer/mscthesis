This section discusses the general approach for evaluation of the perceptual sound zone algorithms proposed in 
\autoref{ch:perceptual_sound_zone:perceptual_minimization:unconstrained} and \autoref{ch:perceptual_sound_zone:perceptual_minimization:constrained}.
All algorithms will be evaluated by means of simulation. 
As all derived algorithms were found to be computationally intensive, many of the parameter considerations 
for the simulations are chosen such that the computational load is kept feasible.

For the evaluation, the constrained and unconstrained perceptual pressure matching are compared to the 
short-time frequency pressure matching approach
given in \autoref{ch:perceptual_sound_zone:stft} by \autoref{eq:perceptual_sound_zone:stft:stft_pm}.
As all approaches are pressure matching approaches, doing so allows for a comparison between a 
perceptual and a non-perceptual sound zone algorithm.

\autoref{ch:results:methodology} is structured as follows.
First, \autoref{ch:results:methodology:configuration} discusses and motivate the configuration used in the simulations.

\subsection{Simulation Configuration}
\label{ch:results:methodology:configuration}
All sound zone algorithms are evaluated in a simulated square room of 5 by 5 meters, with a ceiling height of 3.4 meters.
Two zones each consisting of two control points are in the middle of the room.
In order to obtain a sufficiently challenging problem, the zones are placed in close proximity, which the two closest points 0.5 meters apart.
This is important, as a trivial problem results makes it difficult to highlight the differences in performance.

The room contains 4 loudspeakers placed in the corners of the room at a height of 1.2 meters.
Omnidirectional loudspeakers which radiate energy equally in all directions are used~\cite{habets2006room}.
An image depicting the entire setup is given in \autoref{fig:results:methodology:room}.

\begin{figure}[]
    \centering
    \scalebox{1.0}{\begin{tikzpicture}
    \draw [draw=black] (0,0) rectangle (5,5);

    % Speakers on the top wall
    \pic[scale=0.7] at (1.00, 4.25) {Speaker};
    \pic[scale=0.7] at (4.00, 4.25) {Speaker};

    % Speakers on the bottom wall
    \pic[rotate=180, scale=0.7] at (1.00, 0.75) {Speaker};
    \pic[rotate=180, scale=0.7] at (4.00, 0.75) {Speaker};

    \draw[opacity=0.4, fill=blue, draw=white] (1.25, 2.00) circle[radius=0.15];
    \draw[opacity=0.4, fill=blue, draw=white] (2.25, 2.00) circle[radius=0.15];
    \draw[opacity=0.4, fill=red, draw=white] (3.75, 2.00) circle[radius=0.15];
    \draw[opacity=0.4, fill=red, draw=white] (2.75, 2.00) circle[radius=0.15];
\end{tikzpicture}
}
    \caption{The square room setup used in the simulations for the evaluation of the algorithms.
    Two zones each consisting of two control points are depicted in blue and red. 
    Four loudspeakers are placed in the corners of the room.}
    \label{fig:results:methodology:room}
\end{figure}

In order to synthesize room impulse responses used to relate the loudspeaker inputs and the resulting sound pressure in the control points,
the image method~\cite{allen1979image} is used.
The implementation by E. Habets of the image method is used~\cite{habets2006room}.

For reasons of computational complexity, the simulated room impulse responses is limited to reverberation times of a maximum of 200
milliseconds.
The definition of the reverberation time required for the intensity of reflections to reduce 60 dB relative 
to the direct path sound pressure~\cite{habets2006room}. 
To put the reverberation time used in the experiment into context, it has been found in a investigation into the reverberation time of furnished rooms
that similarly sized furnished rooms have an average maximal reverberation time of 720 milliseconds~\cite{diaz2005reverberation}.

To define the target sound pressure, content must be selected for the two zones.
For this evaluation, speech content is used.
The motivation for this is that the objective speech quality measures described in \autoref{ch:perceptual} are found to be more robust than
the discussed general audio quality measures.
In addition to this, it was found that many general audio quality measures have 
little to no free and openly available implementations.

The speech signals are down sampled to 8000 Hz.
The motivation for this is that it is computationally intensive to run the algorithms at higher sampling rate.

In total, a dataset of 4 loudness-matched speech signals is used for evaluation.
For each experiment, one speech signal is assigned to each zone.
All possible combinations of the speech signals are formed, resulting in a total of 12 possible configurations.

\subsection{Objective Measures for Result Evaluation}
This section will motivate the evaluation criteria used to evaluate the results of the experiment.
The perceptual measures PESQ, STOI, SIIB, and Distraction which are discussed in 
the the a literature review from \autoref{ch:perceptual:review} are used in the evaluation.

In addition to these perceptual measures, two traditional physical measures, namely the ``mean square error'' (MSE) and
the ``acoustic contrast'' (AC) in terms of sound pressure are used in the evaluation of 
sound zone algorithms~\cite{lee2020unified} are also used.

The motivation for including physical measures is to show in \autoref{ch:results:evaluation} that, 
while typically outperforming in terms of perceptual measures, the perceptual sound zone algorithms do not 
outperform the traditional reference sound zone algorithm. 

Before discussing the evaluation measures, a brief summary is given of the outputs and inputs of the sound zone algorithm.
\begin{itemize}
    \item \textbf{Target Sound Pressure $t^{(m)}$:}\\
        The desired sound pressure per control point $m$, defined by the speech signals 
        corresponding to the zone $\zz$ the control point is in.
    \item \textbf{Achieved Sound Pressure $p_\zz^{(m)}$:}\\
        The sound pressure achieved by the algorithm at control point control point $m$ for a zone $\zz$.
        This sound pressure has two different interpretations, depending on the control point under consideration.
        \begin{itemize}
            \item \textbf{Achieved Bright Zone Sound Pressure for Zone $\zz$:}\\
                When $m \in Z$, the achieved sound pressure represents how well the target sound pressure 
                intended for zone $\zz$, namely $t^{(m)}$,  is attained in control point.
            \item \textbf{Achieved Dark Zone Sound Pressure for Zone $\zz$:}\\
                When $m \notin Z$, the achieved sound pressure represents the sound pressure 
                into other zones due to reproducing the bright zone in points $m \in Z$.
                It can be understood as the leakage.
        \end{itemize}
    \item \textbf{Total Achieved Sound Pressure $\sum_\zz p_\zz^{(m)}$:}\\
        The sound pressure that results in a control point $m$ due to contributions of all zones $\zz$.
        This represents the sound pressure that a user of the sound zone system would experience.
\end{itemize}
For more information on these quantities, the reader is referred to \autoref{ch:sound_zone:data_model}, where they 
are introduced.
What follows is a description of various categories of evaluation measures.

\subsubsection{Perceptual Quality Measures}
One of the metrics that is used is the Perceptual Evaluation of Speech Quality (PESQ).
As described in \autoref{ch:perceptual:review:objective}, PESQ is a metric which grades the quality of a 
degraded speech signal with respect to a reference speech signal.
The resulting quality grade will be between 0 and 5, where 5 is the highest obtainable grade.

Another set of metrics that will be used are two speech intelligibility metrics: 
the Short-Time Objective Intelligibility (STOI) and 
the Speech Intelligibility in Bits (SIIB).
STOI provides an intelligibility score between 0 and 1, 1 being the highest.
SIIB instead scores the intelligibility with an information rate given in bits/s, lower-bounded by 0 bits/s.
A maximum intelligibility corresponds to a rate of about 150 bits/s~\cite{van2017instrumental}.

All perceptual measures evaluate the quality of a ``degraded'' input stimuli with respect to a ''reference'' stimuli.
Using the previously introduced quantities, the measures are to evaluate sound zone performance as follows:
\begin{itemize}
    \item \textbf{PESQ, STOI and SIIB of the Total Achieved Sound Pressure with 
        respect to the Target Sound Pressure:}\\
        Corresponds to the quality / intelligibility of the achieved sound pressure,
        including interference, and will be referred 
        to as the \textbf{``total PESQ''}, \textbf{``total STOI''} and the \textbf{``total SIIB''} per control point $m$. 
    \item \textbf{PESQ, STOI and SIIB of the Achieved Bright Zone Sound Pressure with 
        respect to the Target Sound Pressure:}\\
        Corresponds to the quality / intelligibility of the achieved sound pressure sans interference.
        This quantity is referred to as the \textbf{``bright zone PESQ''}, \textbf{``bright zone STOI''} and \textbf{``bright zone SIIB''}
        per control point $m$ for zone~$\zz$.
\end{itemize}

\subsubsection{Perceptual Interference Measure}
Another metric that will be used for evaluation is the Distraction model also introduced in \autoref{ch:perceptual:review:objective}.
This model grades how distracting an interferer is in presence of target audio.
The grade uses a scale from 0 to 100, where 100 is considered maximally distracting.

The distraction will be used as follows:
\begin{itemize}
    \item \textbf{Distraction of the achieved dark Zone Sound Pressure with respect to the Achieved Bright 
        Zone Sound Pressure:}\\
        This quantifies how distracting the dark zone sound pressures are when listening to the bright zone 
        sound pressure per control point $m$. 
        This will simply be referred to as the \textbf{``Distraction''} per control point $m$.
\end{itemize}

\subsubsection{Physical Measures}
One physical measure is the acoustic contrast (AC) between the 
achieved bright zone sound pressure and the achieved dark zone sound pressure can be used as a non-perceptual measure of interference.
Initially introduced in \autoref{ch:sound_zone:approaches}, 
the acoustic contrast between two time-domain sequences $x[n]\in\Real{N}$
and $y[n]\in\Real{N}$ is given as follows: 
\begin{equation}
    \mathrm{AC}(x,y) = \frac{\norm[2][2]{x[n]}}{\norm[2][2]{y[n]}} 
\end{equation}
Another non-perceptual metric that can be used to evaluate the quality of the result is the mean square error (MSE).
The MSE between two time-domain sequences $x[n]$ and $y[n]$ is given as:
\begin{equation}
    \text{MSE}(x,y) = \frac{1}{N}\norm[2][2]{y[n] - x[n]} 
\end{equation}

These physical measures are used to evaluate sound zone performance as follows:
\begin{itemize}
    \item \textbf{MSE of the Total Achieved Sound Pressure with respect to the Target Sound Pressure:}\\
        Describes the MSE between the target and achieved sound pressure, and will be referred to as the 
        \textbf{``total MSE}.
    \item \textbf{MSE of the Achieved Bright Zone Sound Pressure with respect to the Target Sound Pressure:}\\
        This is the MSE between the target and the achieved sound pressure sans interference, 
        and will henceforth referred to as the \textbf{``bright zone MSE''} per control point $m$.
    \item \textbf{Acoustic contrast between the Achieved Bright Zone Sound Pressure and the 
        Achieved Dark Zone Sound Pressure:}\\
        Quantifies the ratio of the acoustic potential energy of the bright zone and of the dark zone.
        From here on referred to as the \textbf{``Acoustic Contrast''} per control point $m$.
\end{itemize}
