\chapter{Introduction}

This is the introduction of the demonstration thesis for Computer Engineering students. It is written in \LaTeX\ which makes life easy for us. In the introduction you usually introduce the main problem you solve in the following chapters, and present the outline of the thesis. 

\begin{figure}%
\centering%
\includegraphics{style/cas}%
\caption{The Circuits and Systems Logo}%
\label{sample figure}%
\end{figure}

\section{The Main Problem}

The main problem was coming up with enough text for this demo, so you get a good idea of the possibilities. For instance, you can create chapters with nice elaborate headings. When you want to create some structure in your thesis, you can also divide your chapters into sections and subsections and even subsubsections, which all have progressively less elaborate headings.

\subsection{A Subsection}

This subsections serves no purpose other than showing off its header style. In general it makes little sense to split up a section into a single subsection. 

\subsection{The Other Subsection}

So, not to make a fool of myself, is simply add another subsection. Thankfully this subsection is not very long, we do not want to complicate things too much, do we? I think you get the idea now anyway, you can probably guess what happens when you type create a subsubsection\ldots

\section{Outline}

The outline of this thesis is very simple, since it does not really have a proper subject to cover. Instead, it demonstrates some of the possibilities of the \texttt{ce} class, and of \LaTeX\ in general. 

I just noticed that I have not demonstrated the formatting for paragraphs yet. You create a paragraph by leaving a blank line between sentences. If you do not like the standard way \LaTeX handles new paragraphs (\ie with indentation and no empty space between them), try the following:
\begin{verbatim}
\setlength{\parindent}{0pt}
\setlength{\parskip}{1ex plus 0.5ex minus 0.2ex}
\end{verbatim}
This removes the indentation and add a little space between them. Be careful: this also affects the formatting of the Table of Contents!
