\documentclass[aspectratio=169]{beamer}
% I sure do use alot of useless packages
\usepackage[english]{babel}
\usepackage{csquotes}
\usepackage{calc}
\usepackage[absolute,overlay]{textpos}
\usepackage{graphicx}
\usepackage{xcolor}
\usepackage{subfig}
\usepackage{amsmath}
\usepackage{amsfonts}
\usepackage{amsthm}
\usepackage{mathtools}
\usepackage{comment}
\usepackage{MnSymbol,wasysym}
\usepackage{textcomp}
\usepackage{hyperref}
\usepackage{multimedia}
\usepackage[]{booktabs} % For \toprule, \midrule and \bottomrule
\usepackage[round-mode=places, round-integer-to-decimal, round-precision=2,
    table-format = 1.2, 
    table-number-alignment=center,
    round-integer-to-decimal,
    output-decimal-marker={.}
    ]{siunitx}

% Niels custom packages!
\usepackage{nielstikz}
\usepackage[]{tikz-3dplot}
\usepackage[]{pgfplots}
\usepackage{blox}
\usetikzlibrary{arrows}
\usetikzlibrary{circuits}

\pgfplotsset{compat=1.16}
\pgfplotsset{scaled x ticks=false}
\usetikzlibrary{pgfplots.groupplots}
\usetikzlibrary{pgfplots.fillbetween}
\usetikzlibrary{patterns}
\usetikzlibrary{external}
\tikzexternalize

\usepackage{nielstex}

% =Definitions===========================================================================
% Thesis specific definitions
\newcommand{\za}{\mathcal{A}}
\newcommand{\zb}{\mathcal{B}}
\newcommand{\zz}{\mathcal{Z}}

% Set theme
\setbeamertemplate{navigation symbols}{} % remove navigation symbols
\mode<presentation>{\usetheme{tud}}

% Bibliography settings 
\usepackage[backend=bibtex,giveninits=true,maxnames=30,maxcitenames=20,url=false,style=authoryear]{biblatex}
\bibliography{bibliography}
\setlength\bibitemsep{0.3cm} % space between entries in the reference list
\renewcommand{\bibfont}{\normalfont\scriptsize}
\setbeamerfont{footnote}{size=\tiny}
\renewcommand{\cite}[1]{\footnote<.->[frame]{\fullcite{#1}}}

% Front page settings
\title{\textbf{Sound Zones with a Cost Function based on Human Hearing\\{\normalsize MSc Thesis Defence}}}
\institute[]{Delft University of Technology, The Netherlands\\
Bang and Olufsen, Denmark}
\author{Niels de Koeijer}

\begin{document}

% Introduction Slide
{\setbeamertemplate{footline}{\usebeamertemplate*{minimal footline}} \frame{\titlepage}}

% Part I: introduction to the project
\begin{frame}{\textbf{Preface:}\\ About}
    % Who I am, what, why
    \begin{columns}[c]
        \column{.60\textwidth}
        \textbf{Niels de Koeijer}\\
        Master Student Thesis at the Research Department at Bang \& Olufsen and the Delft University of Technology\\
        \vspace{0.60cm}
        This presentation will detail the work done during my MSc thesis.
        \column{.33\textwidth}
        \begin{figure}
            \vspace{-0.8cm}
            \centering
            \includegraphics[scale=0.17]{me.jpg}
        \end{figure}
    \end{columns}
\end{frame}

\begin{frame}{\textbf{Preface:}\\ The Sound Zone Approach}
    \begin{columns}[c]
        \column{.44\textwidth}
        The sound zone approach:
        \begin{itemize}
            \item \textbf{Given:}\\ A room, an array of loudspeakers, and a number of zones.
            \item \textbf{Goal:}\\  Reproduction distinct audio in the specified zones, with minimal interference.
        \end{itemize}
        \column{.44\textwidth}
        \begin{figure}[]
            \centering
            \scalebox{0.7}{\begin{tikzpicture}
    \tikzset{
      Speaker/.pic={
        \filldraw[fill=gray!40,pic actions] 
        (-15pt,0) -- 
          coordinate[midway] (-front) 
        (15pt,0) -- 
        ++([shift={(-6pt,8pt)}]0pt,0pt) coordinate (aux1) -- 
        ++(-18pt,0) coordinate (aux2) 
        -- cycle 
        (aux1) -- ++(0,6pt) -- coordinate[midway] (-back) ++(-18pt,0) -- (aux2);
      }
    }

    \draw [draw=black] (0,0) rectangle (8,6);

    % Speakers on the top wall
    \pic[scale=0.7] at (2, 5.6) {Speaker};
    \pic[scale=0.7] at (3, 5.6) {Speaker};
    \pic[scale=0.7] at (4, 5.6) {Speaker};
    \pic[scale=0.7] at (5, 5.6) {Speaker};
    \pic[scale=0.7] at (6, 5.6) {Speaker};

    % Speakers on the bottom wall
    \pic[rotate=180, scale=0.7] at (2, 0.4) {Speaker};
    \pic[rotate=180, scale=0.7] at (3, 0.4) {Speaker};
    \pic[rotate=180, scale=0.7] at (4, 0.4) {Speaker};
    \pic[rotate=180, scale=0.7] at (5, 0.4) {Speaker};
    \pic[rotate=180, scale=0.7] at (6, 0.4) {Speaker};

    \draw[opacity=0.4, fill=blue] (6,3) circle[radius=1.5];
    \draw[thick] (6,3) circle (1.5) node[align=center] {\textbf{Zone B:}\\Music};
    \draw[opacity=0.4, fill=red]  (2,3) circle[radius=1.5];
    \draw[thick] (2,3) circle (1.5) node[align=center] {\textbf{Zone A:}\\Movie};
\end{tikzpicture}
}
        \end{figure}
    \end{columns}
\end{frame}

\begin{frame}{\textbf{Preface:}\\ Introducing Perceptual Sound Zones}
    In order to improve sound zones, one recent approach is to include a model of the human auditory system
    which models how sound is perceived by humans, in the algorithms.\\
    \vspace{10pt}
    The motivation for this approach is as follows:
    \begin{itemize}
        \item Sound zone algorithms typically optimize over \textbf{physical measures} such as sound pressure,
            which do not always correspond well with how sound is actually perceived.
        \item By optimizing over \textbf{perceptual measure} instead, we are optimizing over what matters perceptually.
    \end{itemize}
\end{frame}

\begin{frame}{\textbf{Preface:}\\ Objectives \& Research Questions}
    The goal of the project is to investigate the construction and benefits of including auditory 
    perceptual information in sound zone algorithms.\\
    \vspace{10pt}
    To this end, two research questions are posed: 
    \begin{enumerate}
        \item {\textit{``How can auditory perceptual models be included in sound~zone~algorithms?''}}
        \item {\textit{``What are benefits of including auditory perceptual models in 
            sound zone algorithms?''}}
    \end{enumerate}
\end{frame}

% Structure slide
\begin{frame}{\textbf{Structure:}\\ Answering Research Questions}
    \begin{enumerate}
        \item {\textit{``How can auditory perceptual models be included in sound~zone~algorithms?''}}
            \vspace{7pt}
            \begin{enumerate}
                \item Determination of a suitable perceptual model for sound zone algorithms.
                \vspace{7pt}
                \item Proposal of a perceptual sound zone framework using determined model. 
                \vspace{7pt}
                \item Proposal of perceptual sound zone algorithms through proposed framework.
                \vspace{7pt}
            \end{enumerate}
        \item {\textit{``What are benefits of including auditory perceptual models in sound zone algorithms?''}}
            \vspace{-5pt}
            \begin{enumerate}
                \item Simulation and analysis of proposed perceptual sound zone algorithms.
            \end{enumerate}
    \end{enumerate}
\end{frame}

\begin{frame}{\textbf{Structure:}\\ Answering Research Questions}
    \begin{enumerate}
        \item {\textit{``How can auditory perceptual models be included in sound~zone~algorithms?''}}
            \vspace{7pt}
            \begin{enumerate}
                \item \textbf{Determination of a suitable perceptual model for sound zone algorithms.}
                \vspace{7pt}
                \item Proposal of a perceptual sound zone framework using determined model. 
                \vspace{7pt}
                \item Proposal of perceptual sound zone algorithms through proposed framework.
                \vspace{7pt}
            \end{enumerate}
        \item {\textit{``What are benefits of including auditory perceptual models in sound zone algorithms?''}}
            \vspace{-5pt}
            \begin{enumerate}
                \item Simulation and analysis of proposed perceptual sound zone algorithms.
            \end{enumerate}
    \end{enumerate}
\end{frame}

\begin{frame}{\textbf{Determining a Suitable Perceptual Model:}\\ Approach}
    In order to obtain a suitable perceptual model, various perceptual models from literature 
    were considered.\\
    \vspace{10pt}
    \begin{itemize}
        \item Mainly considered were algorithms that assign a perceptual ``score'' to audio.
        \item These could be used to propose sound zone algorithms that optimize over this score.
    \end{itemize}
\end{frame}

\begin{frame}{\textbf{Determining a Suitable Perceptual Model:}\\ Literature Review}
    To this end, two categories of perceptual models were considered:
        \begin{itemize}
            \item \textbf{Objective Audio Measures:} Perceptual models that seek to predict the outcomes of 
                listening tests, e.g. PESQ, PEAQ, Distraction, and STOI.
            \item \textbf{Audio Coding Models:} Perceptual models that are used to make the quantization noise
                introduced by audio compression as minimally disturbing as possible, e.g. the MPEG perceptual models.
        \end{itemize}
\end{frame}

\begin{frame}{\textbf{Determining a Suitable Perceptual Model:}\\ Introduction to the Par Distortion Detectability}
    % Intuition of detectability
    From this review, the ``Par distortion detectability''\cite{van2005perceptual} is selected as the most promising model because of its
    ease of integration into optimization problems.\\
    \vspace{10pt}
    The Par distortion detectability defines a mathematical function $D(x[n], \varepsilon[n])$ which models how easily a human 
    listener can detect the disturbance signal $\varepsilon[n]$ in presence of the masking signal $x[n]$.\\
    \vspace{10pt}
    The detectability $D(x[n], \varepsilon[n])$ assigns a number between 0 and $\infty$, 0 representing that $\varepsilon[n]$ being 
    completely undetectable in presence of $x[n]$. 
\end{frame}

\begin{frame}{\textbf{Determining a Suitable Perceptual Model:}\\ Perceptual Background for the Par Distortion Detectability}
    The detectability $D(x[n], \varepsilon[n])$ determines the detectability of $\varepsilon[n]$ by making use of two principals: the threshold of hearing 
    and the masking threshold of the masking signal $x[n]$.
    \vspace{10pt}
    \begin{enumerate}
        \item \textbf{Hearing Threshold:} The sound levels as a function of frequency below which humans cannot perceive sound.
            \begin{itemize}
                \item E.g. if the sound is below this threshold, it is not detectable.
            \end{itemize}
        \vspace{10pt}
        \item \textbf{Masking Threshold:} The sound levels required for another sound to be audible in presence of $x[n]$.
            \begin{itemize}
                \item E.g. if $x[n]$ is loud tone, then it will overpower similar tones and they will not be audible.
            \end{itemize}
    \end{enumerate}
\end{frame}

% \begin{frame}{\textbf{Determining a Suitable Perceptual Model:}\\ Mathematical Implementation}
%     % How is it implemented
%     The Par detectability distortion $D(x[n], \varepsilon[n])$ is especially promising because it is convex in $\varepsilon[n]$ if $x[n]$ is held fixed.
%     Convexity is a desirable property for optimization.\\
%     \vspace{10pt}
%     The detectability is implemented as follows:
%     \begin{equation}
%         D(x[n], \varepsilon[n]) = \norm[2][2]{W_x[k]\mathcal{E}[k]}.
%     \end{equation}
%     \vspace{-5pt}
%     \begin{itemize}
%         \item $\mathcal{E}[k]$ is the \textbf{frequency domain representation} of the disturbance signal $\varepsilon[n]$. 
%         \vspace{5pt}
%         \item $W_x[k]$ is a \textbf{perceptual weighting} determined in part by the masking properties of masking signal $x[n]$. 
%     \end{itemize}
% \end{frame}

\begin{frame}{\textbf{Structure:}\\ Answering Research Questions}
    \begin{enumerate}
        \item {\textit{``How can auditory perceptual models be included in sound~zone~algorithms?''}}
            \vspace{7pt}
            \begin{enumerate}
                \item \textbf{Determination of a suitable perceptual model for sound zone algorithms.}
                \vspace{7pt}
                \item Proposal of a perceptual sound zone framework using determined model. 
                \vspace{7pt}
                \item Proposal of perceptual sound zone algorithms through proposed framework.
                \vspace{7pt}
            \end{enumerate}
        \item {\textit{``What are benefits of including auditory perceptual models in sound zone algorithms?''}}
            \vspace{-5pt}
            \begin{enumerate}
                \item Simulation and analysis of proposed perceptual sound zone algorithms.
            \end{enumerate}
    \end{enumerate}
\end{frame}

\begin{frame}{\textbf{Structure:}\\ Answering Research Questions}
    \begin{enumerate}
        \item {\textit{``How can auditory perceptual models be included in sound~zone~algorithms?''}}
            \vspace{7pt}
            \begin{enumerate}
                \item Determination of a suitable perceptual model for sound zone algorithms.
                \vspace{7pt}
                \item \textbf{Proposal of a perceptual sound zone framework using determined model. }
                \vspace{7pt}
                \item Proposal of perceptual sound zone algorithms through proposed framework.
                \vspace{7pt}
            \end{enumerate}
        \item {\textit{``What are benefits of including auditory perceptual models in sound zone algorithms?''}}
            \vspace{-5pt}
            \begin{enumerate}
                \item Simulation and analysis of proposed perceptual sound zone algorithms.
            \end{enumerate}
    \end{enumerate}
\end{frame}

\begin{frame}{\textbf{Proposal of Perceptual Sound Zone Framework:}\\ Approach}
    In the previous part we found that the Par distortion detectability is a suitable perceptual model.
    How can this perceptual model be used to pose perceptual sound zone algorithms?\\
    \vspace{10pt}
    In order to answer this question, a literature review was performed into various sound zone approaches.
    From this review it was found that a \textbf{Pressure Matching} sound zone approach can easily be adapted to 
    use the Par distortion detectability.
\end{frame}

\begin{frame}{\textbf{Proposal of Perceptual Sound Zone Framework:}\\ Review of Pressure Matching I}
    In order to solve the sound zone problem, a pressure matching approach samples the zones into a number of \textbf{control points}:
    \begin{columns}[c]
        \column{.44\textwidth}
        \begin{figure}[]
            \centering
            \scalebox{0.7}{\input{tikz/tikz_2D_room_data_model.tex}}
        \end{figure}
        \column{.44\textwidth}
        \begin{figure}[]
            \centering
            \scalebox{0.7}{\begin{tikzpicture} 
    \draw [draw=black] (0,0) rectangle (8,6);

    % Speakers on the top wall
    \pic[scale=0.7] at (2.5, 5.6) {Speaker};
    \pic[scale=0.7] at (3.5, 5.6) {Speaker};
    \pic[scale=0.7] at (4.5, 5.6) {Speaker};
    \pic[scale=0.7] at (5.5, 5.6) {Speaker};

    % Speakers on the bottom wall
    \pic[rotate=180, scale=0.7] at (2.5, 0.4) {Speaker};
    \pic[rotate=180, scale=0.7] at (3.5, 0.4) {Speaker};
    \pic[rotate=180, scale=0.7] at (4.5, 0.4) {Speaker};
    \pic[rotate=180, scale=0.7] at (5.5, 0.4) {Speaker};

    \draw[opacity=0.7, pattern=wide2] (6,3) circle[radius=1.5];
    \draw[opacity=0.4, fill=blue] (6,3) circle[radius=1.5];
    \draw[thick] (6,3) circle (1.5) node[align=center] {\textbf{Zone $\za$}};
    \draw[opacity=0.7, pattern=wide2]  (2,3) circle[radius=1.5];
    \draw[opacity=0.4, fill=red]  (2,3) circle[radius=1.5];
    \draw[thick] (2,3) circle (1.5) node[align=center] {\textbf{Zone $\zb$}};
\end{tikzpicture}
}
        \end{figure}
    \end{columns}
\end{frame}

\begin{frame}{\textbf{Proposal of Perceptual Sound Zone Framework:}\\ Review of Pressure Matching II}
    \begin{columns}[c]
        \column{.44\textwidth}
        \begin{itemize}
            \item Per control point $m$, a \textbf{target sound pressure} is assigned.
            \item The goal is to use the loudspeakers to \textbf{attain the target sound pressure} in all points.
            \item At the same time, we want to have \textbf{minimal interference} in all control points.
        \end{itemize}
        \column{.44\textwidth}
        \begin{figure}[]
            \centering
            \scalebox{0.7}{\begin{tikzpicture} 
    \draw [draw=black] (0,0) rectangle (8,6);

    % Speakers on the top wall
    \pic[scale=0.7] at (2.5, 5.6) {Speaker};
    \pic[scale=0.7] at (3.5, 5.6) {Speaker};
    \pic[scale=0.7] at (4.5, 5.6) {Speaker};
    \pic[scale=0.7] at (5.5, 5.6) {Speaker};

    % Speakers on the bottom wall
    \pic[rotate=180, scale=0.7] at (2.5, 0.4) {Speaker};
    \pic[rotate=180, scale=0.7] at (3.5, 0.4) {Speaker};
    \pic[rotate=180, scale=0.7] at (4.5, 0.4) {Speaker};
    \pic[rotate=180, scale=0.7] at (5.5, 0.4) {Speaker};

    \draw[opacity=0.7, pattern=wide2] (6,3) circle[radius=1.5];
    \draw[opacity=0.4, fill=blue] (6,3) circle[radius=1.5];
    \draw[thick] (6,3) circle (1.5) node[align=center] {\textbf{Zone $\za$}};
    \draw[opacity=0.7, pattern=wide2]  (2,3) circle[radius=1.5];
    \draw[opacity=0.4, fill=red]  (2,3) circle[radius=1.5];
    \draw[thick] (2,3) circle (1.5) node[align=center] {\textbf{Zone $\zb$}};
\end{tikzpicture}
}
        \end{figure}
    \end{columns}
\end{frame}

\begin{frame}{\textbf{Proposal of Perceptual Sound Zone Framework:}\\ Review of Pressure Matching III}
    Mathematically, the optimization problem can be given as follows:
    \begin{equation}
        \argmin{}{\sum_m \left(\mathrm{RE}^{(m)} + \mathrm{LE}^{(m)}\right)}
    \end{equation}
    Here, the following quantities are minimized:
    \begin{itemize}
        \item $\mathrm{RE}^{(m)}$ is the \textbf{reproduction error} for control point $m$. 
            It quantifies the deviation from the target sound pressure.     
        \item $\mathrm{LE}^{(m)}$ is the \textbf{leakage error} for control point $m$. 
            It quantifies how much interference is present. 
    \end{itemize}
\end{frame}

\begin{frame}{\textbf{Proposal of Perceptual Sound Zone Framework:}\\ Proposal of Pressure Error Detectability Framework}
    Rather than optimizing over the reproduction error and the leakage error per control point, 
    the proposed approach is to instead optimize over the reproduction error detectability and the leakage error detectability.
\end{frame}

\begin{frame}{\textbf{Structure:}\\ Answering Research Questions}
    \begin{enumerate}
        \item {\textit{``How can auditory perceptual models be included in sound~zone~algorithms?''}}
            \vspace{7pt}
            \begin{enumerate}
                \item Determination of a suitable perceptual model for sound zone algorithms.
                \vspace{7pt}
                \item \textbf{Proposal of a perceptual sound zone framework using determined model.}
                \vspace{7pt}
                \item Proposal of perceptual sound zone algorithms through proposed framework.
                \vspace{7pt}
            \end{enumerate}
        \item {\textit{``What are benefits of including auditory perceptual models in sound zone algorithms?''}}
            \vspace{-5pt}
            \begin{enumerate}
                \item Simulation and analysis of proposed perceptual sound zone algorithms.
            \end{enumerate}
    \end{enumerate}
\end{frame}

\begin{frame}{\textbf{Structure:}\\ Answering Research Questions}
    \begin{enumerate}
        \item {\textit{``How can auditory perceptual models be included in sound~zone~algorithms?''}}
            \vspace{7pt}
            \begin{enumerate}
                \item Determination of a suitable perceptual model for sound zone algorithms.
                \vspace{7pt}
                \item Proposal of a perceptual sound zone framework using determined model. 
                \vspace{7pt}
                \item \textbf{Proposal of perceptual sound zone algorithms through proposed framework.}
                \vspace{7pt}
            \end{enumerate}
        \item {\textit{``What are benefits of including auditory perceptual models in sound zone algorithms?''}}
            \vspace{-5pt}
            \begin{enumerate}
                \item Simulation and analysis of proposed perceptual sound zone algorithms.
            \end{enumerate}
    \end{enumerate}
\end{frame}

\begin{frame}{\textbf{Proposal of Perceptual Sound Zone Algorithms:}\\ Approach}
    To propose perceptual sound zone algorithms, the previously introduced framework is used:
    \begin{itemize}
        \item $\textrm{RED}^{(m)}$ is the reproduction error detectability which quantifies how
            noticeable the deviation from the target sound pressure is for point $m$.
        \item $\textrm{LED}^{(m)}$ is the leakage error detectability which quantifies how
            noticeable the interference is for point $m$. 
    \end{itemize}
\end{frame}

\begin{frame}{\textbf{Proposal of Perceptual Sound Zone Algorithms:}\\ Algorithm 1: Unconstrained Perceptual Pressure
    Matching}
    The first algorithm minimizes the total error detectability.
    \begin{equation}
        \argmin{}{\sum_m \left(\mathrm{RED}^{(m)} + \mathrm{LED}^{(m)}\right)}
    \end{equation}
\end{frame}

\begin{frame}{\textbf{Proposal of Perceptual Sound Zone Algorithms:}\\ Algorithm 2: Constrained Perceptual Pressure
    Matching}
    The second algorithm minimizes the leakage error detectability, while constraining the reproduction error detectability.
    \begin{align}
        \argmin{}{&\sum_m \mathrm{LED}^{(m)}}\\
        \subjectto{&\mathrm{RED}^{(m)} \leq D_0 \quad \forall\,m}
    \end{align}
\end{frame}

\begin{frame}{\textbf{Structure:}\\ Answering Research Questions}
    \begin{enumerate}
        \item {\textit{``How can auditory perceptual models be included in sound~zone~algorithms?''}}
            \vspace{7pt}
            \begin{enumerate}
                \item Determination of a suitable perceptual model for sound zone algorithms.
                \vspace{7pt}
                \item Proposal of a perceptual sound zone framework using determined model. 
                \vspace{7pt}
                \item \textbf{Proposal of perceptual sound zone algorithms through proposed framework.}
                \vspace{7pt}
            \end{enumerate}
        \item {\textit{``What are benefits of including auditory perceptual models in sound zone algorithms?''}}
            \vspace{-5pt}
            \begin{enumerate}
                \item Simulation and analysis of proposed perceptual sound zone algorithms.
            \end{enumerate}
    \end{enumerate}
\end{frame}

\begin{frame}{\textbf{Structure:}\\ Answering Research Questions}
    \begin{enumerate}
        \item {\textit{``How can auditory perceptual models be included in sound~zone~algorithms?''}}
            \vspace{7pt}
            \begin{enumerate}
                \item Determination of a suitable perceptual model for sound zone algorithms.
                \vspace{7pt}
                \item Proposal of a perceptual sound zone framework using determined model. 
                \vspace{7pt}
                \item Proposal of perceptual sound zone algorithms through proposed framework.
                \vspace{7pt}
            \end{enumerate}
        \item {\textit{``What are benefits of including auditory perceptual models in sound zone algorithms?''}}
            \vspace{-5pt}
            \begin{enumerate}
                \item \textbf{Simulation and analysis of proposed perceptual sound zone algorithms.}
            \end{enumerate}
    \end{enumerate}
\end{frame}

\begin{frame}{\textbf{Evaluation of Perceptual Sound Zone Algorithms:}\\ Approach}
    In order to determine the benefits of the perceptual sound zone approach, the previously 
    proposed perceptual sound zone algorithms are simulated.
\end{frame}

\begin{frame}{\textbf{Evaluation of Perceptual Sound Zone Algorithms:}\\ Simulation Setup}
    \begin{columns}[c]
        \column{.44\textwidth}
        \begin{itemize}
            \item A 5 by 5 meter square room with 4 loudspeakers is used for the evaluation.
            \item The zones, each consisting of two points, are assigned speech content for the simulations.
        \end{itemize}
        \column{.44\textwidth}
        \begin{figure}[]
            \centering
            \scalebox{0.7}{\begin{tikzpicture}
    \draw [draw=black] (0,0) rectangle (5,5);

    % Speakers on the top wall
    \pic[scale=0.7] at (1.00, 4.25) {Speaker};
    \pic[scale=0.7] at (4.00, 4.25) {Speaker};

    % Speakers on the bottom wall
    \pic[rotate=180, scale=0.7] at (1.00, 0.75) {Speaker};
    \pic[rotate=180, scale=0.7] at (4.00, 0.75) {Speaker};

    \draw[opacity=0.4, fill=blue, draw=white] (1.25, 2.00) circle[radius=0.15];
    \draw[opacity=0.4, fill=blue, draw=white] (2.25, 2.00) circle[radius=0.15];
    \draw[opacity=0.4, fill=red, draw=white] (3.75, 2.00) circle[radius=0.15];
    \draw[opacity=0.4, fill=red, draw=white] (2.75, 2.00) circle[radius=0.15];
\end{tikzpicture}
}
        \end{figure}
    \end{columns}
\end{frame}

\begin{frame}{\textbf{Evaluation of Perceptual Sound Zone Algorithms:}\\ Evaluation Measures}
    In order to effectively compare the reference and the perceptual approach, perceptual measures are used.\\
    \vspace{10pt}
    This presentation will use the Perceptual Evaluation of Speech Quality (PESQ)\cite{rix2001perceptual} 
    and Distraction\cite{ramo2017real} perceptual measures.
\end{frame}

\begin{frame}{\textbf{Evaluation of Perceptual Sound Zone Algorithms:}\\ Evaluation of Unconstrained
    Perceptual Pressure Matching}
    \begin{table}[]
        \sisetup{table-format=1.3(7), round-precision=3, table-comparator=true, round-integer-to-decimal=false, 
        separate-uncertainty=true, table-align-text-post = true}
        \centering
        \begin{tabular}{lS[round-mode=places]S[round-mode=places]}
        \toprule
         \multicolumn{1}{c}{\textbf{Measure}} & 
         \textbf{\begin{tabular}[c]{@{}c@{}}Unconstrained Perceptual PM\\ Mean ($\pm$ 95\% CI)\end{tabular}} & 
         \textbf{\begin{tabular}[c]{@{}c@{}}Reference PM\\ Mean ($\pm$ 95\% CI)\end{tabular}} \\ 
        \midrule
         PESQ (No interference)    & 3,345                  \pm 0,087              & 4,107                 \pm 0,051                \\ 
         PESQ          & 3,154                  \pm 0,081              & 2,609                 \pm 0,084                \\ 
        % Total STOI          & 0,943                  \pm 0,003              & 0,940                 \pm 0,006                \\ 
        % Bright Zone STOI    & 0,950                  \pm 0,003              & 0,989                 \pm 0,001                \\ 
        % Total SIIB          & 1114,306               \pm 23,762             & 893,225               \pm 63,815               \\ 
        % Bright Zone SIIB    & 1260,117               \pm 14,290             & 1311,041              \pm 12,333               \\ 
    % Total NMSE              & -4,929 \pm 0,235 $\mathrm{\hspace{-12pt}dB}$  & -13,529 \pm 0,856 $\mathrm{\hspace{-12pt}dB}$  \\ 
        % Bright Zone NMSE    & -5,241 \pm 0,248 $\mathrm{\hspace{-12pt}dB}$  & -16,600 \pm 0,875 $\mathrm{\hspace{-12pt}dB}$  \\ 
        Distraction         & 7,828                  \pm 1,868              & 12,693  \pm 3,405                              \\ 
        % Acoustic Contrast   & 13,258 \pm 0,379 $\mathrm{\hspace{-12pt}dB}$  & 16,075  \pm 0,936 $\mathrm{\hspace{-12pt}dB}$  \\ 
        \bottomrule
    \end{tabular}
    \end{table}
    The unconstrained perceptual pressure matching approach outperforms the reference when interference is taken into account.
    It seems to make a better perceptual trade-off between minimizing the reproduction error and minimizing interference.
\end{frame}

\begin{frame}{\textbf{Evaluation of Perceptual Sound Zone Algorithms:}\\ Evaluation of Unconstrained
    Perceptual Pressure Matching}
    \begin{figure}[]
        \centering
        \scalebox{0.55}{\begin{tikzpicture}
    \begin{groupplot}[group style={group size=2 by 2, horizontal sep=2.5cm,
        vertical sep=2.5cm}]
        \nextgroupplot[xlabel={Time [samples]}, ylabel={$t[n]$}, ylabel near ticks, 
        title={Target Sound Pressure}, scale only axis, 
        height=3cm, width=9cm, xmin=5000, xmax=6400, xtick={5000, 5200, 5400, 5600, 5800, 6000, 6200, 6400},
        ymin=-0.5, ymax=0.5]
        \addplot [color=red, error bars/.cd, y dir=both, y explicit]
        table [x expr=\thisrow{samples}, y expr=(\thisrow{audio} / 32787), col sep=comma] 
            {data/target.csv};
       \addplot [samples=50, smooth, name path=H11] coordinates {(5500,-0.5) (5500,0.5)};
       \addplot [samples=50, smooth, name path=H12] coordinates {(6200,-0.5) (6200,0.5)};
       \addplot [fill opacity=0.05] fill between [of=H11 and H12];

        \nextgroupplot[xlabel={Time [samples]}, ylabel={$t[n]$}, ylabel near ticks, 
        title={Target Sound Pressure}, scale only axis, 
        height=3cm, width=9cm, xmin=5000, xmax=6400, xtick={5000, 5200, 5400, 5600, 5800, 6000, 6200, 6400},
        ymin=-0.5, ymax=0.5]
        \addplot [color=red, error bars/.cd, y dir=both, y explicit]
        table [x expr=\thisrow{samples}, y expr=(\thisrow{audio} / 32787), col sep=comma] 
            {data/target.csv};
       \addplot [samples=50, smooth, name path=H11] coordinates {(5500,-0.5) (5500,0.5)};
       \addplot [samples=50, smooth, name path=H12] coordinates {(6200,-0.5) (6200,0.5)};
       \addplot [fill opacity=0.05] fill between [of=H11 and H12];

        \nextgroupplot[xlabel={time [samples]}, ylabel={$p[n]$}, ylabel near ticks, 
        title={Perceptual Interference}, scale only axis, 
        height=3cm, width=9cm, xmin=5000, xmax=6400, xtick={5000, 5200, 5400, 5600, 5800, 6000, 6200, 6400},
        ymin=-0.5, ymax=0.5]
        \addplot [color=red, error bars/.cd, y dir=both, y explicit]
        table [x expr=\thisrow{samples}, y expr=(\thisrow{audio} / 32787), col sep=comma] 
            {data/per_leakage.csv};
       \addplot [samples=50, smooth, name path=H11] coordinates {(5500,-0.5) (5500,0.5)};
       \addplot [samples=50, smooth, name path=H12] coordinates {(6200,-0.5) (6200,0.5)};
       \addplot [fill opacity=0.05] fill between [of=H11 and H12];

        \nextgroupplot[xlabel={time [samples]}, ylabel={$p[n]$}, ylabel near ticks, 
        title={Reference Interference}, scale only axis, 
        height=3cm, width=9cm, xmin=5000, xmax=6400, xtick={5000, 5200, 5400, 5600, 5800, 6000, 6200, 6400},
        ymin=-0.5, ymax=0.5]
        \addplot [color=red, error bars/.cd, y dir=both, y explicit]
        table [x expr=\thisrow{samples}, y expr=(\thisrow{audio} / 32787), col sep=comma] 
            {data/ref_leakage.csv};
       \addplot [samples=50, smooth, name path=H11] coordinates {(5500,-0.5) (5500,0.5)};
       \addplot [samples=50, smooth, name path=H12] coordinates {(6200,-0.5) (6200,0.5)};
       \addplot [fill opacity=0.05] fill between [of=H11 and H12];
    \end{groupplot}
\end{tikzpicture}
}
    \end{figure}
\end{frame}

\begin{frame}{\textbf{Evaluation of Perceptual Sound Zone Algorithms:}\\ Evaluation of Constrained
    Perceptual Pressure Matching}
\end{frame}

\begin{frame}{\textbf{Structure:}\\ Answering Research Questions}
    \begin{enumerate}
        \item {\textit{``How can auditory perceptual models be included in sound~zone~algorithms?''}}
            \vspace{7pt}
            \begin{enumerate}
                \item Determination of a suitable perceptual model for sound zone algorithms.
                \vspace{7pt}
                \item Proposal of a perceptual sound zone framework using determined model. 
                \vspace{7pt}
                \item Proposal of perceptual sound zone algorithms through proposed framework.
                \vspace{7pt}
            \end{enumerate}
        \item {\textit{``What are benefits of including auditory perceptual models in sound zone algorithms?''}}
            \vspace{-5pt}
            \begin{enumerate}
                \item \textbf{Simulation and analysis of proposed perceptual sound zone algorithms.}
            \end{enumerate}
    \end{enumerate}
\end{frame}

\begin{frame}{\textbf{Conclusions:}}
\end{frame}

\begin{frame}{\textbf{Future Work:}}
\end{frame}

\begin{frame}{\textbf{Sources:}}
    \printbibliography
\end{frame}
\end{document}
