I envision the outline of my thesis as follows:
\begin{itemize}
    \item \textbf{Introduction:}
        \begin{itemize}
            \item Introduction to sound zones:
                \begin{itemize}
                    \item Explain sound zones, and why we want them.
                    \item Explain where the state of the art comes short.
               \end{itemize}
            \item Introduction to perceptual approach.
                \begin{itemize}
                    \item Motivate the perceptual approach.
                        \begin{itemize}
                            \item Optimizes for perceptual experience, rather than sound pressure
                        \end{itemize}
                    \item Briefly discuss prior work in this approach.
                        \begin{itemize}
                            \item Work done by Taewoong Lee 
                                \cite{lee2020signal}\cite{lee2019towards}.
                            \item Work done by Jacob Donley
                                \cite{donley2015multizone}\cite{donley2018multizone}
                        \end{itemize}
                    \item Introduce goal of thesis:
                \end{itemize}
            \item Give structure of the document
        \end{itemize}
    \item \textbf{Review of Sound Zones}
        \begin{itemize}
            \item Formal introduction to the sound zone problem.
                \begin{itemize}
                    \item Introduction of room model:
                        \begin{itemize}
                            \item Room
                            \item Zones
                            \item Loudspeakers
                        \end{itemize}
                    \item Goal of sound zone algorithm:
                        \begin{itemize}
                            \item Attaining Target Sound Pressure
                        \end{itemize}
                    \item Way of attaining goal:
                        \begin{itemize}
                            \item Controlling loudspeaker inputs
                            \item Relation sound pressure and loudspeaker inputs
                        \end{itemize}
                \end{itemize}
            \item Short review of prior work, typical approaches.
                \begin{itemize}
                    \item Pressure Matching
                    \item Acoustic Contrast Control
                    \item Mode Matching
                \end{itemize}
        \end{itemize}
    \item \textbf{Literature Review of Perceptual Models for use in Sound Zones}
        \begin{itemize}
            \item Criteria for perceptual models for sound zones
                \begin{itemize}
                    \item Easy to optimize for
                    \item Feasible to compute in real time
                \end{itemize}
            \item Masking Models
                \begin{itemize}
                    \item Par Distraction Models
                    \item Taal Distraction Models
                    \item MPEG Model I and II
                    \item Dau Model
                \end{itemize}
            \item Objective Measures
                \begin{itemize}
                    \item Distraction Model
                    \item Speech Metrics
                        \begin{itemize}
                            \item STOI
                            \item SIIB
                            \item PESQ
                        \end{itemize}
                    \item Audio Metrics
                        \begin{itemize}
                            \item ViSQOL
                            \item PEAQ
                        \end{itemize}
                \end{itemize}
            \item Selection of perceptual model
                \begin{itemize}
                    \item Par Distraction selected.
                        \begin{itemize}
                            \item Convex, easy to optimize for.
                            \item Feasible to compute in real time.
                        \end{itemize}
                    \item Note that other models can be used for evaluation.
                \end{itemize}
        \end{itemize}
    \item \textbf{Analysis of Detectability for Sound Zone Algorithms}
        \begin{itemize}
            \item Detailed discussion of Detectability
                \begin{itemize}
                    \item Psycho-acoustical background
                    \item Implementation details
                        \begin{itemize}
                            \item Constructing frequency domain weights
                            \item Calibration of the model
                        \end{itemize}
                \end{itemize}
            \item Discussion of which Sound Zone Approach best suited for Detectability 
                \begin{itemize}
                    \item Review of discussed sound zone algorithms with respect to Detectability
                        \begin{itemize}
                            \item Mode Matching, Pressure Matching in a different domain.
                            \item Acoustic Contrast Control, no content control.
                            \item Pressure Matching, Detectability is an alternative to squared error.
                        \end{itemize}
                    \item Conclusion that Pressure Matching is the best
                \end{itemize}
        \end{itemize}
    \item \textbf{Implementation of Detectability Minimization Algorithms}
        \begin{itemize}
            \item Derivation of the Reference Pressure Matching Approach
            \item Derivation of the Detectability Minimization Approach
            \item Derivation of the Detectability Constraining Approach
        \end{itemize}
\end{itemize}
