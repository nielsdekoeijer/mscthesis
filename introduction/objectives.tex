\section{Objectives and Organization}
\label{ch:introduction:objectives}
This section will state the goals of the thesis and organization of the rest of this document.
As stated in the motivation, the goal of the thesis is to construct a perceptual sound zone algorithm.
The central question in this thesis is: 

{\centering\textit{``How can perceptual models of the human auditory system be used to improve sound zone algorithms?''}}

This question is answered in three steps. 
In first step covered by \autoref{ch:perceptual}, a suitable perceptual model is selected.
Next, a sound zone algorithm is selected in \autoref{ch:sound_zone}.
Finally, the two are combined and the resulting algorithm is evaluated in \autoref{ch:perceptual_sound_zone}.

The organization of the three chapters is given below.

\subsection{Search and Implementation of a Perceptual Model}
\label{ch:introduction:objectives:perceptual}
The goal of \autoref{ch:perceptual} is to find a perceptual model suitable for use in a perceptual sound zone algorithm.
The chapter starts in \autoref{ch:perceptual:background} with a summary of necessary psycho-acoustics background .

In order to find a suitable perceptual model, it must first be determined what perceptual models exist.
As such, a literature review into promising state of the art perceptual models is given in\autoref{ch:perceptual:review}.

From the models discussed during the literature one must be selected.
As such, \autoref{ch:perceptual:selection} reflects on what the desirable properties of a perceptual model are, and selects one accordingly. 

Finally, the selected perceptual model is discussed in greater detail in \autoref{ch:perceptual:implementation}, concluding the chapter. 

\subsection{Search and Implementation of a Sound Zone Approach}
\label{ch:introduction:objectives:sound_zone}
In \autoref{ch:sound_zone} will document the search of a sound zone algorithm that is suitable for the selected perceptual model.

To this end, a literature review is performed in \autoref{ch:sound_zone:approach_review} to document what sound zone approaches exist.
Using this review, \autoref{ch:sound_zone:approach_selection} will reflect on the selected perceptual model to select one of the 
reviewed approaches.

The implementation of the selected sound zone approach is then given by first formalizing the sound zone problem mathematically in 
\autoref{ch:sound_zone:data_model}, giving the mathematical framework from which algorithms can be derived.
The framework is then used to give the implementation in \autoref{ch:sound_zone:approach_implementation}.

The following sections \autoref{ch:sound_zone:block_based} and \autoref{ch:sound_zone:frequency_domain_conversion} rewrite the problem slightly to 
prepare for integration with the perceptual model.

The derived sound zone algorithm will form the foundation on which the perceptual sound zone algorithm will be built.
In addition to this, it will serve as a reference implementation with which the perceptual sound zone algorithm
can be compared.

\subsection{Implementation of a Perceptual Sound Zone Algorithm}
\label{ch:introduction:objectives:perceptual_sound_zone}
