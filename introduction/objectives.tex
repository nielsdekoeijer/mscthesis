\section{Objectives and Organization}
\label{ch:introduction:objectives}
As stated in the preface, this thesis investigates the methodology and the benefits of including perceptual information in sound zone algorithms.
To this end, the work in this thesis seeks to answer two research questions:
\begin{itemize}
    \item \textbf{RQ1:} {\textit{``How can auditory perceptual models be included in sound~zone~algorithms?''}}
    \item \textbf{RQ2:} {\textit{``What are the benefits of including auditory perceptual models in sound~zone~algorithms?''}}
\end{itemize}

The answers to these questions are summarized in the conclusion given by \autoref{ch:conclusion}. 
What follows is a description of the approach that is taken in answering these research questions, 
alongside the structure of the rest of this document.

\subsection{Creation of Perceptual Sound Zone Algorithms}
The first research question RQ1, 

\begin{center}
    {\textit{``How can auditory perceptual models be included in sound~zone~algorithms?''}}
\end{center}

is answered in \autoref{ch:perceptual}, \autoref{ch:sound_zone}, and \autoref{ch:perceptual_sound_zone}.
These chapters document the design of a perceptual sound zone algorithm.
The chapters are structured as follows.
\begin{itemize}
    \item First, in \autoref{ch:perceptual} a literature review is performed to determine which perceptual models are 
        suitable for use in a perceptual sound zone algorithm.
        In this pursuit, one perceptual model is found to be the most promising and discussed in further detail.
    \item Next, in \autoref{ch:sound_zone} a perceptual sound zone framework is proposed, which uses the selected perceptual model.
        This framework is motivated through a literature review of existing sound zone approaches 
        and by reflecting on the mathematical properties of the perceptual model.
    \item Finally, in \autoref{ch:perceptual_sound_zone} the proposal of two perceptual sound zone algorithms 
        that employ the proposed perceptual sound zone framework is discussed.
\end{itemize}

\subsection{Determining Benefits of Perceptual Sound Zone Algorithms}
The second research question RQ2, 

\begin{center}
    {\textit{``What are the benefits of including auditory perceptual models in sound~zone~algorithms?''}}
\end{center}

is answered in \autoref{ch:results}, where the perceptual sound zone algorithms derived in answering RQ1 are 
analyzed and compared with a non-perceptual reference sound zone algorithm.
