\section{Objectives and Organization}
\label{ch:introduction:objectives}
This section will state the goals of the thesis and organization of the rest of this document.
Als stated in the motivation, the goal of the thesis is to construct a perceptual sound zone algorithm.
The central question in this thesis is: 

{\centering\textit{``How can perceptual models of the human auditory system be used to improve sound zone algorithms?''}}

This question is answered in three steps. 
First, a perceptual model is selected in \autoref{ch:perceptual}.
Next, a sound zone algorithm is selected in \autoref{ch:sound_zone}.
Finally, the two are combined and the resulting algorithm is evaluated in \autoref{ch:perceptual_sound_zone}.
The organization of the three chapters is given below.

\subsection{Search and Implementation of a Perceptual Model}
\label{ch:introduction:objectives:perceptual}
In order to construct a perceptual sound zone algorithm, a perceptual model is required.
This begs the question: ``What perceptual models exist?''.
To answer this question, this chapter starts with a summary of necessary psycho-acoustics background in 
\autoref{ch:perceptual:background} followed by a literature review into state of the art perceptual models
in \autoref{ch:perceptual:review}.

To select a perceptual model suitable for combination with a sound zone algorithm, one must consider what 
desired properties of such a model.
These criteria and the resulting selection of one of the perceptual models from the 
literature review is given in \autoref{ch:perceptual:selection}.

The selected perceptual model is then discussed in greater detail in \autoref{ch:perceptual:implementation}. 
Here, implementation details are given and analysis are performed to give the reader an intuition into the model.

\subsection{Search and Implementation of a Sound Zone Approach}
\label{ch:introduction:objectives:sound_zone}
After selecting a perceptual model, a suitable sound zone approach must be selected.
Before determining which sound zone approaches are suitable, a literature review is performed in
\autoref{ch:sound_zone:approach_review} to document what sound zone approaches exist.

The perceptual model will impose certain constraints on which sound zone algorithm is best suited for integration.
As such, \autoref{ch:sound_zone:approach_selection} will reflect on these constraints to select one of the documented
sound zone approaches as the most promising for integration.

The sections that follow will discuss implementation details of the selected sound zone approach.
In \autoref{ch:sound_zone:data_model} a general sound zone data model will be given, formalizing the sound zone problem
and laying the mathematical foundation.
This mathematical foundation is then used in \autoref{ch:sound_zone:approach_implementation} and 
\autoref{ch:sound_zone:block_based} to derive a sound zone algorithm based on the selected sound zone approach.

The derived sound zone algorithm will form the foundation on which the perceptual sound zone algorithm will be built.
In addition to this, it will serve as a reference implementation with which the perceptual sound zone algorithm
can be compared.

\subsection{Implementation of a Perceptual Sound Zone Algorithm}
\label{ch:introduction:objectives:perceptual_sound_zone}
