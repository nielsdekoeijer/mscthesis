\section{Objectives and Organization}
\label{ch:introduction:objectives}
This section states the goals of the thesis and organization of the rest of this document.

This thesis seeks to answer two research questions:
\begin{itemize}
    \item \textbf{RQ1:} {\textit{``How can auditory perceptual models be included in sound zone algorithms?''}}
    \item \textbf{RQ2:} {\textit{``What are the benefits of including auditory perceptual models in sound zone algorithms?''}}
\end{itemize}

What follows is how these research questions will be answered, alongside the structure of the rest of this document.

\subsection{Creation of Perceptual Sound Zone Algorithms}
The first research question RQ1, 

\begin{center}
    {\textit{``How can auditory perceptual models be included in sound~zone~algorithms?''}}
\end{center}

is answered in \autoref{ch:perceptual}, \autoref{ch:sound_zone}, and \autoref{ch:perceptual_sound_zone}.
These chapters document the design of a perceptual sound zone algorithm.
The chapters are structured as follows.
\begin{itemize}
    \item First, in \autoref{ch:perceptual} a literature review is performed to determine which perceptual models are suitable for use in a 
        sound zone algorithm.
        In this pursuit, one perceptual model is found to be the most promising is selected and discussed in further detail.
    \item Next, in \autoref{ch:sound_zone} motivates and discusses which existing sound zone approaches can be used to formulate a perceptual sound zone 
        algorithm.
        This is done by reviewing sound zone literature and reflecting on the mathematical properties of the selected perceptual model.
    \item Finally, in \autoref{ch:perceptual_sound_zone} proposes two perceptual sound zone algorithms using the determined approaches.
\end{itemize}

\subsection{Determining Benefits of Perceptual Sound Zone Algorithms}
The second research question RQ2, 

\begin{center}
    {\textit{``What are the benefits of including auditory perceptual models in sound~zone~algorithms?''}}
\end{center}

is answered in \autoref{ch:results}.
This is done by comparing the perceptual sound zone algorithms derived in answering RQ1 with a reference sound zone algorithm.
