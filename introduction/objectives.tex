\section{Objectives and Organization}
\label{ch:introduction:objectives}
This section states the goals of the thesis and organization of the rest of this document.

As stated in the preface, the goal of the work described in this thesis is to investigate the construction and 
the benefits of including perceptual information in sound zone algorithms.
To this end, work in this thesis seeks to answer two research questions:
\begin{itemize}
    \item \textbf{RQ1:} {\textit{``How can auditory perceptual models be included in sound zone algorithms?''}}
    \item \textbf{RQ2:} {\textit{``What are the benefits of including auditory perceptual models in sound zone algorithms?''}}
\end{itemize}

The answers to these questions are given in \autoref{ch:conclusion}. 
What follows is the approach that is taken to answer these research questions, alongside the structure of the rest of this document.

\subsection{Creation of Perceptual Sound Zone Algorithms}
The first research question RQ1, 

\begin{center}
    {\textit{``How can auditory perceptual models be included in sound~zone~algorithms?''}}
\end{center}

is answered in \autoref{ch:perceptual}, \autoref{ch:sound_zone}, and \autoref{ch:perceptual_sound_zone}.
These chapters document the design of a perceptual sound zone algorithm.
The chapters are structured as follows.
\begin{itemize}
    \item First, in \autoref{ch:perceptual} a literature review is performed to determine which perceptual models are suitable for use in a 
        sound zone algorithm.
        In this pursuit, one perceptual model is found and hypothesized to be the most promising, then selected and discussed in further detail.
    \item Next, in \autoref{ch:sound_zone} motivates a perceptual sound zone approach.
        This is done by reviewing literature to determine which sound zone approaches exist 
        and reflecting on the mathematical properties of the selected perceptual model.
        This results in the proposal of a perceptual sound zone framework.
    \item Finally, in \autoref{ch:perceptual_sound_zone} implements two perceptual sound zone algorithms using proposed perceptual approach.
\end{itemize}

\subsection{Determining Benefits of Perceptual Sound Zone Algorithms}
The second research question RQ2, 

\begin{center}
    {\textit{``What are the benefits of including auditory perceptual models in sound~zone~algorithms?''}}
\end{center}

is answered in \autoref{ch:results}.
This is done by comparing the perceptual sound zone algorithms derived in answering RQ1 with a reference sound zone algorithm.
