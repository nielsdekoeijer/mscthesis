\section{Distortion Measures}
Distortion measures are typically used in transparent audio coding.
Audio coding algorithms attempts to find an low-bitrate representation of an audio input signal.
This process is usually lossy, as reducing the bitrate introduces errors.

These errors can be a detriment to the listening experience.
As such, most audio coding algorithms use a perceptual model.
The perceptual model is used to introduce encoding errors in such a way that the audio output
signal is perceptually indistinguishable from the audio input signal~\cite{taal2012low}.

The perceptual model typically takes form of a distortion function which determines how
audible the difference between a reference input audio signal and a distorted output audio signal is.
This function is used to encode an input audio signal such that it has minimal distortion for a
specified bitrate.

The functions are designed based on findings from psycho-acoustical literature.
Typically, they operate on short-time scales as psycho-acoustical effects are time-local.

One possibility is to use such distortion functions in a sound zone algorithm.
As such, in this section, a number of such distortion metrics are discussed.

\subsection{ISO MPEG Models}
The ISO/IEC 11172-3 standard specifies a coded representation for audio files~\cite{ISO11172-3}, 
and a decoder for said representation.
An encoder said representation is not part of the standard.
This is done deliberately, to allow for future improvements to the encoder, without having to change the standard~\cite{pan1995tutorial}.

The standard does however provide a number of examples of possible encoders, with increasing complexity.
Alongside these example encoders, two psycho-acoustical models are included for use during the encoding process. 

The psycho-acoustical models work by subdividing the input audio signal into different frequency bands, 
modeling the frequency bands in the human auditory system.
Separately per band, the model then determines how much quantization noise can be added without it becoming audible.
As such, the model assumes that the distortion signal is noise-like~\cite{van2005perceptual}.

The output of the psycho-acoustical model is thus the amount of noise that can be added per band.
In the case of audio coding, this can then be used to control quantization noise.
This technique has however also been used for various signal processing purposes, such as audio watermarking~\cite{taal2012low}.

\subsection{Par Detectability}
In 2005, van der Par et al. proposed a novel perceptual model~\cite{van2005perceptual}.
The perceptual model defines a distortion measure which determines the ``detectability'' of a distortion signal 
in presence of a masking signal.
That is to say, how likely is a human to detect the distortion signal.

The proposed method differentiates itself from the previously discussed ISO MPEG models in two ways.

Firstly, the paper uses newer findings from psycho-acoustic literature, namely spectral integration.
In spectral integration, the masking effects from neighboring bands are taken into account when computing the masking effects.
The psycho-acoustical models defined in the ISO MPEG standard does not do this, and effectively works independently~\cite{taal2012low}.

Secondly, it assumes that the distortion signal is sinusoidal, rather than noise-like.
As such, it is more effective in hiding sinusoidal distortion.

On top of this, the proposed distortion measure can be expressed as an L2-norm.
This mathematical tractability makes for easy integration into existing least-square problems~\cite{taal2012low}.

The van der Par model has been used in many signal processing applications, examples ranging from speech enhancement to removing perceptually irrelevant sinusoidal 
components~\cite{balazs2009time, taal2013optimal}.

\subsection{Taal Detectability}
A paper from 2012 by Taal et al. proposed a novel perceptual model \cite{taal2012low}.
The perceptual model also defines a distortion measure which determines the detectability, with identical interpretation from the Par Detectability model.

In contrast to the Par Detectability, the Taal Detectability measure takes temporal information into account.
The inclusion of temporal information allows for the suppression of pre-echoes, 
an artifact introduced by assuming that the masking effects are stationary over the short time frame over which it operates.

In contrast to other temporal perceptual models, the Taal Detectability has a relatively low computational complexity.
In addition to this, it can also be expressed as an L2-norm, which makes it a good candidate for optimization.
The computational demand was however shown to be higher than the Par Detectability~\cite{taal2012low}, especially for larger number of input samples.
