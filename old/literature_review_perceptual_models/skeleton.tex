In order to build a perceptual sound zone algorithm, we review literature for perceptual models to find a suitable perceptual model.
\begin{itemize}
    \item I will discuss the criteria which will determine which perceptual model is chosen.
        \begin{itemize}
            \item Complexity
            \item Feasibility to optimize
        \end{itemize}
    \item I will discuss my literature review into perceptual models to find a model that best fits the criteria.
        \begin{itemize}
            \item Dau Model
            \item Detectability Models, i.e. Par and Taal
            \item Distraction Model
            \item Audio quality models, PEAQ, VISQOL
            \item Speech Intelligibility Based, i.e. SIIB and STOI
        \end{itemize}
    \item I will discuss and motivate what perceptual model will be used in the optimization problem.
        \begin{itemize}
            \item This is done by means of summarizing the findings, and then reflecting on the previously introduced
                criteria.
        From this, I will conclude that the \textbf{Par Detectability} is best suited.
        \end{itemize}
        
    \item I will discuss and motivate what perceptual models will be used for evaluation.
        From this I will conclude that the speech intelligibility metrics are suitable.
\end{itemize}
