In the preceding chapter it was concluded that a pressure matching approach was best suited for building a 
perceptual sound zone algorithm.
In this chapter, a reference pressure matching implementation will be given that will function as a the basis for the 
perceptual algorithm to be introduced in later chapters.
\begin{itemize}
    \item Introduction of the data model
        \begin{itemize}
            \item Mathematical description of the room, loudspeakers, zones, room impulse responses.
            \item Definition of the sound pressure in their room, how it relates to the loudspeaker input signals.
        \end{itemize}
    \item Introduction of the Multi-Zone Pressure-Matching (MZ-PM) approach.
        \begin{itemize}
            \item Contains the complete derivation of the approach, starting from the previously introduced data model.
        \end{itemize}
    \item Extension of the MZ-PM approach to work on a short-time scale
        \begin{itemize}
            \item The motivation for this is that the previously-derived approach requires knowledge of the full
                time domain signal, which is unrealistic in practice.
            \item In addition to this, the perceptual model works on a short time-scale
        \end{itemize}
    \item Extension of the short-time MZ-PM approach to work in the frequency domain.
        \begin{itemize}
            \item This is a requirement for the perceptual model, as it functions in the STFT domain.
        \end{itemize}
\end{itemize}
