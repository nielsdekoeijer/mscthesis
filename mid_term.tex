\documentclass[aspectratio=169]{beamer}
\usepackage[english]{babel}
\usepackage{calc}
\usepackage[absolute,overlay]{textpos}
\usepackage{graphicx}
\usepackage{xcolor}
\usepackage{subfig}
\usepackage{amsmath}
\usepackage{amsfonts}
\usepackage{amsthm}
\usepackage{mathtools}
\usepackage{comment}
\usepackage{MnSymbol,wasysym}
\usepackage{textcomp}
\usepackage{hyperref}
\usepackage{multimedia}

\usepackage{nielstikz}
\usepackage{nielstex}
\usetikzlibrary{patterns}
\pgfdeclarepatternformonly{mydots}{\pgfqpoint{-1pt}{-1pt}}{\pgfqpoint{1pt}{1pt}}{\pgfqpoint{3pt}{3pt}}% original definition: \pgfqpoint{3pt}{3pt}
{%
    \pgfpathcircle{\pgfqpoint{0pt}{0pt}}{.5pt}%
    \pgfusepath{fill}%
}%

\setbeamertemplate{navigation symbols}{} % remove navigation symbols
\mode<presentation>{\usetheme{tud}}

% BIB SETTINGS
\usepackage[backend=bibtex,firstinits=true,maxnames=30,maxcitenames=20,url=false,style=authoryear]{biblatex}
\bibliography{bib}
\setlength\bibitemsep{0.3cm} % space between entries in the reference list
\renewcommand{\bibfont}{\normalfont\scriptsize}
\setbeamerfont{footnote}{size=\tiny}
\renewcommand{\cite}[1]{\footnote<.->[frame]{\fullcite{#1}}}
\AtBeginBibliography{\tiny}

\title[]{\textbf{Sound Zones with a Cost Function Based on Models of Human Hearing}\\
{\normalsize Midterm Presentation for CAS Group}}
\institute[]{Delft University of Technology, The Netherlands\\Bang and Olufsen, Denmark}
\author{Niels de Koeijer}

\begin{document}

% Title page
{\setbeamertemplate{footline}{\usebeamertemplate*{minimal footline}} \frame{\titlepage}}

% ==========================================================================================================
\begin{frame}{\textbf{Presentation Outline}}
    \begin{block}{\textbf{Table of Contents}}
        \begin{enumerate}
            \item \textbf{Introduction}
            \item Introduction to Sound Zones
            \item Perceptual Models 
            \item Perceptual Sound Zones
        \end{enumerate} 
    \end{block}
\end{frame}

\begin{frame}{\textbf{Introduction I:} About Me} 
    \begin{columns}[c]
        \column{.55\textwidth}
        \textbf{Niels de Koeijer}\\
        Master Student Currently doing Thesis at the Research Department at Bang \& Olufsen (B\&O)\\
        \vspace{0.60cm}
        Supervised by: 
        \begin{itemize}
            \item Richard Hendriks (TUD)
            \item Jorge Martinez (TUD)
            \item Martin Bo M\o ller (B\&O)
            \item Pablo Martinez-Nuevo (B\&O)
        \end{itemize}
        \column{.33\textwidth}
        \begin{figure}
            \vspace{-0.8cm}
            \centering
            \includegraphics[scale=0.17]{me.jpg}
        \end{figure}
    \end{columns}
\end{frame}

% ==========================================================================================================
\begin{frame}{\textbf{Presentation Outline}}
    \begin{block}{\textbf{Table of Contents}}
        \begin{enumerate}
            \item Introduction
            \item \textbf{Introduction to Sound Zones}
            \item Perceptual Models 
            \item Perceptual Sound Zones
        \end{enumerate} 
    \end{block}
\end{frame}

\begin{frame}{\textbf{Introduction to Sound Zones I:} What are Sound Zones?}
    In sound zones, we attempt to use an \textbf{array of loudspeakers} to create \textbf{multiple areas with different audio} in the \textbf{same room}.
    \begin{figure}
        \centering
        \scalebox{0.6}{\input{tikz-sound-zones-problem.tex}}
    \end{figure}
\end{frame}

\begin{frame}{\textbf{Introduction to Sound Zones II:} Introducing the Data Model}
    \begin{figure}
        \centering
        \scalebox{0.6}{\input{tikz-sound-zones-problem-quantities.tex}}
    \end{figure}
    \begin{itemize}
        \item Let $t_A^{(m)}[n]$ and $t_B^{(m)}[n]$ be the \textbf{desired target sound pressure} for point $m$.
        \item Let $x^{(l)}[n]$ be the \textbf{loudspeaker input signals} for loudspeaker $l$.
        \item Let $h_A^{(l,m)}[n]$ and $h_B^{(l,m)}[n]$  be the \textbf{RIRs} from loudspeaker $l$ to point $m$.
    \end{itemize}
\end{frame}

\begin{frame}{\textbf{Introduction to Sound Zones III:} The Problem}
    Find \textbf{loudspeaker input signals} $x^{(l)}[n]$ such that, when transfered into the room by the \textbf{RIRs} $h_A^{(l,m)}[n]$ and $h_B^{(l,m)}[n]$, 
    achieve the \textbf{target sound pressures} $t_A^{(m)}[n]$ and $t_B^{(m)}[n]$.
    \begin{figure}
        \centering
        \scalebox{0.6}{\input{tikz-sound-zones-problem-quantities.tex}}
    \end{figure}
\end{frame}

\begin{frame}{\textbf{Introduction to Sound Zones IV:} Typical Approaches}
    Typically, this problem is solved by finding the \textbf{loudspeaker input signals} $x^{(l)}[n]$ that minimizes the difference in sound pressure.
    One way we could write this is: 
    \begin{align*}
        \argmin{x^{(l)}[n]}{
            & \sum_{m\in A} \norm[2][2]{\sum_l^{N_L} \left(h_A^{(l,m)} \ast x^{(l)}\right)[n] - t_A^{(m)}[n]} + \\
            & \sum_{m\in B} \norm[2][2]{\sum_l^{N_L} \left(h_B^{(l,m)} \ast x^{(l)}\right)[n] - t_B^{(m)}[n] } 
        }
    \end{align*}
\end{frame}

% ==========================================================================================================
\begin{frame}{\textbf{Presentation Outline}}
    \begin{block}{\textbf{Table of Contents}}
        \begin{enumerate}
            \item Introduction
            \item Introduction to Sound Zones
            \item \textbf{Perceptual Models} 
            \item Perceptual Sound Zones
        \end{enumerate} 
    \end{block}
\end{frame}

\begin{frame}{\textbf{Perceptual Models I:} Why Add Perceptual information?}
    Reasons to include a model of human hearing to sound zones:
    \begin{itemize}
        \item As mentioned, typical algorithms take an approach like this:
            \begin{align*}
                \argmin{x^{(l)}[n]}{
            & \sum_{m\in A} \norm[2][2]{\sum_l^{N_L} \left(h_A^{(l,m)} \ast x^{(l)}\right)[n] - t_A^{(m)}[n]} + \\
            & \sum_{m\in B} \norm[2][2]{\sum_l^{N_L} \left(h_B^{(l,m)} \ast x^{(l)}\right)[n] - t_B^{(m)}[n] }}
            \end{align*}

        \item \textbf{Problem:} Typical algorithms attempt to recreate the target sound pressure perfectly. 
            However, physical metrics do not directly relate to how sound is actually experienced.
    \end{itemize}
\end{frame}

\begin{frame}{\textbf{Perceptual Models I:} Why Add Perceptual information? Cont.}
    \textbf{Solution:} Use perceptual model of the human hearing, allowing us to focus on what matters perceptually! 
    More effective use of effort $\rightarrow$ better experience.
    Indications from literature show that this is promising.
\end{frame}


\begin{frame}{\textbf{Perceptual Models II:} The Approach}
    The following approach is taken:
    \begin{itemize}
        \item We include perceptual information in our algorithms by including it in the cost function.
        \item In order to do so, we should first find perceptual models that have nice properties for optimization.
            Examples of the properties that I considered were:
            \begin{enumerate}
                \item \textbf{Convexity:} to preserve a unique minimum and for potentially efficient solvers.
                \item \textbf{Computational Complexity:} in practice, a sound zone system cannot optimize for days just to play 4 minutes of audio.
                    Therefore, perceptual models with real-time potential are interesting.
            \end{enumerate}
    \end{itemize}
\end{frame}

\begin{frame}{\textbf{Perceptual Models III:} The Perceptual Detectibility}
    I settled on the \textbf{Detectibility}, a perceptual metric introduced by van der Par\cite{van2005perceptual} en Taal\cite{taal2012low}.
    \begin{itemize}
        \item Detectibility is defined as a function $D(x[n], \epsilon[n])$ which quantifies how well humans can detect an 
            \textbf{interferer} $\epsilon[n]$ is in presence of \textbf{signal} $x[n]$.
            \begin{itemize}
                \item If $\epsilon[n]$ is easily detectibile, $D(x[n], \epsilon[n]) \gg 1$ 
                \item If $\epsilon[n]$ is undetectible, $D(x[n], \epsilon[n]) \ll 1$ 
            \end{itemize}
        \item Originally designed for use in \textbf{audio coding}, where they want make the quantization error undetectible. 
    \end{itemize}
\end{frame}

\begin{frame}{\textbf{Perceptual Models III:} The Perceptual Detectibility Cont.}
    In order to get some intuition on how Detectibility works:
    \begin{itemize}
        \item $D(\text{\textbf{Jet Engine}},\,\text{\textbf{Mouse Sounds}}) \ll 1$:\\The detectibility of ``\textbf{Mouse Sounds}'' in presence of ``\textbf{Jet Engine}'' will be very low.
        \item $D(\text{\textbf{Mouse Sounds}},\,\text{\textbf{Jet Engine}}) \gg 1$:\\The detectibility of ``\textbf{Jet Engine}'' in presence of ``\textbf{Mouse Sounds}'' will be very high. 
    \end{itemize}
    Detectibility quantifies how well humans can hear an interferer given a background signal.
\end{frame}

\begin{frame}{\textbf{Perceptual Models IV:} Detectibility Facts}
    I won't go into details on how detectibililty works exactly, but the following is important to remember:
    \begin{itemize}
        \item Detectibility $D(x[n],\epsilon[n])$  is \textbf{convex} in $\epsilon[n]$ and can be computed in \textbf{realtime}.
        \item Detectibility uses time-local psycho-acoustical phenomena, thus $x[n]$ and $\epsilon[n]$ must be considered in \textbf{short time frames} (i.e. 20 ms for speech).
    \end{itemize}
    The next step is to design a sound zone algorithm using the detectibililty. 
\end{frame}

% ==========================================================================================================
\begin{frame}{\textbf{Presentation Outline}}
    \begin{block}{\textbf{Table of Contents}}
        \begin{enumerate}
            \item Introduction
            \item Introduction to Sound Zones
            \item Perceptual Models 
            \item \textbf{Perceptual Sound Zones}
        \end{enumerate} 
    \end{block}
\end{frame}

\begin{frame}{\textbf{Detectibility Based Sound Zones I:} How to use Detectibility} 
    Detectibility is interesting, but how can it be used? Consider again the typical algorithm. 
    \begin{align*}
        \argmin{x^{(l)}[n]}{
            & \sum_{m\in A} \norm[2][2]{\sum_l^{N_L} \left(h_A^{(l,m)} \ast x^{(l)}\right)[n] - t_A^{(m)}[n]} + \\
            & \sum_{m\in B} \norm[2][2]{\sum_l^{N_L} \left(h_B^{(l,m)} \ast x^{(l)}\right)[n] - t_B^{(m)}[n] } 
        }
    \end{align*}
    Here, we attempt to minimize the reproduction error between the target sound pressure and the sound pressure reproduced by the loudspeakers. 
\end{frame}

\begin{frame}{\textbf{Detectibility Based Sound Zones I:} How to use Detectibility Cont.} 
    Let the reproduction error $\epsilon_A^{(m)}[n]$ and $\epsilon_B^{(m)}[n]$ be defined as: 
    \begin{align*}
        \epsilon_A^{(m)}[n] &= \sum_l^{N_L} \left(h_A^{(l,m)} \ast x^{(l)}\right)[n] - t_A^{(m)}[n] \\
        \epsilon_B^{(m)}[n] &= \sum_l^{N_L} \left(h_B^{(l,m)} \ast x^{(l)}\right)[n] - t_B^{(m)}[n]
    \end{align*}
    The previous problem can be written as: 
    \begin{align*}
        \argmin{x^{(l)}[n]}{
            & \sum_m^{N_A} \norm[2][2]{\epsilon_A^{(m)}[n]} + 
            \sum_m^{N_B} \norm[2][2]{\epsilon_B^{(m)}[n]} 
        }
    \end{align*}
\end{frame}

\begin{frame}{\textbf{Detectibility Based Sound Zones I:} How to use Detectibility Cont.} 
    We can rewrite definition of the reproduction error to the following expression
    \begin{align*}
        \sum_l^{N_L} \left(h_A^{(l,m)} \ast x^{(l)}\right)[n] = t_A^{(m)}[n] + \epsilon_A^{(m)}[n] \\
        \sum_l^{N_L} \left(h_B^{(l,m)} \ast x^{(l)}\right)[n] = t_B^{(m)}[n] + \epsilon_B^{(m)}[n] 
    \end{align*}
    Here, we have written the sound pressure reproduced by the loudspeakers as combination of the target pressure and the reproduction error.
    So, we want to \textbf{minimize the reproduction error in presence of the target pressure}\\ 
    \vspace{10pt}
    This perspective motivates minimizing the \textbf{detectibility} of the reproduction error in presence of the target. 
\end{frame}

\begin{frame}{\textbf{Detectibility Based Sound Zones I:} How to use Detectibility Cont.} 
    This line of thinking results in the following optimization problem.
    \begin{align*}
        &\argmin{x^{(l)}[n]}{
              \sum_{m\in A} D\left(t_A^{(m)}[n],\,\epsilon_A^{(m)}[n]\right) + 
              \sum_{m\in B} D\left(t_B^{(m)}[n],\,\epsilon_B^{(m)}[n]\right) 
        } 
    \end{align*}
    As can be seen, the \textbf{detectibility} of the reproduction error is minimized in presence of the target pressure.
    This problem is convex, as detectibility $D(x[n], \epsilon[n])$ is convex in $\epsilon[n]$. 
\end{frame}

\begin{frame}{\textbf{Detectibility Based Sound Zones II:} Problems with this Approach} 
    \begin{align*}
        &\argmin{x^{(l)}[n]}{
              \sum_{m\in A} D\left(t_A^{(m)}[n],\,\epsilon_A^{(m)}[n]\right) + 
              \sum_{m\in B} D\left(t_B^{(m)}[n],\,\epsilon_B^{(m)}[n]\right) 
        } 
    \end{align*}
    \textbf{Problem with the Current Problem:} It penalizes audio leakage between zones as an reproduction error.
        In practice, it is very difficult to supress all leakage. 
        The problem will still attempt to do so, ruining audio quality. 
    \begin{figure}
        \centering
        \scalebox{0.5}{\input{tikz-sound-zones-problem-quantities.tex}}
    \end{figure}
\end{frame}

\begin{frame}{\textbf{Detectibility Based Sound Zones II:} Problems with this Approach} 
    \textbf{Solution:} Rewrite the optimization problem in such a way that audio quality is always preserved.
        In essence, we attempt to minimize the audio leakage subject to a constraint that preserves audio quality.
        This can be viewed as a relaxed version of the problem, relaxing in the sense that we allow for zone leakage.\\
        \vspace{10pt}
        {\color{red} 
            This is what I'm currently working on! 
            I would love to show you what I have done so far, but I'm afraid time doesn't allow for it...
        }\\
        \vspace{10pt}
        Future work for my project:
        \begin{itemize}
            \item Implement Relaxed Algorithm.
            \item Quantify Improvements Perceptual vs Non-Perceptual: how good is it?
        \end{itemize}
\end{frame}

\begin{frame}{\textbf{Questions / Suggestions / Anything Whatsover?}}
    {\tiny \color{black} Ideas always welcome! Send me an e-mail: N.E.M.dekoeijer@student.tudelft.nl}\\
    \printbibliography
\end{frame}

\begin{frame}{\textbf{Detectibility Based Sound Zones III:} Relaxed Problem} 
    If I somehow managed to do it under 20 minutes, here you go:
    \begin{align*}
        \argmin{x_A^{(l)},\,x_B^{(l)}[n]}{
        &\sum_{m\in A} D\left(t_A^{(m)}[n],\,\sum_l^{N_L} \left(h_A^{(l,m)} \ast x_B^{(l)}\right)[n]\right) +  
         \sum_{m\in A} D\left(t_B^{(m)}[n],\,\sum_l^{N_L} \left(h_B^{(l,m)} \ast x_A^{(l)}\right)[n]\right) 
        }\\ 
    \subjectto{
        &D\left(t_A^{(m)}[n],\,\sum_l^{N_L} \left(h_A^{(l,m)} \ast x_A^{(l)}\right)[n] - t_A^{(m)}[n]\right) \leq Q_0,\quad \forall\,m\in A \\
        &D\left(t_B^{(m)}[n],\,\sum_l^{N_L} \left(h_B^{(l,m)} \ast x_B^{(l)}\right)[n] - t_B^{(m)}[n]\right) \leq Q_0,\quad \forall\,m\in B 
    }
    \end{align*}
    Then we reconstruct $x^{(l)}[n] = x_A^{(l)}[n] + x_B^{(l)}[n]$.
\end{frame}

\end{document}
