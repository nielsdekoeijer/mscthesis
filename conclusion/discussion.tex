This work seeks to explore the implementation and possible benefits of including perceptual information in 
sound zone algorithms.
To this end, the introduction posed two research questions.

\begin{center}
    {\textit{``How can auditory perceptual models be included in sound zone algorithms?''}}\\
\end{center}
In this work it is shown that a perceptual sound zone algorithm can be stated directly using perceptual models. 
Specifically, the Par detectability distortion can be used to find a perceptual variant of the 
popular pressure-matching approach from literature.
This results in a sound zone algorithm that minimizes a cost function with a perceptual interpretation.

\begin{center}
    {\textit{``What are the benefits of including auditory perceptual models in sound zone algorithms?''}}\\
\end{center}
Current findings suggest that the benefits of including perceptual models are twofold:
\begin{itemize}
    \item The perceptual sound zone algorithm outperforms the reference
    \item The use of perceptual constraints.
\end{itemize}
