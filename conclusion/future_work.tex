The following work is found to be promising:
\begin{itemize}
    \item Reformulating the short-time frequency-domain pressure-matching defined in \autoref{ch:perceptual_sound_zone:stft}
        such that it allows for optimization over the frequency-domain representation of the loudspeaker input signals while 
        limiting or entirely preventing the resulting artifacts in the resulting time-domain sequence.

        This is found to be promising as the resulting computational complexity of the algorithm could potentially be much
        lower due to the removal of the transformation between the time-domain and frequency-domain representation of the
        loudspeaker input signals.
    \item Reformulating the constrained perceptual pressure matching algorithm given in \autoref{ch:perceptual_sound_zone:perceptual_minimization:constrained}
        to instead constrain the leakage error detectability.

        This approach is promising as it allows for constraints to be placed on the detectability sound zone leakage which have 
        a consistent interpretation.
        This approach can then be compared to traditional pressure matching approaches which often also constrain sound zone leakage~\cite{betlehem2015personal}.
    \item Evaluation of the performance difference between perceptual and non-perceptual sound zone approaches.
        In \autoref{ch:results:evaluation}, the perceptual sound zone algorithms are shown to outperform the traditional non-perceptual reference in terms of 
        various objective measures which are introduced in \autoref{ch:perceptual:review}.

        However, as discussed in \autoref{ch:perceptual:review}, these objective measures can only be used to give an indication of performance.
        To objectively determine if the perceptual approach does indeed outperform traditional approaches, a listening test must be conducted.
\end{itemize}
