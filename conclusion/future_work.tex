While this work successfully answers its research questions, 
there are still many promising directions of perceptual sound zones research.
The following future work is found to be of interest:
\begin{itemize}
    \item As discussed, it is shown that the unconstrained perceptual pressure matching algorithm outperforms 
        the reference non-perceptual pressure matching algorithm in terms of various objective perceptual measures.
        However, as discussed in \autoref{ch:perceptual:review}, 
        these objective measures can only be used to give an indication of performance.

        To objectively determine if the perceptual approach does indeed outperform traditional approaches, 
        formal listening tests must be conducted.

    \item As shown, the constrained perceptual pressure matching approach can be used to control the reproduced audio quality through the detectability of the reproduction error.
        However, the degree to which this is possible with a non-perceptual pressure matching approach is not explored in this work.

        It is of interest to compare the performance of perceptual and non-perceptual constraints to obtain a complete understanding of the differences.

    \item The proposed perceptual sound zone framework can be used to formulate more perceptual sound zone 
        algorithms.
        One algorithm that is of particular interest is an algorithm that constrains the detectability of 
        the interference rather than the reproduction error. 
        This can then be readily compared to non-perceptual pressure matching approaches, which often 
        include a similar, non-perceptual constraint.

    \item Currently, the proposed perceptual sound zone algorithms are posed as optimization problems that use both time and frequency domain representations of the optimizers.
        The translation between domains is suspected to greatly increase the computational complexity of the algorithm.

        As such, it is of interest to obtain a version of the algorithm that operates in a single domain to reduce the computational complexity. 
\end{itemize}
