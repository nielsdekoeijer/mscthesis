The two main approaches in sound zone literature are ``pressure matching'' (PM) and ``acoustic contrast control'' (ACC).
Pressure matching is used as the main inspiration for the perceptual sound zone framework 
proposed in \autoref{ch:sound_zone:approach_selection}, which is in turn used to propose perceptual sound zone algorithms 
in \autoref{ch:perceptual_sound_zone}.
A description of acoustic contrast control is included for completeness.
This section introduces and describes both approaches using the previously derived data model to sketch their mathematical properties.

Classically, the sound zone problem is divided up into subproblems as described in the introduction of this chapter.
The resulting loudspeaker input signals $x^{(l)}[n]$ are determined for a single bright-dark zone pair:
the loudspeaker input signals are found such that the target audio is achieved in the bright zone, while leakage is minimized in the dark zone.
If a solution for multiple zones is desired, multiple problems must be solved independently and their resulting loudspeaker input signals combined~\cite{betlehem2015personal}.

There is another approach, however.
In a multi-zone approach, the loudspeaker input signals are instead determined jointly for all zones, 
rather than decomposing into bright-dark zone pairs.

A multi-zone approach is taken in this thesis, as it is found to be more general.
For simplicity, but without loss of generality, this thesis limits the number of zones to two.
The approach is, however, generalizable to any multiplicity of zones.

In a two zone multi-zone approach, the loudspeaker input signals $x^{(l)}[n]$ are decomposed into two parts as follows:
\begin{equation}
    x^{(l)}[n] = x_\za^{(l)}[n] + x_\zb^{(l)}[n].
\end{equation}
Here, $x_\za^{(l)}[n]$ and $x_\zb^{(l)}[n]$ are the parts of the loudspeaker input signal responsible for reproducing the target sound pressure 
in zone $\za$ and $\zb$ respectively.

Through this decomposition, it is possible to consider the sound pressure that arises at a specified control point due to the separate loudspeaker input signals:
\begin{align}
    p_\zz^{(m)}[n] &= \sum_{l=0}^{N_L} \left(h^{(l,m)} \ast x_\zz^{(l)}\right)[n],
\end{align}
\label{eq:sound_zone:approaches:pressure}
where $\zz \in \left(\za,\,\zb\right)$ represents either zones.
Here, $p_\za^{(m)}[n]$ and $p_\zb^{(m)}[n]$ can be understood to be the achieved sound pressure that arises due to 
playing loudspeaker input signals $x_\za^{(l)}[n]$ and $x_\zb^{(l)}[n]$ respectively. 

The total achieved sound pressure at control point $m$ is then given by the addition of the two achieved sound pressures:
\begin{equation}
    p^{(m)}[n] = p_\za^{(m)}[n] + p_\zb^{(m)}[n].
\end{equation}

This decomposition is used to describe a multi-zone variant of both a pressure matching approach 
in \autoref{ch:sound_zone:approaches:pressure_matching},
and acoustic contrast control approach in \autoref{ch:sound_zone:approaches:acoustic_contrast_control}.

\subsection{Description of Pressure Matching}
\label{ch:sound_zone:approaches:pressure_matching}
The ``pressure matching'' (PM) approach is widely used in the literature to 
solve the sound zone problem~\cite{betlehem2015personal, moller2016sound}.
In this section, a multi-zone pressure matching algorithm is derived using the data model 
given in \autoref{ch:sound_zone:data_model}.

In pressure matching approaches, one attempts to design suitable loudspeaker input signals in such a way that the resulting sound pressure in the zone matches the specified target sound pressure for that zone. 
Simultaneously the sound pressure that results in other zones as interference or cross-talk due to reproducing the target in the bright zone is minimized~\cite{betlehem2015personal, olik2013comparative}.

This goal can be stated formally as choosing $x_\za^{(l)}[n]$ such that the resulting achieved pressure 
$p_\za^{(m)}[n]$ attains the target sound pressure $t^{(m)}[n]$ in all control points $m \in A$.   
At the same time, however, $p_\za^{(m)}[n]$ should result in minimal sound pressure in all control points $m \in B$.
Any sound pressure resulting from $x_\za^{(l)}[n]$ in zone $\zb$ can be understood as leakage or cross-talk between the zones. 
Similar arguments can be given for $x_\zb^{(l)}[n]$.

An optimization problem that achieves this goal is formulated as follows:
\begin{align}
    \begin{aligned}
        \argmin{x_\za^{(l)}[n],\,x_\zb^{(l)}[n]\,\forall\,l}{
           &\sum_{m\in A} \norm[2][2]{p_\za^{(m)}[n] - t^{(m)}[n]} +
            \sum_{m\in A} \norm[2][2]{p_\zb^{(m)}[n]} + \\
           &\sum_{m\in B} \norm[2][2]{p_\zb^{(m)}[n] - t^{(m)}[n]} + 
            \sum_{m\in B} \norm[2][2]{p_\za^{(m)}[n]}
        },
    \end{aligned}
\end{align}
where the $\norm[2][2]{\,\,\cdot\,\,}$ operator denotes the squared $L^2$-norm.
In this case, the norm is taken over the time-domain sequence indexed by $n$.

To further understand the optimization problem, consider the following definitions: 
\begin{alignat}{2}
    \text{RE}^{(m)}_\zz &= \norm[2][2]{p_\zz^{(m)}[n] - t^{(m)}[n]} \qquad&& \forall\,\, m\in Z, \label{eq:sound_zone:approaches:RE}\\
    \text{LE}^{(m)}_\zz &= \norm[2][2]{p_\zz^{(m)}[n]} \qquad&& \forall\,\, m\notin Z. \label{eq:sound_zone:approaches:LE} 
\end{alignat}
With the following interpretations:
\begin{itemize}
    \item $\text{RE}^{(m)}_\zz$ is the reproduction error for zone $\zz \in \left(\za,\,\zb\right)$ 
        for control point $m \in Z$.
        This error corresponds to how well the achieved sound pressure $p_\zz^{(m)}[n]$ matches the 
        target sound pressure $t^{(m)}[n]$ for a control point in the 
        bright zone. 
    \item $\text{LE}^{(m)}_\zz$ is the leakage error in zone $\zz \in \left(\za,\,\zb\right)$ for control point $m \notin Z$.
        This error can be understood as the total sound energy that ``leaks'' into control point $m$ in zones 
        other than $\zz$ when attempting to reproduce the target sound pressure $t^{(m)}[n]$ in zone $\zz$. 
        This can be also be understood as the ``interference'' or ``cross-talk'' between zones.
\end{itemize}
Using these definition we can rewrite the optimization problem:
\begin{align}
    \argmin{x_\za^{(l)}[n],\,x_\zb^{(l)}[n]\,\forall\,l}{
       &\sum_{m\in A} \text{RE}^{(m)}_\za +  \sum_{m\in B} \text{LE}^{(m)}_\za + \sum_{m\in B} \text{RE}^{(m)}_\zb + \sum_{m\in A} \text{LE}^{(m)}_\zb
    }.
\end{align}
From this, it becomes clear that this approach results in a trade-off between minimizing the reproduction errors $\text{RE}^{(m)}_\zz$ 
and leakage errors $\text{LE}^{(m)}_\zz$. 

Some pressure matching approaches attempt to control this trade-off by introducing weights for the different error terms, 
or by adding constraints.
Choosing constraints can, however, be challenging as the squared $L^2$ pressure error does not always correlate well with how
the error is perceived by humans.

\subsection{Description of Acoustic Contrast Control}
\label{ch:sound_zone:approaches:acoustic_contrast_control}
The ``acoustic contrast control'' (ACC) method is another widely used sound zone approach from literature.
The ACC approach attempts to maximize the acoustic contrast between the bright zone and the dark zone. 
Acoustic contrast is the ratio of the total sound energy of the bright and dark zones.
Essentially, the goal is to maximize the difference in sound pressure level between the bright and dark zones.

In this section, a multi-zone ACC algorithm is described.
As the previously described data model is in the time domain, this approach will take inspiration from a time-domain approach found in literature known as the
broadband acoustic contrast control (BACC) approach~\cite{elliott2011regularisation, cai2014time, moller2016sound}.

In contrast to the multi-zone PM approach, the multi-zone ACC approach does not optimize directly over 
the loudspeaker input signals $x_\za^{(l)}[n]$ and $x_\zb^{(l)}[n]$.
Instead, it indirectly controls the loudspeaker input signals by optimizing over 
FIR filter coefficients $w_\za^{(l)}[n]\in\Real{N_w}$ and $w_\zb^{(l)}[n]\in\Real{N_w}$.
These filters are applied to the desired playback signals $s_\za^{(l)}$ and $s_\zb^{(l)}$ 
respectively to form the final loudspeaker input signals.

This relationship between the loudspeaker input signals and the filter coefficients is thus given as follows:
\begin{equation}
    x_\zz^{(l)}[n] = \left(w_\zz^{(l)} \ast s_\zz\right)[n].
\end{equation}
This definition also relates the filter coefficients to the resulting sound pressure 
through \autoref{eq:sound_zone:data_model:achieved_pressure}.

As mentioned, the goal of the ACC approach is to maximize the acoustic contrast between bright and dark zones,
which is defined as the ratio between the sound energy in the bright and dark zones.
The total sound energy in a zone will be defined as the sum of squares of the sound pressure in a control point.
As such, the acoustic contrast $\text{AC}_\zz$ for a zone $\zz$ can be defined as follows: 
\begin{equation}
    \text{AC}_{\zz} = \frac{\sum_{m\in Z} \norm[2][2]{p_\zz^{(m)}[n]}}{\sum_{m\notin Z} \norm[2][2]{p_\zz^{(m)}[n]}}.
\end{equation}
In an ACC approach, the goal is to maximize the total acoustic contrast.
Thus, consider the following optimization problem:
\begin{align}
    \argmax{w_\za^{(l)}[n],\,w_\zb^{(l)}[n]\,\forall\,l}{
       &\text{AC}_\za + \text{AC}_\zb
    }.
\end{align}
As mentioned, the optimization is performed over the loudspeaker filter coefficients rather than over the loudspeaker input signals.

Acoustic contrast control is discussed mainly for completion as an alternative to the pressure matching approach.
As is discussed in \autoref{ch:sound_zone:approach_selection}, 
pressure matching is the main inspiration for the proposed perceptual sound zone approach used in the proposed perceptual sound zone algorithm.
