The two main approaches in sound zone literature are pressure matching (PM) and acoustic contrast control (ACC).
One of these two approaches will be used in the perceptual sound zone algorithm to be introduced in \autoref{ch:perceptual_sound_zone}.
This section will mathematically introduce both approaches with the goal of sketching their mathematical properties.
This will then later be used determine in \autoref{} if it is suitable for integration with the Par detectability discussed in \autoref{ch:perceptual}.

In the typical sound zone approach, the resulting loudspeaker input signals $x^{(l)}[n]$ are determined for just a bright-dark zone pair.
Here, the loudspeaker input signals are found such that the a target audio is achieved in the bright zone, while leakage is minimized in the dark zone.
If the solution for multiple zones is desired, than multiple problems must be solved and their resulting loudspeaker input signals combined, as discussed
in \autoref{ch:sound_zone:problem}. 

In a multi-zone approach, the loudspeaker input signals are instead determined for jointly for all zones, rather than decomposing into bright-dark zone pairs.
This is the approach that will be taken in this thesis.
For simplicity, this thesis will focus on a situation with two zones\footnote{The approach is however generalizable to any multiplicity of zones}.

In a two zone approach, the loudspeaker input signals $x^{(l)}[n]$ are decomposed into two parts as follows:
\begin{equation}
    x^{(l)}[n] = x_\za^{(l)}[n] + x_\zb^{(l)}[n]
\end{equation}
Here, $x_\za^{(l)}[n]$ and $x_\zb^{(l)}[n]$ are the parts of the loudspeaker input signal responsible for reproducing the target sound pressure 
in zone $\za$ and $\zb$ respectively.

Through this decomposition, it is possible to consider the sound pressure that arises due to the separate loudspeaker input signals:
\begin{align}
    p_\zz^{(m)}[n] &= \sum_{l=0}^{N_L} \left(h^{(l,m)} \ast x_\zz^{(l)}\right)[n] 
\end{align}
Where $\zz \in \left(\za,\,\zb\right)$ represents either zones.
Here, $p_\za^{(m)}[n]$ and $p_\zb^{(m)}[n]$ can be understood to be the pressure that arises due to 
playing loudspeaker input signals $x_\za^{(l)}[n]$ and $x_\zb^{(l)}[n]$ respectively. 
The total sound pressure is then given by the addition of the two sound pressures:
\begin{equation}
    p^{(m)}[n] = p_\za^{(m)}[n] + p_\zb^{(m)}[n]
\end{equation}

What follows is using this decomposition to describe a multi-zone variant of both pressure matching in \autoref{ch:sound_zone:approaches:pressure_matching}
and acoustic contrast control in \autoref{ch:sound_zone:approaches:acoustic_contrast_control}.

\subsection{Pressure Matching}
\label{ch:sound_zone:approaches:pressure_matching}
The ``Pressure Matching'' (PM) is widely used in literature to solve the sound zone problem.
In this section, a ``Multi-Zone Pressure Matching'' (MZ-PM) algorithm will be derived using the previously derived data model.

In pressure matching approaches (PM), one attempts to control the output of the loudspeaker array in such a way that resulting sound pressure in the zone 
matches the specified target sound pressure for that zone.
While doing so, the PM approach attempts to minimize the sound pressure that results in other zones as to minimize the interference or crosstalk between zones
~\cite{olik2013comparative, betlehem2015personal}.

The idea in this approach is to chose $x_\za^{(l)}[n]$ and such that the resulting pressure $p_\za^{(m)}[n]$ attains the target sound pressure $t^{(m)}[n]$ in all $m \in A$.   

At the same time however, $p_\za^{(m)}[n]$ should not result in any sound pressure in all $m \in B$.
Any sound pressure resulting from $x_\za^{(l)}[n]$ in zone $\zb$ is essentially leakage, or cross-talk between zones. 
Similar arguments can be given for $x_\zb^{(l)}[n]$.

In the MZ-PM approach, the loudspeaker input signals $x_\za^{(l)}[n]$ and $x_\zb^{(l)}[n]$ that attain the target with minimal leakage can be found by 
minimizing the difference between the intended pressure and the realized pressure as follows:
\begin{align}
    \argmin{x_\za^{(l)}[n],\,x_\zb^{(l)}[n]\,\forall\,l}{
       &\sum_{m\in A} \norm[2][2]{p_\za^{(m)}[n] - t^{(m)}[n]} +
        \sum_{m\in A} \norm[2][2]{p_\zb^{(m)}[n]} + \\
       &\sum_{m\in B} \norm[2][2]{p_\zb^{(m)}[n] - t^{(m)}[n]} + 
        \sum_{m\in B} \norm[2][2]{p_\za^{(m)}[n]}
    }
\end{align}

Here, the first two terms can be understood as the reproduction error and the leakage for zone $\za$.
Similarly, the last two terms are the reproduction error and leakage for zone $\zb$. 
To make this more clear, the following definitions are introduced:
\begin{align}
    \text{RE}_\zz &= \sum_{m\in A} \norm[2][2]{p_\za^{(m)}[n] - t^{(m)}[n]} \\
    \text{LE}_\zz &= \sum_{m\in A} \norm[2][2]{p_\zb^{(m)}[n]} 
\end{align}

Here, $\text{RE}_\zz$ is the reproduction error and $\text{LE}_\zz$ is the leakage error in zone $\zz \in \left(\za,\,\zb\right)$.
This allows for the following rewrite of the previously introduced optimization problem:

\begin{align}
    \argmin{x_\za^{(l)}[n],\,x_\zb^{(l)}[n]\,\forall\,l}{
       &\text{RE}_\za + \text{LE}_\za + \text{RE}_\zb + \text{LE}_\zb
    }
\end{align}

From this it becomes clear that this approach results in trade-off between minimizing the reproduction errors $\text{RE}_\zz$ 
and leakages $\text{LE}_\zz$. 
Some pressure matching approaches attempt to control this trade-off by introducing weights for the different error terms, 
or by adding constraints.
Choosing constraints can however be challenging as the mean square pressure error is difficult to interpret.

\subsection{Acoustic Contrast Control}
\label{ch:sound_zone:approaches:acoustic_contrast_control}
``Acoustic Contrast Control'' is another widely used sound zone approach from literature.
The acoustic contrast control (ACC) approach to sound zones attempts to maximize the acoustic contrast between the bright zone and the dark zone for 
each specified target sound pressure.

The acoustic contrast is defined as the ratio of the acoustic potential energy of the bright zone and the dark zone.
Essentially, the goal is to maximize the difference in sound between the bright and dark zone.

In this section, a ``Multi-Zone Acoustic Contrast Control'' (MZ-ACC) algorithm will be described.
As the previously described data model is in the time domain, this approach will take inspiration from a time-domain approach found in literature known as the
broadband acoustic contrast control (BACC) approach~\cite{elliott2011regularisation, cai2014time, moller2016sound}.

In contrast to the MZ-PM approach, the MZ-ACC approach does not optimize directly over the loudspeaker input signals $x_\za^{(l)}[n]$ and $x_\zb^{(l)}[n]$.
Instead, it optimizes over filter coefficients $w_\za^{(l)}[n]\in\Real{N_w}$ and $w_\zb^{(l)}[n]\in\Real{N_w}$.

These filters are applied to the desired playback signals $s_\za^{(l)}$ and $s_\zb^{(l)}$ respectively to form the final loudspeaker input signals.
This relationship be expressed as follows:
\begin{equation}
    x_\zz^{(l)}[n] &= \left(w_\zz^{(l)} \ast s_\zz\right)[n] 
\end{equation}
As such, the MZ-ACC approach optimizes indirectly over the loudspeaker input signals.
Note that this definition also relates the filter coefficients and the resulting sound pressure as given by \autoref{}.

As mentioned, the goal of the ACC approach is to maximize the acoustic contrast between bright and dark zones.
Consider the acoustic contrast for zone $\za$.
This can be defined as the ratio of the total acoustic potential energy of the sound pressure in bright zone $\za$, and the total acoustic potential energy in 
dark zone $\zb$: 
\begin{equation}
    \text{AC}_{\za} = \frac{\sum_{m\in A} \norm[2][2]{p_\za^{(m)}}}{\sum_{m\in B} \norm[2][2]{p_\za^{(m)}}} 
\end{equation}

In an ACC approach, this contrast is maximized. 
To this end, consider the following optimization problem:
\begin{align}
    \argmax{w_\za^{(l)}[n],\,w_\zb^{(l)}[n]\,\forall\,l}{
       &\text{AC}_\za + \text{AC}_\zb
    }
\end{align}
Here, the total acoustic contrast is maximized.

As mentioned, the optimization is over the loudspeaker filter coefficients rather than over the loudspeaker input signals.
