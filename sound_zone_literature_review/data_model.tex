In the previous section, the sound zone problem was introduced heuristically.
In this section a mathematical framework for a room containing sound zones will be introduced.
This framework will be used later in the derivation of the sound zone algorithms in \autoref{ch:sound_zone:approaches}.

The contents of this section are as follows.
First, \autoref{ch:sound_zone:data_model:room_model} develops a spatial description of a room containing
two zones and a loudspeaker array.
Then, \autoref{ch:sound_zone:data_model:target_pressure} defines the objective of the sound zone algorithm formally
as realizing target sound pressure at discrete points in the room.

The relation between the sound pressure in the room and loudspeaker input signals will then be given in
\autoref{ch:sound_zone:data_model:realizing_pressure}, completing the mathematical framework.
This is then used in \autoref{ch:sound_zone:data_model:target_pressure_choice} to select a suitable target
sound pressure which will be used in the remainder of this thesis.

\subsection{Room Topology}
\label{ch:sound_zone:data_model:room_model}
A room $\mathcal{R}$ can be modeled as a closed subset of three dimensional space, $\mathcal{R} \subset \Real{3}$.
The two non-overlapping zones $\za$ and $\zb$ are contained within the room $\mathcal{R}$, 
i.e. $\za \subset \mathcal{R}$ and $\zb \subset \mathcal{R}$ where $\za \cap \zb = \emptyset$.
In general, the room can contain any number of zones, but this thesis will focus on the two zone case. 
In addition to the zones, the room $\mathcal{R}$ also contains $N_L$ loudspeakers, whose locations are modeled as discrete points.
An example of a possible room, loudspeakers and pair of zones are visualized in \autoref{fig:data_model:room_model:3D_room}.

\begin{figure}
    \centering
    \begin{tikzpicture} 
    \draw [draw=black] (0,0) rectangle (8,6);

    % Speakers on the top wall
    \pic[scale=0.7] at (2.5, 5.6) {Speaker};
    \pic[scale=0.7] at (3.5, 5.6) {Speaker};
    \pic[scale=0.7] at (4.5, 5.6) {Speaker};
    \pic[scale=0.7] at (5.5, 5.6) {Speaker};

    % Speakers on the bottom wall
    \pic[rotate=180, scale=0.7] at (2.5, 0.4) {Speaker};
    \pic[rotate=180, scale=0.7] at (3.5, 0.4) {Speaker};
    \pic[rotate=180, scale=0.7] at (4.5, 0.4) {Speaker};
    \pic[rotate=180, scale=0.7] at (5.5, 0.4) {Speaker};

    \draw[opacity=0.4, fill=blue] (6,3) circle[radius=1.5];
    \draw[thick] (6,3) circle (1.5) node[align=center] {\textbf{Zone $\text{B}$}};
    \draw[opacity=0.4, fill=red]  (2,3) circle[radius=1.5];
    \draw[thick] (2,3) circle (1.5) node[align=center] {\textbf{Zone $\text{A}$}};
\end{tikzpicture}

    \caption{A birds-eye view of a room $\mathcal{R}\subset \Real{3}$ containing the zones $\za\subset\mathcal{R}$ 
    and $\zb\subset\mathcal{R}$ depicted in red and blue respectively. 
    The room contains $N_L = 8$ loudspeakers, which are denoted by the red dots in the corners of the room.}
    \label{fig:data_model:room_model:3D_room}
\end{figure}

The goal of the sound zone algorithm is to use the loudspeakers to realise a specified target sound pressure
in the space described by zones $\za$ and $\zb$.
This is to be done in such a way that there is minimal interference between zones, 
meaning that target sound pressure intended for one zone should not be audible in the other zones.

The sound field generated by loudspeakers can be controlled by specifying their input signals.
As such, the goal of the sound zone algorithm is finding loudspeaker input signals in such a way that 
specified target sound pressure is attained.

The rest of this section will focus on formalizing this notion mathematically.

\subsection{Defining Target Pressure}
\label{ch:sound_zone:data_model:target_pressure}
As mentioned, the goal of the sound zone algorithm is to realize a specified target sound pressure
in the different zones $\za$ and $\zb$ in the room $\mathcal{R}$.

Currently, the zones are given as continuous regions in space.
However, most sound zone approaches will instead discretize the zones by sampling the continuous zones 
$\za$ and $\zb$ into so-called control points.
The sound pressure is then controlled only in these control points.

Thus, we discretize zones $\za$ and $\zb$ into a total of $N_a$ and $N_b$ control points respectively.   
Let $A$ and $B$ denote the sets of the resulting control points points contained within zones $\za$ and $\zb$ respectively.

Now let $t^{m}[n]$ denote the target sound pressure at control point $m$ in either $A$ or $B$, i.e. $m\in A \cup B$.
Our goal is thus to realize $t^{m}[n]$ in all control points $m\in A \cup B$ using the loudspeakers present in the room.
% The relationship between the loudspeaker input signals and the sound pressure is the topic of the next section.

\begin{figure}
    \centering
    \begin{tikzpicture} 
    \draw [draw=black] (0,0) rectangle (8,6);

    % Speakers on the top wall
    \pic[scale=0.7] at (2.5, 5.6) {Speaker};
    \pic[scale=0.7] at (3.5, 5.6) {Speaker};
    \pic[scale=0.7] at (4.5, 5.6) {Speaker};
    \pic[scale=0.7] at (5.5, 5.6) {Speaker};

    % Speakers on the bottom wall
    \pic[rotate=180, scale=0.7] at (2.5, 0.4) {Speaker};
    \pic[rotate=180, scale=0.7] at (3.5, 0.4) {Speaker};
    \pic[rotate=180, scale=0.7] at (4.5, 0.4) {Speaker};
    \pic[rotate=180, scale=0.7] at (5.5, 0.4) {Speaker};

    \draw[opacity=0.7, pattern=wide2] (6,3) circle[radius=1.5];
    \draw[opacity=0.4, fill=blue] (6,3) circle[radius=1.5];
    \draw[thick] (6,3) circle (1.5) node[align=center] {\textbf{Zone $\text{B}$}};
    \draw[opacity=0.7, pattern=wide2]  (2,3) circle[radius=1.5];
    \draw[opacity=0.4, fill=red]  (2,3) circle[radius=1.5];
    \draw[thick] (2,3) circle (1.5) node[align=center] {\textbf{Zone $\text{A}$}};
\end{tikzpicture}

    \caption{The previously introduced room $\mathcal{R}$ with zones $\za$ and $\zb$ discretized.}
\end{figure}

\subsection{Realizing Sound Pressure through the Loudspeakers}
\label{ch:sound_zone:data_model:realizing_pressure}
The sound pressure produced by the loudspeakers can be controlled by specifying their input signals.
Let $x^{(l)}[n]\in\Real{N_x}$ denote the loudspeaker input signal of length $N_x$ for the $l^\text{th}$ loudspeaker.
For now, it is assumed that the loudspeaker input signals are of finite length. 
In a later part of the thesis, a short-time formulation will be given which supports infinite length sequences. 

As such, the goal of the sound zone algorithm is to find loudspeaker inputs $x^{(l)}[n]$ 
such that the target sound pressure $t^{m}[n]$ is realized for all $m\in A \cup B$.

In order to do so, a relationship must be established between the loudspeaker inputs $x^{(l)}[n]$
and the resulting sound pressure at control points $m\in A \cup B$. 
This relationship can be modeled by room impulse responses (RIRs) $h^{(l,m)}[n]\in\Real{N_h}$.

The RIRs $h^{(l,m)}[n]$ determine the sound pressure at control point $m$ due to playing loudspeaker signal $x^{(l)}[n]$ from loudspeaker $l$. 
Mathematically, let $p^{(l,m)}[n]\in\Real{N_x + N_h - 1}$ represent said sound pressure. 
It can be defined as follows:
\begin{equation}
    p^{(l,m)}[n] = \left(h^{(l,m)} \ast x^{(l)}\right)[n]
\end{equation}
Here, the $\ast$ operator is used to denote linear convolution. 
The achieved sound pressure $p^{(l,m)}[n]$ only considers the contribution of loudspeaker $l$ at reproduction point $m$.
Let $p^{(m)}[n]\in\Real{N_x + N_h - 1}$ denote the total sound pressure due to all $N_L$ loudspeakers,
which can be expressed as the sum over all contributions $p^{(l,m)}[n]$ as follows: 
\begin{align}
    p^{(m)}[n] &= \sum_{l=0}^{N_L - 1} \left(h^{(l,m)} \ast x^{(l)}\right)[n]\label{eq:sound_zone:data_model:achieved_pressure}
\end{align}
With the data model completed, the goal of the sound zone algorithm can be restated formally.
Namely, the goal is to find $x^{(l)}[n]$ such that the achieved sound pressure $p^{(m)}[n]$ attains the
target sound pressure $t^{(m)}[n]$ for all control points $m\in A \cup B$.

\subsection{Choice of Target Pressure}
\label{ch:sound_zone:data_model:target_pressure_choice}
The target sound pressure $t^{(m)}[n]$ describes the desired content for a specific control point $m$. 
So far, the choice of target sound pressure $t^{(m)}[n]$ has been kept general. 
In this section, a choice for the target pressure will be made and motivated.

Assume that the users of the sound zone system have selected desired playback audio signals $s_\za[n]\in\Real{N_x}$ and
$s_\zb[n]\in\Real{N_x}$ that they wish to hear in zone $\za$ and $\zb$ respectively.
In order to accommodate the wishes of the user, the target sound pressure is chosen as follows: 
\begin{align}
    \begin{aligned}
        t^{(m)}[n] = \sum_{l=0}^{N_L} \left(h^{(l,m)} \ast s_\za\right)[n]\qquad &\forall\,m\in A\\
        t^{(m)}[n] = \sum_{l=0}^{N_L} \left(h^{(l,m)} \ast s_\zb\right)[n]\qquad &\forall\,m\in B
    \end{aligned}
    \label{eq:sound_zone:data_model:target_pressure}
\end{align}
This choice for the target pressure can be understood as the sound pressure that arises in a certain zone
when playing only the desired playback audio for that zone from the loudspeaker array. 
For example, when in zone $m\in A$, the target sound pressure is set equal to the sound pressure corresponding to 
the sound pressure that arises when playing only $s_\za[n]$ from the loudspeaker array.

The motivation for choosing this target is that it physically attainable in each zone seperately
with the given loudspeakers and room.
