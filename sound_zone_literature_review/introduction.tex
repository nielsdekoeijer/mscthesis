The goal of this chapter is find a suitable sound zone approach for integration with the perceptual model selected in \autoref{ch:perceptual}.
The chapter will start with \autoref{ch:sound_zone:approach_review} which will provide a review of various sound zone approaches from literature to provide a selection to choose 
from.

After documenting the state of the art, one sound zone algorithm will be selected in \autoref{ch:sound_zone:approach_selection} as the most promising 
for combination with the perceptual model selected in \autoref{ch:perceptual}.
This is done by reflecting on the mathematical properties of the selected perceptual model.

After selecting a sound zone approach, the rest of the chapter will focus on the derivation and the implementation of a sound zone algorithm that takes 
said approach.
This algorithm will serve as the basis upon which the perceptual sound zone algorithm will be built.
In addition to this, it will serve as a reference with which the perceptual sound zone algorithm can be compared.

To derive said sound zone algorithm, \autoref{ch:sound_zone:data_model} will start by formalizing the sound zone problem by introducing it in a 
mathematical framework.
This is followed by the implementation of the selected algorithm in \autoref{ch:sound_zone:approach_implementation}.

The implementation is then extended in \autoref{ch:sound_zone:block_based} to operate in a short-time block-based fashion, 
giving the algorithm the potential to run in real-time.
A frequency-domain version of the algorithm is then given in \autoref{ch:sound_zone:frequency_domain_conversion}.

The chapter ends with a summary and concluding remarks in \autoref{ch:sound_zone:conclusion}.
