The goal of this chapter is find a suitable sound zone approach for integration with the perceptual model selected in \autoref{ch:perceptual}.

The chapter will start with a brief description of sound zones in \autoref{ch:sound_zone:problem} to give the reader 
the necessary background information on the sound zone problem.
Afterwards, \autoref{ch:sound_zone:data_model} will flesh out the sound zone further by presenting a data model
used to describe the sound zone problem.

One of these three approaches will be used to construct a perceptual sound zone algorithm using the model selected in \autoref{ch:perceptual}.
To this end, \autoref{ch:sound_zone:constraints} will reflect on the mathematical properties of the Par detectability.
From this, a sound zone approach will be selected.

After selecting a sound zone approach, \autoref{ch:sound_zone:data_model} to \autoref{ch:sound_zone:frequency_domain_conversion} will focus on its implementation.
This algorithm will serve as the basis upon which the perceptual sound zone algorithm will be built in \autoref{ch:perceptual_sound_zone}.
In addition to this, it will serve as a reference with which the perceptual sound zone algorithm can be compared.

The chapter ends with a summary and concluding remarks in \autoref{ch:sound_zone:conclusion}.
