The goal of this chapter is find a suitable sound zone approach for integration with the perceptual model selected in \autoref{ch:perceptual}.

Before selecting an approach, a review of sound zone literature is given.
As such, the chapter will start with a brief description of sound zones in \autoref{ch:sound_zone:problem} to give the reader 
the necessary background information on the sound zone problem.
Afterwards, \autoref{ch:sound_zone:data_model} will present the mathematical framework  
used to describe the sound zone problem.

This mathematical framework is then used to describe the two main sound zone approaches, ``Pressure Matching'' and ``Acoustic Contrast Control'', 
in \autoref{ch:sound_zone:approaches}.
One of these approaches will be used in the perceptual sound zone algorithm.

In order to determine which one is most suited, \autoref{ch:sound_zone:approach_selection} will reflect on the mathematical properties of the 
Par detectability selected in \autoref{ch:perceptual}.
The selected approach is then adjusted in \autoref{ch:sound_zone:block_based} and \autoref{ch:sound_zone:frequency_domain} as to prepare it for 
integration with the perceptual model in \autoref{ch:perceptual_sound_zone}.

The chapter ends with a summary and concluding remarks in \autoref{ch:sound_zone:conclusion}.
