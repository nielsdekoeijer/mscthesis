This section will focus on the selection of a sound zone approach suitable for integration with the perceptual model selected in \autoref{ch:perceptual}.

In \autoref{ch:perceptual} the Par detectability was selected as the most promising perceptual model.
In order to select a suitable sound zone approach from the approaches considered in the review given in \autoref{ch:sound_zone:approach_review} the 
mathematical properties of this model will be considered.

\subsection{Mathematical Properties of Detectability}
Recall from \autoref{ch:perceptual:implementation} that the detectability $D(x[n],\varepsilon[n])$ quantifies how detectable a disturbance
$\varepsilon[n]\in\Real{N_x}$ is in presence of a masking signal $x[n]\in\Real{N_x}$.
In \autoref{ch:perceptual:implementation} it was also noted that the detectability is computed in the frequency domain.
To this end, $X[k]$ and $\mathcal{E}[k]$ denote the frequency domain representations of $x[n]$ and $\varepsilon[n]$ respectively.

In \autoref{ch:perceptual:implementation:least_squares} it was found that the detectability could be expressed as in least-squares fashion as follows:
\begin{align}
    D(x[n],\varepsilon[n]) &= \norm[2][2]{\mat{W}\vecsymbol{\mathcal{E}}} 
\end{align}
Here, the matrix $\mat{W}\in\Real{N_x \times N_x}$ is a diagonal matrix that models the masking effects of $x[n]$.
The vector $\vecsymbol{\mathcal{E}}$ contains the frequency domain representation $\mathcal{E}[k]$.
As such, $\mat{W}$ weighs each frequency component contained $\vecsymbol{\mathcal{E}}$.

Note the following about the detectability:
\begin{enumerate}
    \item It is computed in the frequency domain. 
    \item It operates on a short time scale (20 to 200 ms).
    \item It is a convex function of the disturbance $\vecsymbol{\mathcal{E}}$.
\end{enumerate}

\subsection{Selecting an Appropriate Sound Zone Approach}
\todo{Write this}
In \autoref{ch:sound_zone:approach_review} three different sound zone approaches were discussed: delay and sum beamforming (DS), pressure matching (PM),
and acoustic contrast control (ACC).
It was found that the delay and sum beamformer framework was too limiting to include any perceptual information, thus PM and ACC remain.

As stated in \autoref{ch:sound_zone:approach_review}, the PM approach minimizes the squared error between the target sound pressure and the 
sound pressure reproduced by the loudspeaker array.

