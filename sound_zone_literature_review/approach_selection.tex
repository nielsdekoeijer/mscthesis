In \autoref{ch:perceptual} it was determined that the Par detectability is the perceptual model most suited for use in a 
perceptual sound zone algorithm.
Previously, in \autoref{ch:sound_zone:problem}, three different sound zone techniques were discussed:
delay-and-sum beamforming (DS), pressure matching (PM), and acoustic contrast control (ACC). 

In this section, a reflection will be made on what sound zone approach is must suited for integration with the Par
detectability.
This will be done by first reflecting on its mathematical properties in \autoref{ch:sound_zones:approach_selection:mathematical}.
Next, \autoref{ch:sound_zones:approach_selection:selection} will argue why pressure matching is a promising approach for 
the Par detectability.

\subsection{Mathematical Properties of Par Detectability}
\label{ch:sound_zones:approach_selection:mathematical}
This section will reflect on the mathematical properties of the Par detectability.
Recall from \autoref{ch:perceptual:implementation} that the detectability $D(x[n],\varepsilon[n])$ quantifies how detectable a disturbance
$\varepsilon[n]\in\Real{N_x}$ is in presence of a masking signal $x[n]\in\Real{N_x}$.

The detectability is computed using the frequency domain representation of its input signals.
To this end, $X[k]$ and $\mathcal{E}[k]$ were introduced to denote the frequency domain representations of $x[n]$ and $\varepsilon[n]$ respectively.
In addition to this it is important to note that the Par detectability operates short-time scales (20 to 200 ms).

Finally, in \autoref{ch:perceptual:implementation:least_squares} it was found that the detectability could be expressed as in least-squares fashion as follows:
\begin{align}
    D(x[n],\varepsilon[n]) &= \norm[2][2]{W_x[k]\mathcal{E}[k]} 
\end{align}
Here, perceptual weighting $W_x[k]$ models the masking effects of the masking signal $x[n]$.
Note that the detectability is a convex function of the disturbance signal $\mathcal{E}[k]$. 

\subsection{Selecting an Appropriate Sound Zone Approach}
\label{ch:sound_zones:approach_selection:selection}
In \autoref{ch:sound_zone:approach_review} two different sound zone approaches were discussed: pressure matching (PM)
and acoustic contrast control (ACC).
