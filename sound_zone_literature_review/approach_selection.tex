In \autoref{ch:perceptual} it was determined that the Par detectability is the perceptual model most suited for use in a 
perceptual sound zone algorithm.
Previously, in \autoref{ch:sound_zone:approaches}, two different sound zone techniques were discussed 
pressure matching (PM), and acoustic contrast control (ACC). 

In this section, a selection will be made on what sound zone approach is must suited for integration with the Par
detectability.
This will be done by first reflecting on its mathematical properties in \autoref{ch:sound_zones:approach_selection:mathematical}.
Next, \autoref{ch:sound_zones:approach_selection:selection} will argue why pressure matching is a promising approach for 
the Par detectability.

\subsection{Mathematical Properties of Par Detectability}
\label{ch:sound_zones:approach_selection:mathematical}
This section will reflect on the mathematical properties of the Par detectability.
Recall from \autoref{ch:perceptual:implementation} that the detectability $D(x[n],\varepsilon[n])$ quantifies how detectable a disturbance
$\varepsilon[n]\in\Real{N_x}$ is in presence of a masking signal $x[n]\in\Real{N_x}$.

The detectability is computed using the frequency domain representation of its input signals.
To this end, $X[k]$ and $\mathcal{E}[k]$ were introduced to denote the frequency domain representations of $x[n]$ and $\varepsilon[n]$ respectively.
In addition to this it is important to note that the Par detectability operates short-time scales (20 to 200 ms).

Finally, in \autoref{ch:perceptual:implementation:least_squares} it was found that the detectability could be expressed as in least-squares fashion as follows:
\begin{align}
    D(x[n],\varepsilon[n]) &= \norm[2][2]{W_x[k]\mathcal{E}[k]} 
\end{align}
Here, perceptual weighting $W_x[k]$ models the masking effects of the masking signal $x[n]$.
The weights are applied to the frequency bins of the disturbance signal $\mathcal{E}[k]$, and the sum of squares is used determine the final detectability rating.

Note that the detectability is a convex function of the disturbance signal $\mathcal{E}[k]$. 
As such, one approach is to specify compute a sound zone algorithm which leverages the optimization over this disturbance signal in some way.
This will be done by adopting a model for the disturbance $\mathcal{E}[k]$.

In summary, the detectability has the following properties that must be taken into account:
\begin{enumerate}
    \item It is computed in the frequency domain.
    \item It operates on short-time scales.
    \item It is convex in the disturbance signal $\mathcal{E}[k]$.
\end{enumerate}

\subsection{Selecting an Appropriate Sound Zone Approach}
\label{ch:sound_zones:approach_selection:selection}
In \autoref{ch:sound_zone:approaches} two different sound zone approaches were discussed: pressure matching (PM)
and acoustic contrast control (ACC).
In this section it will be determined which approach of these two approaches is most promising for integration with a perceptual model.

Currently, neither approach operates in the frequency domain on short-time scales. 
However, it has been shown in literature that both ACC and PM can be adapted to operate in this fashion.
As such, the decision between ACC and PM will depend on which approach can more naturally use the detectability.

It was found that the PM formulation can be easily extended to use detectability directly.
In the discussion of PM in \autoref{ch:sound_zone:approaches}, PM was shown to minimize the sum of the reproduction error in the bright zone $\text{RE}_\zz$ 
and the leakage to the dark zone $\text{LE}_\zz$. 
Their definitions were given as follows:
\begin{align}
    \text{RE}_\zz &= \sum_{m\in Z} \norm[2][2]{p_\zz^{(m)}[n] - t^{(m)}[n]} \\
    \text{LE}_\zz &= \sum_{m\notin Z} \norm[2][2]{p_\zz^{(m)}[n]} 
\end{align}
Here, the reproduction error $\text{RE}_\zz$ can be understood as the total error between the achieved and target sound pressure in the bright zone of zone $\zz$. 
The leakage $\text{LE}_\zz$ can be understood as the sound pressure for zone $\zz$ leaking into the dark zones.
Both can be thus be considered as ``errors'' that we wish to minimize.
Viewing it this way, one choice for model these errors as the disturbance signals $\varepsilon[n]$ 

This can be done as follows.
Consider the perceptual reproduction error $\text{PRE}_\zz$ and the perceptual leakage error $\text{LE}_\zz$ given as:
\begin{align}
    \text{PRE}_\zz &= \sum_{m\in Z} D(t^{(m)}[n],\,p_\zz^{(m)}[n] - t^{(m)}[n]) \\
    \text{PLE}_\zz &= \sum_{m\notin Z} D(t^{(m)}[n],\,p_\zz^{(m)}[n])  
\end{align}

In the ACC approach, it was found that there was not an as obvious choice for the disturbance and masking signals.
As such, the PM approach will be used in the perceptual sound zone algorithm in \autoref{ch:perceptual_sound_zone}.
