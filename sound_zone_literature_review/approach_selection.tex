In \autoref{ch:perceptual} it was determined that the Par detectability is the perceptual model most suited for use in a 
perceptual sound zone algorithm.
Previously, in \autoref{ch:sound_zone:approaches}, two different sound zone techniques were discussed 
pressure matching (PM), and acoustic contrast control (ACC). 

In this section proposes how the Par detectability measure can be combined with the previously discussed sound zone approaches.
This is done by first reflecting on the mathematical properties of the Par detectability measure 
in \autoref{ch:sound_zones:approach_selection:mathematical}.
Next, \autoref{ch:sound_zones:approach_selection:selection} introduces the proposed perceptual sound zone approach.

\subsection{Mathematical Properties of Par Detectability}
\label{ch:sound_zones:approach_selection:mathematical}
This section will reflect on the mathematical properties of the Par detectability.

Recall from \autoref{ch:perceptual:implementation} that the detectability $D(x[n],\varepsilon[n])$ quantifies how detectable a disturbance
$\varepsilon[n]\in\Real{N_x}$ is in presence of a masking signal $x[n]\in\Real{N_x}$.
The Par detectability assumes that the time-scale of its inputs are short, in the order of 20 to 200 ms.

The detectability is computed using the frequency domain representation of its input signals.
To this end, $X[k]$ and $\mathcal{E}[k]$ were introduced to denote the frequency domain representations of $x[n]$ and $\varepsilon[n]$ respectively.

In \autoref{ch:perceptual:implementation:least_squares} it was found that the detectability could be expressed in least-squares fashion as follows:
\begin{align}
    D(x[n],\varepsilon[n]) &= \norm[2][2]{W_x[k]\mathcal{E}[k]} 
    \label{eq:perceptual:approach_selection:detectability}
\end{align}
Here, perceptual weighting $W_x[k]$ models the psycho-acoustical masking effects of the masking signal $x[n]$.
The weights are applied to the frequency bins of the disturbance signal $\mathcal{E}[k]$, and the sum of squares is used determine the final detectability rating.

It was noted in \autoref{ch:perceptual:implementation:least_squares} that the Par detectability measure
is a convex function of the disturbance signal $\mathcal{E}[k]$ when the masking signal is held constant. 
As such, one approach is to specify a sound zone algorithm which leverages the optimization over this disturbance signal in some way.
This will be done by adopting a model for the disturbance $\mathcal{E}[k]$.
For example, the disturbance could model the sound pressure error.

In summary, the detectability has the following properties that must be taken into account:
\begin{enumerate}
    \item It is computed in the short-time frequency-domain.
    \item It is convex when optimizing over the disturbance signal $\mathcal{E}[k]$.
\end{enumerate}

\subsection{Introducing Proposed Perceptual Sound Zone Approach}
\label{ch:sound_zones:approach_selection:selection}
In \autoref{ch:sound_zone:approaches} two different sound zone approaches were discussed: pressure matching (PM)
and acoustic contrast control (ACC).
Currently, neither approach operates in the frequency domain on short-time scales.
However, it has been shown in literature that formulations of both ACC and PM exist that satisfy this condition~\cite{lee2018unified}.

Either ACC or PM can be used.
Through inspection, it was found that the Par detectability measure can be used to formulate a PM approach directly
by modeling the errors minimized by PM as disturbance signals. 
As such, in this work the focus will be on the PM approach.

To see this, recall in the discussion of PM given in \autoref{ch:sound_zone:approaches} that PM was shown to minimize the sum of the 
reproduction error in the bright zone $\text{RE}^{(m)}_\zz$ and the leakage to the dark zone $\text{LE}^{(m)}_\zz$. 
The definitions of $\text{RE}^{(m)}_\zz$ and $\text{LE}^{(m)}_\zz$ are given by \autoref{eq:sound_zone:approaches:RE} 
and \autoref{eq:sound_zone:approaches:LE} respectively.
Here, the reproduction error $\text{RE}^{(m)}_\zz$ can be understood as the total error between the 
achieved and target sound pressure in the bright zone of zone $\zz$. 
The leakage $\text{LE}^{(m)}_\zz$ can be understood as the sound pressure that arises in zones other than $\zz$, essentially ``leaking'' into other zones,
when reproducing the target in the bright zone.

Both can be thus be considered as ``errors'' that we wish to minimize.
Viewing it this way, one choice for model these errors as the disturbance signals $\varepsilon[n]$.
To this end, consider the reproduction error detectability $\text{RED}^{(m)}_\zz$ and the leakage error detectability $\text{LED}^{(m)}_\zz$.
Fom now, ignore to the fact that the quantities for target and achieved sound pressure defined so far do not satisfy the short-time requirements. 
The perceptual errors are defined as: 
\begin{alignat}{2}
    \text{RED}^{(m)}_\zz &= D(t^{(m)}[n],\,p_\zz^{(m)}[n] - t^{(m)}[n]) \qquad&& \forall\,\, m\in Z \label{eq:sound_zone:approach_selection:RED}\\
    \text{LED}^{(m)}_\zz &= D(t^{(m)}[n],\,p_\zz^{(m)}[n]) \qquad&& \forall\,\, m\notin Z \label{eq:sound_zone:approach_selection:LED} 
\end{alignat}
The reproduction error detectability $\text{RED}^{(m)}_\zz$ is determined by the detectability of the sound pressure difference between the achieved sound
pressure $p_\zz^{(m)}[n]$ and the target sound pressure $t^{(m)}[n]$ for the control point $m$.
This sound pressure difference can be understood as the ``error'' in sound pressure.

The target sound pressure is used as the masking signal, in doing so the masking properties of the target sound pressure $t^{(m)}[n]$ is used.
Note that this is an approximation: in reality, it cannot be assumed that the sound algorithm exactly attains the target sound pressure.
In the ideal case, the masking properties of the total achieved sound pressure would be used instead.

However, this quantity depends on the loudspeaker input signals over which the optimization is performed, and is thus not available a priori. 
One approach would be to include the masking signal in the optimization, but this would violate convexity, as stated in 
\autoref{ch:perceptual:implementation:least_squares}.
As such, by the masking effects of the achieved pressure are modeled by those of the target sound pressure.

The $\text{RED}^{(m)}_\zz$ models how detectable the deviation from target sound pressure is
One interpretation of minimizing $\text{RED}^{(m)}_\zz$ is finding the sound pressure difference which is least detectable. 

Similarly, the leakage error detectability $\text{LED}^{(m)}_\zz$ is determined by how detectable the sound pressure of zone $\zz$ is in other zones.
Note as $m\notin\zz$, this definition considers all points outside of the bright zone for $\zz$.  
Here, the masking signal is again chosen as the target sound pressure of the control point in question.
$\text{LED}^{(m)}_\zz$ models how detectable the leakage of zone $\zz$ is in presence of the intended target sound pressure. 
Minimizing $\text{LED}^{(m)}_\zz$ could thus be understood as finding the leakage that is minimally-detectable in presence of the target sound pressure.

These perceptually modified errors are found to be a promising way of combining pressure matching and the Par detectability,
and are used to state perceptual sound zone algorithms in \autoref{ch:perceptual_sound_zone}.
The short-time frequency-domain quantities required for the definitions of $\text{RED}^{(m)}_\zz$ and $\text{LED}^{(m)}_\zz$ are introduced in this chapter.

Using the ACC approach to formulate perceptual sound zone algorithms is not explored further in this work, but is found to be promising future work.
