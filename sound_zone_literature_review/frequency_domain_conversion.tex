In the previous section, the Block-Based Multi-Zone Pressure-Matching (BB-MZ-PM) algorithm was derived.
When deriving this algorithm it was stated that it's advantages are twofold.

Firstly, one advantage of using this algorithm over its non block-based counterpart is that it can work in real-time.
Secondly, the block-based approach can operate on short time-scales.
This is useful, as the perceptual model that we wish to integrate operates on time-scales of the order of 20 to 200 milliseconds.

There is however an additional adjustment that needs to be made before the perceptual model can be integrated. 
Currently, the BB-MZ-PM algorithm minimizes L2-error in the time-domain, whereas the detectability is computed in the 
frequency domain.
In order to integrate the perceptual model and the sound zone algorithm, they must operate in the same domain.

In this case, it was chosen to use convert the existing sound zone algorithm to the frequency domain.
This was chosen as to remain close to the definition of the detectability.
Approximating detectability in the time domain was not explored further and could potentially be of interest for future work.

To this end, this section will adjust the existing time domain BB-MZ-PM algorithm to an equivalent frequency domain formulation.
This will be done by first introducing a suitable transformation relating the time and frequency domain quantities.
The transformed quantities can than be used define the frequency domain version of the BB-MZ-PM algorithm.

\subsection{Frequency Domain Transformation}
A suitable transform to obtain the frequency domain representation of a signal is the DFT.
However, it is important to take a number of precautions before applying the DFT directly,
as the computation of the sound pressures used in the optimization problem introduced previously involves taking the 
linear convolution of the loudspeaker input signals.

Time domain circular convolution can be computed in the frequency domain through the Hadamard product.
Time domain circular convolution coincides with time domain linear convolution only if the two operands are zero-padded sufficiently.
To be specific, both operands need be zero-padded to the length of the resulting convolution.

As such, the frequency domain transform requires this zero padding to be built in.
The convolutions described in the previous chapter are between the window coefficients of size $N_w$ 
and the room impulse responses of size $N_h$.
Thus, the both must be zero padded to convolution length $N_w + N_h - 1$ before going to the frequency domain.

A suitable transform is thus given by a $N_w + N_h - 1$ point DFT.
\begin{equation}
    X[k] = \sum_{n=0}^{N_w + N_h - 2} x[n]\exp\left(\frac{-j2\pi k n}{N_w + N_h - 1}\right)
\end{equation}
Here, $x[n]$ and $X[k]$ are the time- and frequency-domain representations of an arbitrary sequence.  

\subsection{Quantities the Frequency Domain}
In this section, the quantities used in the Block-Based Multi-Zone Pressure-Matching approach will be converted to their frequency domain counterparts.

Essentially, this involves converting the sound pressures $\tilde{p}_{\zz}^{(m)}[n,\mu]$ and $\tilde{t}^{(m)}[n,\mu]$ 
to their frequency domain versions given by $\tilde{P}_{\zz,\mu}^{(m)}[k]$ and $\tilde{T}_\mu^{(m)}[k]$ respectively.

This can be done by applying the previously derived transform directly to these quantities.
This results in the following expressions.
\begin{align}
    \tilde{T}^{(m)}[k, \mu] &= \tilde{T}^{(m)}[k, \mu - 1] + \sum_{l=0}^{N_L}H^{(l,m)}[k]\tilde{S}_{\zz,\mu}[k] \\
    \tilde{P}_\zz^{(m)}[k, \mu] &= \tilde{P}_\zz^{(m)}[k, \mu - 1] 
        + \sum_{l=0}^{N_L}H^{(l,m)}[k]\tilde{X}^{(l)}_{\zz,\mu}[k]  
\end{align}
Here, $H^{(l,m)}[k]\in\Complex{N_w + N_h - 1}$ is the transformed version of the room impulse responses.
Furthermore, $\tilde{S}_{\zz,\mu}[k]\in\Complex{N_w + N_h - 1}$ and 
$\tilde{X}^{(l)}_{\zz,\mu}[k]\in\Complex{N_w + N_h - 1}$ are the frequency domain versions of
the desired playback signal and the loudspeaker input signal, which are defined as follows:
\begin{align}
    \tilde{S}_{\zz,\mu}[k] &= \sum_{n=0}^{N_w + N_h - 2} \tilde{s}_{\zz,\mu}[n]w[n - \mu H]
        \exp\left(\frac{-j2\pi k n}{N_w + N_h - 1}\right) \\
    \tilde{X}^{(l)}_{\zz,\mu}[k] &= \sum_{n=0}^{N_w + N_h - 2} \tilde{x}^{(l)}_{\zz,\mu}[n]w[n - \mu H]
        \exp\left(\frac{-j2\pi k n}{N_w + N_h - 1}\right)
\end{align}

Note that the window is implicitly included in the transformed quantities.
This is done for ease of notation.

\subsection{Proposed Frequency Domain Approach}
Using the previously derived quantities, it is possible express the frequency domain version of the 
Block-Based Multi-Zone Pressure-Matching (BB-MZ-PM) approach as follows:
\begin{align}
    \argmin{\tilde{x}_{\za,\mu}^{(l)}[n],\,\tilde{x}_{\zb,\mu}^{(l)}[n]\,\forall\,l}{
       &\sum_{m\in A} \norm[2][2]{\tilde{P}_{\za}^{(m)}[k,\mu] - \tilde{T}^{(m)}[k,\mu]} +
        \sum_{m\in A} \norm[2][2]{\tilde{P}_{\zb}^{(m)}[k,\mu]} + \\
       &\sum_{m\in B} \norm[2][2]{\tilde{P}_{\zb}^{(m)}[k,\mu] - \tilde{T}^{(m)}[k,\mu]} + 
        \sum_{m\in B} \norm[2][2]{\tilde{P}_{\za}^{(m)}[k,\mu]}
    }
\end{align}
Note how the optimization is still performed over the time domain signal.
This was done to constrain the loudspeaker input signal coefficient to size $N_w$, 
as that is an assumption made by the frame-based processing.

In principal, this introduces more complexity than solving directly over the frequency domain loudspeaker input coefficient
$\tilde{X}^{(l)}_{\zz,\mu}[k]$.
This however introduces issues as it requires the truncation of the time-domain version to $N_w$ samples, which 
was found to introduce artifacts.
