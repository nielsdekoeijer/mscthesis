An initial description of the sound zone problem was given in the introduction of the thesis.
This section seeks to build on this description in order to provide the understanding necessary 
for the rest of this work.

As mentioned in the introduction, the problem that sound zones seeks to solve is the reproduction
of multiple types of audio content in the same room with minimal interference.
This way, multiple people can enjoy different audio content without disturbing one another.

Controlling the spatial distribution of sound is done by controlling the 
audio that is produced by an array of loudspeakers.
The space inside the enclosure is divided up into multiple zones.
Each zone is assigned target sound pressure that we would like to have reproduced inside of it.
This target sound pressure could be various audio content, for example music or the sound of a movie.

To understand this principal, consider the example given by \autoref{fig:sound_zones:background:content}.
The loudspeakers array that is present in the room is to be controlled by the sound zone algorithm 
in such a way that the desired content is reproduced in each zone.
As mentioned, this is to be done in a way that results in minimal interference, e.g. it is undesirable
to be able to hear content B when inside the red zone.

\begin{figure}[h]
    \centering
    \scalebox{1.0}{\input{tikz/tikz_2D_room_concept_content.tex}}
    \caption{A birds-eye view of a room is depicted.
        It is divided into two zones: a red zone and a blue zone.
        Each zone is assigned different content: content A and content B.
        In the northern and southern parts of the room, a loudspeaker array is mounted on the walls.
        }
    \label{fig:sound_zones:background:content}
\end{figure}

There are various approaches to solving the sound zone problem.
An important concept that is used frequently is that bright zones and dark zones.

Sound zone problems are typically decomposed into a separate subproblem for every zone.
Each one of these subproblems considers only two zones: one bright zone and one dark zone.
The goal of each subproblems is to reproduce a specified target sound pressure in the bright zone while restricting the 
sound pressure in the dark zones.

The combination of all subproblems provides a solution to the sound zone problem. 
To ease the understanding of this concept, consider an example of this decomposition is given in 
\autoref{fig:sound_zones:background:bright_dark_example}.

Here, a decomposition of the example given in \autoref{fig:sound_zones:background:content} into 
two bright-dark zone pairs.
For the first problem, the goal is to reproduce content A in bright zone A while minimizing 
the amount of sound pressure in dark zone A.
Similarly for the second problem: reproduce content B in bright zone B while minimizing the 
amount of sound pressure in dark zone B.
Combining the two solutions results in a solution with content reproduced in both zones with 
minimal interference between zones.

\begin{figure}[]
    \centering
    \begin{subfigure}{0.49\linewidth}
        \centering
        \scalebox{0.9}{\input{tikz/tikz_2D_room_concept_bright_dark_zone_A.tex}}
    \end{subfigure}
    \begin{subfigure}{0.49\linewidth}
        \centering
        \scalebox{0.9}{\input{tikz/tikz_2D_room_concept_bright_dark_zone_B.tex}}
    \end{subfigure}
    \caption{A birds-eye view of a room is depicted twice containing two different sound zone problems.}
    \label{fig:sound_zones:background:bright_dark_example}
\end{figure}

This concludes the introduction to the sound zone problem.
The goal of the rest of this chapter is find a suitable sound zone approach 
for integration with the perceptual model selected in \autoref{ch:perceptual}.

Before selecting an approach, a review of sound zone literature is given.
As such, the chapter will start with a brief description of sound zones in \autoref{ch:sound_zone:problem} to give the reader 
the necessary background information on the sound zone problem.
Afterwards, \autoref{ch:sound_zone:data_model} will present the mathematical framework  
used to describe the sound zone problem.

This mathematical framework is then used to describe the two main sound zone approaches, ``Pressure Matching'' and ``Acoustic Contrast Control'', 
in \autoref{ch:sound_zone:approaches}.
One of these approaches will be used in the perceptual sound zone algorithm.

In order to determine which one is most suited, \autoref{ch:sound_zone:approach_selection} will reflect on the mathematical 
properties of the 
Par detectability selected in \autoref{ch:perceptual}.
From this, one approach will be selected.

% \subsection{Sound Zone Approaches}
% \label{ch:sound_zone:problem:solutions}
% With the sound zone problem sufficiently explained, this section will now cover various sound zone approaches.
% These approaches describe a general manner of solving the sound zone problem.

% \subsubsection{Delay and Sum Beamforming}
% One traditional approach to creating sound zones is delay and sum (DS) beamforming~\cite{olik2013comparative}.
% In beamforming an array of loudspeakers is used to focus a ``beam'' of audio.
% This is done using the principals of constructive and destructive interference by playing with slight delays from each loudspeaker in the array (hence, 
% the name ``delay and sum'').
% The beam is constructed through constructive interference, whereas outside of the beam the audio content is partially canceled through destructive interference.

% For sound zone applications, the angle of the beam is chosen such that the target audio is directed towards its respective bright zone.
% Beamforming is somewhat limited for sound zone applications as it limits the spatial control to an angled beam.
% The other sound zone algorithms that will be discussed in this section can be considered a generalization of beamforming, as their framework supports more
% spatial control.

% \subsubsection{Pressure Matching Approach}
% In pressure matching approaches (PM), one attempts to control the output of the loudspeaker array in such a way that resulting sound pressure in the zone 
% matches the specified target sound pressure for that zone.
% While doing so, the PM approach attempts to minimize the sound pressure that results in other zones as to minimize the interference or crosstalk between zones
% ~\cite{olik2013comparative, betlehem2015personal}.

% PM is usually implemented as an optimization problem which minimizes the least-squares error between the target and resulting sound pressures in the bright zone.
% The optimization problem often includes constraints which limit the squared resulting sound pressure in the dark zone(s).
% The minimization of the least-squares error has been done in both the frequency and the time domain.

% \subsubsection{Acoustic Contrast Control Approach}
% The acoustic contrast control (ACC) approach to sound zones attempts to maximize the acoustic contrast between the bright zone and the dark zone for 
% each specified target sound pressure.
% The acoustic contrast is defined as the ratio of the acoustic potential energy of the bright zone and the dark zone.
% Just as with PM, ACC is defined as an optimization problem.

% Typically, ACC is performed in the frequency domain.
% This is done in a narrow band fashion, where the solution computed separately per frequency band ~\cite{olik2013comparative, betlehem2015personal}.
% However, Elliot et al. explored a broadband approach to ACC in 2011 which was further refined by Cai et al in 2014 called broad band acoustic contrast control 
% (BACC)~\cite{elliott2011regularisation, cai2014time}.
% % This was done by translating the ACC problem into the time domain rather than the frequency domain.

% ACC approaches typically have good acoustic contrast, but reproduce the target sound pressure less faithfully than PM approaches~\cite{lee2020signal}.
% As such, hybrid approaches such as ACC-PM~\cite{galvez2015time} and BACC-PM~\cite{lee2018unified} have been proposed recently which allow for what is
% essentially a compromise between ACC and PM.
