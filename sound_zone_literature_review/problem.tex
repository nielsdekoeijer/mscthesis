An initial description of the sound zone problem was given in the introduction of the thesis.
This section seeks to build on this description in order to provide the understanding necessary 
for the rest of this work.

As mentioned in the introduction, the problem that sound zones seeks to solve is the reproduction
of multiple types of audio content in the same room with minimal interference.
This way, multiple people can enjoy different audio content without disturbing one another.

Controlling the spatial distribution of sound is done by controlling the 
audio that is produced by an array of loudspeakers.
The space inside the enclosure is divided up into multiple zones.
Each zone is assigned target sound pressure that we would like to have reproduced inside of it.
This target sound pressure could be various audio content, for example music or the sound of a movie.

To understand this principal, consider the example given by \autoref{fig:sound_zones:background:content}.
The loudspeakers array that is present in the room is to be controlled by the sound zone algorithm 
in such a way that the desired content is reproduced in each zone.
As mentioned, this is to be done in a way that results in minimal interference, e.g. it is undesirable
to be able to hear content B when inside the red zone.

\begin{figure}[h]
    \centering
    \scalebox{1.0}{\begin{tikzpicture}
    \draw [draw=black] (0,0) rectangle (8,6);

    % Speakers on the top wall
    \pic[scale=0.7] at (2, 5.6) {Speaker};
    \pic[scale=0.7] at (3, 5.6) {Speaker};
    \pic[scale=0.7] at (4, 5.6) {Speaker};
    \pic[scale=0.7] at (5, 5.6) {Speaker};
    \pic[scale=0.7] at (6, 5.6) {Speaker};

    % Speakers on the bottom wall
    \pic[rotate=180, scale=0.7] at (2, 0.4) {Speaker};
    \pic[rotate=180, scale=0.7] at (3, 0.4) {Speaker};
    \pic[rotate=180, scale=0.7] at (4, 0.4) {Speaker};
    \pic[rotate=180, scale=0.7] at (5, 0.4) {Speaker};
    \pic[rotate=180, scale=0.7] at (6, 0.4) {Speaker};

    \draw[opacity=0.4, fill=blue] (6,3) circle[radius=1.5];
    \draw[thick] (6,3) circle (1.5) node[align=center] {\textbf{Content B}};
    \draw[opacity=0.4, fill=red]  (2,3) circle[radius=1.5];
    \draw[thick] (2,3) circle (1.5) node[align=center] {\textbf{Content A}};
\end{tikzpicture}
}
    \caption{A birds-eye view of a room is depicted.
        It is divided into two zones: a red zone and a blue zone.
        Each zone is assigned different content: content A and content B.
        In the northern and southern parts of the room, a loudspeaker array is mounted on the walls.
        }
    \label{fig:sound_zones:background:content}
\end{figure}

There are various approaches to solving the sound zone problem.
An important concept that is used frequently is that bright zones and dark zones.

Sound zone problems are typically decomposed into a separate subproblem for every zone.
Each one of these subproblems considers only two zones: one bright zone and one dark zone.
The goal of each subproblems is to reproduce a specified target sound pressure in the bright zone while restricting the 
sound pressure in the dark zones.

The combination of all subproblems provides a solution to the sound zone problem. 
To ease the understanding of this concept, consider an example of this decomposition is given in 
\autoref{fig:sound_zones:background:bright_dark_example}.

Here, a decomposition of the example given in \autoref{fig:sound_zones:background:content} into 
two bright-dark zone pairs.
For the first problem, the goal is to reproduce content A in bright zone A while minimizing 
the amount of sound pressure in dark zone A.
Similarly for the second problem: reproduce content B in bright zone B while minimizing the 
amount of sound pressure in dark zone B.
Combining the two solutions results in a solution with content reproduced in both zones with 
minimal interference between zones.

\begin{figure}[]
    \centering
    \begin{subfigure}{0.49\linewidth}
        \centering
        \scalebox{0.9}{\begin{tikzpicture}
    \draw [draw=black] (0,0) rectangle (8,6);

    % Speakers on the top wall
    \pic[scale=0.7] at (2, 5.6) {Speaker};
    \pic[scale=0.7] at (3, 5.6) {Speaker};
    \pic[scale=0.7] at (4, 5.6) {Speaker};
    \pic[scale=0.7] at (5, 5.6) {Speaker};
    \pic[scale=0.7] at (6, 5.6) {Speaker};

    % Speakers on the bottom wall
    \pic[rotate=180, scale=0.7] at (2, 0.4) {Speaker};
    \pic[rotate=180, scale=0.7] at (3, 0.4) {Speaker};
    \pic[rotate=180, scale=0.7] at (4, 0.4) {Speaker};
    \pic[rotate=180, scale=0.7] at (5, 0.4) {Speaker};
    \pic[rotate=180, scale=0.7] at (6, 0.4) {Speaker};

    \draw[opacity=0.0, fill=blue] (6,3) circle[radius=1.5];
    \draw[thick] (6,3) circle (1.5) node[align=center] {\textbf{Dark Zone A}};
    \draw[opacity=0.4, fill=red]  (2,3) circle[radius=1.5];
    \draw[thick] (2,3) circle (1.5) node[align=center] {\textbf{Bright Zone A}};
\end{tikzpicture}
}
    \end{subfigure}
    \begin{subfigure}{0.49\linewidth}
        \centering
        \scalebox{0.9}{\begin{tikzpicture}
    \draw [draw=black] (0,0) rectangle (8,6);

    % Speakers on the top wall
    \pic[scale=0.7] at (2, 5.6) {Speaker};
    \pic[scale=0.7] at (3, 5.6) {Speaker};
    \pic[scale=0.7] at (4, 5.6) {Speaker};
    \pic[scale=0.7] at (5, 5.6) {Speaker};
    \pic[scale=0.7] at (6, 5.6) {Speaker};

    % Speakers on the bottom wall
    \pic[rotate=180, scale=0.7] at (2, 0.4) {Speaker};
    \pic[rotate=180, scale=0.7] at (3, 0.4) {Speaker};
    \pic[rotate=180, scale=0.7] at (4, 0.4) {Speaker};
    \pic[rotate=180, scale=0.7] at (5, 0.4) {Speaker};
    \pic[rotate=180, scale=0.7] at (6, 0.4) {Speaker};

    \draw[opacity=0.4, fill=blue] (6,3) circle[radius=1.5];
    \draw[thick] (6,3) circle (1.5) node[align=center] {\textbf{Bright Zone B}};
    \draw[opacity=0.0, fill=red]  (2,3) circle[radius=1.5];
    \draw[thick] (2,3) circle (1.5) node[align=center] {\textbf{Dark Zone B}};
\end{tikzpicture}
}
    \end{subfigure}
    \caption{A birds-eye view of a room is depicted twice containing two different sound zone problems.}
    \label{fig:sound_zones:background:bright_dark_example}
\end{figure}

This concludes the introduction to the sound zone problem.

\subsection*{Chapter Structure}
The goal of the rest of this chapter is to motivate the proposal of a perceptual sound zone approach which uses
the Par distortion detectability introduced in \autoref{ch:perceptual}.
It is structured as follows:
\begin{itemize}
    \item This chapter begins in \autoref{ch:sound_zone:data_model} with the presentation of  
a mathematical framework which can be used to describe the sound zone problem.
    \item This mathematical framework is then used to describe the two main sound zone approaches, ``Pressure Matching'' and ``Acoustic Contrast Control'', 
in \autoref{ch:sound_zone:approaches}.
    \item Finally, in \autoref{ch:sound_zone:approach_selection} will motivate the proposed perceptual sound zone approach.
This is done by reflect on the mathematical properties of the Par detectability and the sound zone approaches review in \autoref{ch:sound_zone:approaches}.
\end{itemize}
