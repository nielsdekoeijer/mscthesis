In \autoref{ch:perceptual} the Par detectability is selected as the most promising perceptual model for use in a perceptual sound zone algorithm.
Next, \autoref{ch:sound_zone} various sound zone algorithms are discussed, ultimately leading to the proposal of a sound zone approach in 
\autoref{ch:sound_zone:approach_selection} that uses the Par detectability measure in order to create sound zones.

This chapter uses the proposed perceptual sound zone approach to propose and implement two perceptual sound zone algorithms.

\subsection*{Chapter Structure}
This chapter is structured as follows.
\begin{itemize}
    \item First, \autoref{ch:perceptual_sound_zone:stft} discusses the reformulation of the time-domain pressure matching approach given in 
        \autoref{ch:sound_zone:approaches} to a short-time frequency-domain pressure-matching approach.
        This is necessary for the perceptual approach discussed in \autoref{ch:sound_zone:approach_selection} to be implementable.
    \item Next, \autoref{ch:perceptual_sound_zone:perceptual_minimization} discusses using the approach proposed 
        in \autoref{ch:sound_zone:approach_selection} to formulate two perceptual sound zone algorithms.
\end{itemize}

