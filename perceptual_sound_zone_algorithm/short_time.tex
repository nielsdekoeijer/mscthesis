In \autoref{ch:sound_zone:approach_selection} a perceptual sound zone framework is proposed based on the pressure matching approach discussed in \autoref{ch:sound_zone:approaches}.
In this framework, the Par detectability measure quantifies the perceptual cost of sound pressure errors.
In doing so, \autoref{ch:sound_zone:approaches} introducing the concepts of reproduction error detectability 
$\text{RED}_\zz^{(m)}$ and the leakage error detectability $\text{LED}_\zz^{(m)}$
per control point $m$.

As noted, the original pressure matching approach from \autoref{ch:sound_zone:approaches} operates on full-length input sequences in the time domain.
The detectability, however, operates on short-time segments of 20 to 200 milliseconds in the frequency domain.
To define the reproduction error detectability and the leakage error detectability, this section proposes a short-time frequency-domain pressure matching approach.

First in \autoref{ch:perceptual_sound_zones:stft:block_based} the existing pressure matching approach is reformulated to operate on short-time segments through a ``block-based'' approach. 
Next, \autoref{ch:perceptual_sound_zones:stft:stft_pm} adapts the short-time pressure matching algorithm to operate in the frequency domain. 

\subsection{Short-Time Pressure Matching}
\label{ch:perceptual_sound_zones:stft:block_based}
In order to operate on short-time segments, all quantities introduced in the data model from \autoref{ch:sound_zone:data_model} are converted to their short-time equivalent representations.
This is done by expressing quantities using overlapping blocks containing samples of these quantities.

Here, the blocks are each of size $N_w$ and overlap $N_w - H$ samples.
The constant $H$ denotes the hop size, the number of samples between each successive block.

First, the short-time equivalent representations of the desired playback signal $s_{\zz}[n]$ and the loudspeaker input
signals $x^{(l)}_{\zz}[n]$ for zone $\zz$ and loudspeaker $l$ are discussed.

In order to formulate their short-time representations, $s_\zz[n]$ and $x^{(l)}_{\zz}[n]$ are split up into multiple overlapping blocks by using shifted windows $w[n - kH]$. 

The window $w[n]\in\Real{H}$ is a non-causal window with support $-N_w + 1 \leq n \leq 0$. 
Here, $w[n]$ is chosen such that it complies with the Constant Overlap Add (COLA) condition for a given hop size $H$.
The COLA condition requires that the sum of all $H$-shifted windows add to unity for all samples $n$. 
It is given as follows:
\begin{equation}
     \sum_{k=-\infty}^{\infty} w[n - kH] = 1 \quad\forall\,n.
\end{equation}
Using the windows as defined above, consider the following representation of $s_\zz[n]$,
\begin{equation}
    \begin{aligned}
        s_\zz[n] &= s_\zz[n]\sum_{k=-\infty}^{\infty} w[n - kH] \\
                 &= \sum_{k=-\infty}^{\infty} \tilde{s}_{\zz,k}[n]w[n - kH],
    \end{aligned}
    \label{eq:perceptual_sound_zones:stft:block_based:desired_playback}
\end{equation}
and of $x^{(l)}_{\zz}[n]$,
\begin{equation}
    \begin{aligned}
        x^{(l)}_\zz[n] &= x^{(l)}_\zz[n]\sum_{k=-\infty}^{\infty} w[n - kH] \\
                       &= \sum_{k=-\infty}^{\infty} \tilde{x}^{(l)}_{\zz,k}[n]w[n - kH].
    \end{aligned}
    \label{eq:perceptual_sound_zones:stft:block_based:loudspeaker_inputs}
\end{equation}
Where $\tilde{s}_{\zz,k}[n]$ and $\tilde{x}^{(l)}_{\zz,k}[n]$ represent the content of the $k^\text{th}$ blocks of the playback signal $s_{\zz}[n]$ and loudspeaker input signals $x^{(l)}_{\zz}[n]$.

As such, $\tilde{s}_{\zz,k}[n] = s_{\zz}[n]$ and $\tilde{x}^{(l)}_{\zz,k}[n] = s_{\zz}[n]$ for $-N_w + 1 + kH \leq n \leq kH$ and zero for all other samples $n$.  
One interpretation is that the windows decimate the signal into segments of size $N_w$, which can be reconstructed perfectly through addition due to the COLA condition.

One way of interpreting the equations above is as a projection of $s_\zz[n]$ and $x^{(l)}_{\zz}[n]$ on a basis of frames spanned by shifted overlapping windows $w[n - kH]$.
Here, $\tilde{s}_{\zz,k}[n]$ and $\tilde{x}^{(l)}_{\zz,k}[n]$ can be thought of as the coefficients for the basis functions resulting from the projection.

Let $\tilde{s}_\zz[n, \mu]$ and $\tilde{x}^{(l)}_\zz[n, \mu]$ represent the desired playback signal and the loudspeaker input signals with contributions up to and including the $\mu^\mathrm{th}$block. 
This can be expressed as follows:
\begin{align}
    \tilde{s}_\zz[n, \mu] &= \sum_{k=-\infty}^{\mu} \tilde{s}_{\zz,k}[n]w[n - kH], \\
    \tilde{x}^{(l)}_\zz[n, \mu] &= \sum_{k=-\infty}^{\mu} \tilde{x}^{(l)}_{\zz,k}[n]w[n - kH].
\end{align}
This form will converge to the actual desired playback signal as $\mu\to\infty$.
As such, $\tilde{s}_\zz[n, \infty] = s_\zz[n]$ and $\tilde{x}^{(l)}_\zz[n, \infty] = x^{(l)}_\zz[n]$.

This representation is beneficial, as it can be used to show that the $\tilde{x}^{(l)}_\zz[n, \mu]$ can be computed recursively:
\begin{align}
    \begin{aligned}
    \tilde{x}^{(l)}_\zz[n, \mu] &= \tilde{x}^{(l)}_{\zz,\mu}[n]w[n - \mu H] +
                               \sum_{k=-\infty}^{\mu - 1} \tilde{x}^{(l)}_{\zz,k}[n]w[n - kH] \\
                           &=  \tilde{x}^{(l)}_{\zz,\mu}[n]w[n - \mu H] + \tilde{x}^{(l)}_\zz[n, \mu - 1].
    \end{aligned}
\end{align}
As the newest block depends on the previous blocks, this representation shows that $x^{(l)}_\zz[n]$ can be computed block-by-block.

With the block-based equivalents of the desired playback signal $\tilde{s}_\zz[n, \mu]$ and the loudspeaker input signals $\tilde{x}^{(l)}_\zz[n, \mu]$ defined, the block-based equivalents of the target and achieved sound pressure $\tilde{t}_\zz[n, \mu]$ and $\tilde{p}^{(m)}_\zz[n, \mu]$ can be computed: 

\begin{itemize}
    \item 
        The block-based target sound pressure $\tilde{t}^{(m)}[n, \mu]$ can be defined by simply 
        substituting the definition for the block-based desired playback signal $\tilde{s}_\zz[n, \mu]$ into the definition of the 
        target pressure given by \autoref{eq:sound_zone:data_model:target_pressure}:
        \begin{equation}
            \begin{aligned}
                \tilde{t}^{(m)}[n, \mu] &= \sum_{l=0}^{N_L-1} \left(h^{(l,m)} \ast \tilde{s}_\zz[\mu]\right)[n]\\
                                        &= \sum_{l=0}^{N_L - 1}\sum_{k=-\infty}^{\mu}\left(h^{(l,m)}\ast\tilde{s}_{\zz,k}w_k\right)[n] \\
                                   &= \sum_{l=0}^{N_L-1}\left(h^{(l,m)}\ast\tilde{s}_{\zz,\mu}w_\mu\right)[n] + \tilde{t}^{(m)}[n, \mu - 1].  
            \end{aligned}
        \end{equation}
        Here, $w_k[n]$ is defined to be equal to $w[n - kH]$ and is introduced for notational convenience.  
        The definition above holds for all points $m\in Z$, i.e., the points contained in zone $\zz$.  

        As can be seen, the block-based target sound pressure for the block $\mu$ can be computed recursively by adding the contribution of the newest block of $\tilde{s}_{\zz,\mu}[n]$ to the target sound pressure of the previous block.
    \item 
        The block-based resulting sound pressure $\tilde{p}_\zz^{(m)}[n, \mu]$ can be defined by simply substituting the definition for the block-based loudspeaker input signals $\tilde{x}_\zz^{(l)}[n, \mu]$ into the definition of the resulting pressure given by \autoref{eq:sound_zone:data_model:achieved_pressure}.
        This results in the following: 
        \begin{equation}
            \begin{aligned}
                \tilde{p}_\zz^{(m)}[n, \mu] &= \sum_{l=0}^{N_L-1}                       \left(h^{(l,m)} \ast \tilde{x}^{(l)}_{\zz}[\mu]\right)[n]\\
                                            &= \sum_{l=0}^{N_L-1}\sum_{k=-\infty}^{\mu} \left(h^{(l,m)} \ast \tilde{x}^{(l)}_{\zz,k}w_k\right)[n] \\
                                            &= \sum_{l=0}^{N_L-1}                       \left(h^{(l,m)} \ast \tilde{x}^{(l)}_{\zz,\mu}w_\mu\right)[n] +
                                                \tilde{p}_\zz^{(m)}[n, \mu - 1].
            \end{aligned}
            \label{eq:perceptual_sound_zones:stft:block_based:achieved_pressure}
        \end{equation}
        The definition above again holds for all points $m\in Z$.  

        As can be seen, the block-based resulting sound pressure for the block $\mu$ can also be computed recursively.
\end{itemize}

With this, all quantities required for the block-based formulation of the pressure matching approach are defined.

It is shown in \autoref{eq:perceptual_sound_zones:stft:block_based:achieved_pressure} and \autoref{eq:sound_zone:data_model:target_pressure} that all quantities can be computed recursively.

This is used in the block-based pressure matching approach by computing the blocks of the loudspeaker input signal$x_\zz^{(l)}[n]$ one by one. 
As such, the $k^\text{th}$ loudspeaker input signal coefficient $\tilde{x}^{(l)}_{\zz,\mu}[n]$ is computed such that the resulting resulting sound pressure $\tilde{p}_\zz^{(m)}[n, \mu]$ best matches the target sound pressure $\tilde{t}^{(m)}[n, \mu]$. 

Note that in this approach, only the newest loudspeaker coefficients $\tilde{x}^{(l)}_{\zz,\mu}$ are being controlled. 
The previous coefficients $\tilde{x}^{(l)}_{\zz,k}[n]$ for $-\infty \leq k \leq \mu - 1$ are held fixed.

The block-based optimization problem can be found by replacing all quantities in the previously derived optimization problem with their block-based counterparts.
The problem is given as follows:
\begin{align}
    \begin{aligned}
        \argmin{\tilde{x}_{\za,\mu}^{(l)}[n],\,\tilde{x}_{\zb,\mu}^{(l)}[n]\,\forall\,l}{
           &\sum_{m\in A} \norm[2][2]{\tilde{p}_{\za}^{(m)}[n,\mu] - \tilde{t}_\mu^{(m)}[n,\mu]} +
           \sum_{m\in A} \norm[2][2]{\tilde{p}_{\zb}^{(m)}[n,\mu]} + \\
           &\sum_{m\in B} \norm[2][2]{\tilde{p}_{\zb}^{(m)}[n,\mu] - \tilde{t}_\mu^{(m)}[n,\mu]} + 
           \sum_{m\in B} \norm[2][2]{\tilde{p}_{\za}^{(m)}[n,\mu]}
        }.
    \end{aligned}
\end{align}
Note that this problem implicitly contains the target sound pressure and resulting sound pressure of the previous blocks $-\infty \leq k \leq \mu - 1$ due to the aforementioned recursive definitions.
As a result, the history of what has been transmitted by the loudspeaker previously is included in the optimization.

This is beneficial, as due to overlap, this allows block $\mu$ to potentially improve the results of previous blocks.
However, the loudspeaker input signals of the current block $\mu$ can only affect so many previous blocks (depending on the overlap).
As such, to reduce the complexity of the optimization without affecting the results, one may choose to truncate the number of previous blocks $-\infty \leq k \leq \mu - 1$ in the history of the optimization.

The problem above is solved recursively for all loudspeaker input signal coefficients $\tilde{x}_{\za,\mu}^{(l)}[n]$ and $\tilde{x}_{\zb,\mu}^{(l)}[n]$.
The final loudspeaker input signals $x_\zz^{(l)}[n]$  can then be found by means of \autoref{eq:perceptual_sound_zones:stft:block_based:loudspeaker_inputs}.

\subsection{Short-Time Frequency-Domain Pressure Matching}
\label{ch:perceptual_sound_zones:stft:stft_pm}
This section will adjust the block-based data model equivalent frequency-domain formulation to propose a short-time frequency-domain pressure matching algorithm.
This is done by first introducing a transformation relating the time and frequency domain quantities.

A suitable transform is the discrete Fourier transform (DFT).
However, it is important to take some precautions before applying the DFT directly.
As shown in \autoref{eq:sound_zone:data_model:achieved_pressure}, the computation of the sound pressures used in the optimization problem introduced previously involves taking the linear convolution of the loudspeaker input signals with the room impulse responses.

Time-domain circular convolution can be computed in the frequency domain through the Hadamard product.
Time-domain circular convolution coincides with time-domain linear convolution only if the two operands are zero-padded sufficiently.
To be specific, both operands need to be zero-padded to the length of the resulting linear convolution.

As such, the frequency domain transform requires this zero padding to be built-in.
The convolutions described in the previous chapter are between the window coefficients of size $N_w$ and the room impulse responses of size $N_h$.
Thus, both must be zero-padded to convolution length $N_w + N_h - 1$ before going to the frequency domain.

Let $x[n]\in\Real{N_w}$ and $X[k]\in\Complex{N_w + N_h - 1}$ denote the time- and frequency-domain representations of an arbitrary sequence.  
A suitable transform is given by the following $N_w + N_h - 1$ point DFT:
\begin{equation}
    X[k] = \sum_{n=0}^{N_w - 1} x[n]\exp\left(\frac{-j2\pi k n}{N_w + N_h - 1}\right).
\end{equation}

Converting the previously introduced block-based pressure matching to a frequency domain equivalent version
essentially involves converting the sound pressures $\tilde{p}_{\zz}^{(m)}[n,\mu]$ and $\tilde{t}^{(m)}[n,\mu]$ 
to their frequency domain counterparts, which are denoted by $\tilde{P}_{\zz,\mu}^{(m)}[k]\in\Complex{N_w + N_h - 1}$ and $\tilde{T}_\mu^{(m)}[k]\in\Complex{N_w + N_h - 1}$ respectively.

This is done as follows:
\begin{align}
    \tilde{T}^{(m)}[k, \mu] &= 
    \sum_{l=0}^{N_L}H^{(l,m)}[k]\tilde{S}_{\zz,\mu}[k],\label{eq:perceptual_sound_zones:stft:t_freq} \\
    \tilde{P}_\zz^{(m)}[k, \mu] &= 
    \sum_{l=0}^{N_L}H^{(l,m)}[k]\tilde{X}^{(l)}_{\zz,\mu}[k]. \label{eq:perceptual_sound_zones:stft:p_freq} 
\end{align}
Here, $H^{(l,m)}[k]\in\Complex{N_w + N_h - 1}$ is the transformed version of the room impulse responses.
Furthermore, $\tilde{S}_{\zz,\mu}[k]\in\Complex{N_w + N_h - 1}$ and 
$\tilde{X}^{(l)}_{\zz,\mu}[k]\in\Complex{N_w + N_h - 1}$ are the frequency domain versions of
the desired playback signal and the loudspeaker input signals, which are defined as follows:
\begin{align}
    \tilde{S}_{\zz,\mu}[k] &= 
    \sum_{n=\mu H - N_w + 1}^{\mu H}\left[\sum_{k=-\infty}^{\mu}\tilde{s}_{\zz,\mu}[n]w[n - kH]\right]
    \exp\left({\textstyle\frac{-j2\pi k \left(n - \mu H + N_w - 1\right)}{N_w + N_h - 1}}\right), \\
    \tilde{X}^{(l)}_{\zz,\mu}[k] &= 
    \sum_{n=\mu H - N_w + 1}^{\mu H}\left[\sum_{k=-\infty}^{\mu}\tilde{x}^{(l)}_{\zz,\mu}[n]w[n - kH]\right]
    \exp\left({\textstyle\frac{-j2\pi k \left(n - \mu H + N_w - 1\right)}{N_w + N_h - 1}}\right).
\end{align}
This definition takes the short-time Fourier transformation of block $\mu$ of all contributions to the desired playback signal and the loudspeaker input signals up to and including block $\mu$. 
As such, the history formed by the previous blocks is also taken into account.
Note that the window is implicitly included in the transformed quantities.
This is done for ease of notation.

Using the previously derived quantities, it is possible express the frequency domain version of the short-time pressure matching approach as follows:
\begin{align}
    \begin{aligned}
    \argmin{\tilde{x}_{\za,\mu}^{(l)}[n],\,\tilde{x}_{\zb,\mu}^{(l)}[n]\,\forall\,l}{
       &\sum_{m\in A} \norm[2][2]{\tilde{P}_{\za}^{(m)}[k,\mu] - \tilde{T}^{(m)}[k,\mu]} +
        \sum_{m\in A} \norm[2][2]{\tilde{P}_{\zb}^{(m)}[k,\mu]} + \\
       &\sum_{m\in B} \norm[2][2]{\tilde{P}_{\zb}^{(m)}[k,\mu] - \tilde{T}^{(m)}[k,\mu]} + 
        \sum_{m\in B} \norm[2][2]{\tilde{P}_{\za}^{(m)}[k,\mu]}
    }
    \end{aligned}
    \label{eq:perceptual_sound_zone:stft:stft_pm}
\end{align}
Note also how the optimization is still performed over the time domain loudspeaker input signals 
$\tilde{x}_{\za,\mu}^{(l)}[n]$ and $\tilde{x}_{\zb,\mu}^{(l)}[n]$.
This was done to constrain the loudspeaker input signal coefficient to size $N_w$, as that is an assumption made by the frame-based processing.

In principle, this introduces more complexity than solving directly over the frequency domain loudspeaker input coefficient
$\tilde{X}^{(l)}_{\zz,\mu}[k]$.
This, however, introduces issues as it requires the truncation of the time-domain version to the first $N_w$ samples.

Naively truncating $N_w$ this way introduced artifacts. 
In experiments in which this approach is attempted, the time-domain representation of the resulting frequency-domain loudspeaker input signals results in significant energy contained in the last $N_h - 1$ samples.
If truncated, a significant portion of the signal energy would be disregarded, which serves as a possible explanation for the artifacts.

However, due to the computational benefits, formulating a frequency domain approach that minimizes the impact of or prevents these artifacts is found to be promising future work.
