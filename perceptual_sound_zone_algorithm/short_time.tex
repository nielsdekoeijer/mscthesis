In \autoref{ch:sound_zone:approach_selection} a perceptual sound zone approach is proposed.
This approach is based on the pressure matching sound zone approach discussed in \autoref{ch:sound_zone:approaches}.
In this approach, the Par detectability measure is used to quantify the perceptual cost of errors defined by the pressure matching approach.

In doing so, the reproduction error detectability $\text{RED}_\zz^{(m)}$ and the leakage error detectability $\text{LED}_\zz^{(m)}$
per control point $m$ were introduced.

The original pressure matching approach from \autoref{ch:sound_zone:approaches} operates on full-length input sequences in the time-domain.
The detectability however operates on short-time segments of 20 to 200 milliseconds in the frequency-domain.
As such, in order to define the reproduction error detectability and the leakage error detectability, this section proposes a short-time
frequency-domain pressure matching approach.

First, the existing pressure matching approach is reformulated to operate on short-time segments through a ``block-based'' approach. 
Here, where smaller ``blocks'' of samples are processed block-by-block rather than processing the full-length input sequence in its entirety.

\subsection{Block-Based Data Model}
In order to operate on short-time segments, all quantities introduced in the data model from \autoref{ch:sound_zone:data_model}
are converted to their short-time equivalent representations.
This is done by considering overlapping blocks of samples of the quantities.
Here, the blocks are each of size $N_w$ and $N_w - H$ samples.
The constant $H$ denotes the hop size, which is the number of samples between each successive block.

First, the short-time equivalent representations of the desired playback signal $s_{\zz}[n]$ and the loudspeaker input
signals $x^{(l)}_{\zz}[n]$ for zone $\zz$ and loudspeaker $l$ are given.  

In order to obtain a short-time representation, $s_\zz[n]$ and $x^{(l)}_{\zz}[n]$ will be split up into multiple overlapping blocks by using shifted windows $w[n - kH]$. 

The window $w[n]\in\Real{H}$ is a non-causal window with support $-N_w + 1 \leq n \leq 0$. 
Here, $w[n]$ is chosen such that it is COLA-condition compliant for hop size $H$.
The COLA condition requires that the the sum of all shifted windows add to unity for all samples $n$ for a given hop size $H$. 
It is given as follows:
\begin{equation}
     \sum_{k=-\infty}^{\infty} w[n - kH] = 1 \quad\forall\,n
\end{equation}
Using the windows as defined above, consider the following representation of $s_\zz[n]$,
\begin{equation}
    \begin{aligned}
        s_\zz[n] &= s_\zz[n]\sum_{k=-\infty}^{\infty} w[n - kH] \\
                 &= \sum_{k=-\infty}^{\infty} \tilde{s}_{\zz,k}[n]w[n - kH]
    \end{aligned}
\end{equation}
and of $x^{(l)}_{\zz}[n]$,
\begin{equation}
    \begin{aligned}
        x^{(l)}_\zz[n] &= x^{(l)}_\zz[n]\sum_{k=-\infty}^{\infty} w[n - kH] \\
                       &= \sum_{k=-\infty}^{\infty} \tilde{x}^{(l)}_{\zz,k}[n]w[n - kH]
    \end{aligned}
\end{equation}
Here, $\tilde{s}_{\zz,k}[n]$ and $\tilde{x}^{(l)}_{\zz,k}[n]$ represent the content of the $k^\text{th}$ blocks of the playback signal $s_{\zz}[n]$ and loudspeaker input signals $x^{(l)}_{\zz}[n]$.

As such, $\tilde{s}_{\zz,k}[n] = s_{\zz}[n]$ and $\tilde{x}^{(l)}_{\zz,k}[n] = s_{\zz}[n]$ for $-N_w + 1 + kH \leq n \leq kH$ and zero for all other samples $n$.  
The windows decimate the signal, into segments of size $N_w$, which, due to the COLA condition, can be reconstructed perfectly.

One way of interpreting the equations above is as a projection of $s_\zz[n]$ and $x^{(l)}_{\zz}[n]$ on a 
basis spanned by shifted overlapping windows $w[n - kH]$.
Here, $\tilde{s}_{\zz,k}[n]$ and $\tilde{x}^{(l)}_{\zz,k}[n]$ can be thought of as the coefficients for the basis functions resulting from the projection.

Let $\tilde{s}_\zz[n, \mu]$ and $\tilde{x}^{(l)}_\zz[n, \mu]$ represent the desired playback signal and the loudspeaker input signals with the contributions of the first $\mu$ blocks. 
This can be expressed as follows:
\begin{align}
    \tilde{s}_\zz[n, \mu] &= \sum_{k=-\infty}^{\mu} \tilde{s}_{\zz,k}[n]w[n - kH] \\
    \tilde{x}^{(l)}_\zz[n, \mu] &= \sum_{k=-\infty}^{\mu} \tilde{x}^{(l)}_{\zz,k}[n]w[n - kH]
\end{align}
This form will converge to the real desired playback signal as $\mu\to\infty$.
As such, $\tilde{s}_\zz[n, \infty] = s_\zz[n]$ and $\tilde{x}^{(l)}_\zz[n, \infty] = x^{(l)}_\zz[n]$.

This representation is beneficial, as it can be used to show that the $\tilde{x}^{(l)}_\zz[n, \mu]$ can be computed recursively:
\begin{align}
    \tilde{x}^{(l)}_\zz[n, \mu] &= \tilde{x}^{(l)}_{\zz,\mu}[n]w[n - \mu H] +
                               \sum_{k=-\infty}^{\mu - 1} \tilde{x}^{(l)}_{\zz,k}[n]w[n - kH] \\
                           &=  \tilde{x}^{(l)}_{\zz,\mu}[n]w[n - \mu H] + \tilde{x}^{(l)}_\zz[n, \mu - 1]
\end{align}
As the newest block depends on the previous blocks, this represent shows that $x^{(l)}_\zz[n]$ can be computed block-by-block.

With the block-based equivalents of the desired playback signal $\tilde{s}_\zz[n, \mu]$ and the loudspeaker input signals $\tilde{s}_\zz[n, \mu]$ defined,
the block-based equivalents of the target and achieved sound pressure $\tilde{t}_\zz[n, \mu]$ and $\tilde{p}^{(m)}_\zz[n, \mu]$ can be computed: 

\begin{itemize}
    \item 
        The block-based target sound pressure $\tilde{t}^{(m)}[n, \mu]$ can be defined by simply 
        substituting the definition for the block-based desired playback signal $\tilde{s}_\zz[n, \mu]$ into the definition of the target pressure \autoref{}:
        \begin{equation}
            \begin{aligned}
                \tilde{t}^{(m)}[n, \mu] &= \sum_{l=0}^{N_L-1} \left(h^{(l,m)} \ast \tilde{s}_\zz[\mu]\right)[n]\\
                                        &= \sum_{l=0}^{N_L - 1}\sum_{k=-\infty}^{\mu}\left(h^{(l,m)}\ast\tilde{s}_{\zz,k}w_k\right)[n] \\
                                   &= \sum_{l=0}^{N_L-1}\left(h^{(l,m)}\ast\tilde{s}_{\zz,\mu}w_\mu\right)[n] + \tilde{t}^{(m)}[n, \mu - 1]  
            \end{aligned}
        \end{equation}
        Here, $w_k[n]$ is defined to be equal to $w[n - kH]$ and is introduced for notational convenience.  
        The definition above holds for all points $m\in Z$, i.e. the points contained in zone $\zz$.  

        As can be seen, the block based target sound pressure for the block $\mu$ can be computed recursively by adding the contribution of the newest block 
        $\tilde{s}_{\zz,\mu}[n]$ the target sound pressure of the previous block.
    \item 
        The block-based resulting sound pressure $\tilde{p}_\zz^{(m)}[n, \mu]$ can be defined by simply 
        substituting the definition for the block-based loudspeaker input signals $\tilde{x}_\zz^{(l)}[n, \mu]$ into the definition of the resulting pressure \autoref{}.
        This results in the following: 
        \begin{equation}
            \begin{aligned}
                \tilde{p}_\zz^{(m)}[n, \mu] &= \sum_{l=0}^{N_L-1}                       \left(h^{(l,m)} \ast \tilde{x}^{(l)}_{\zz}[\mu]\right)[n]\\
                                            &= \sum_{l=0}^{N_L-1}\sum_{k=-\infty}^{\mu} \left(h^{(l,m)} \ast \tilde{x}^{(l)}_{\zz,k}w_k\right)[n] \\
                                            &= \sum_{l=0}^{N_L-1}                       \left(h^{(l,m)} \ast \tilde{x}^{(l)}_{\zz,\mu}w_\mu\right)[n] +
                                                \tilde{p}_\zz^{(m)}[n, \mu - 1]  
            \end{aligned}
        \end{equation}
        The definition above again holds for all points $m\in Z$.  

        As can be seen, the block based resulting sound pressure for the block $\mu$ can also be computed recursively.
\end{itemize}

With this, all quantities required for the block-based formulation of the pressure matching approach are defined.

\subsection{Block-Based Pressure-Matching}
In the previous section it is shown that all quantities can be computed recursively.
This is used in the block-based pressure matching approach by computing the blocks of the loudspeaker input signal
$x_\zz^{(l)}$ one by one. 

As such, the $k^\text{th}$ loudspeaker input signal coefficient $\tilde{x}^{(l)}_{\zz,\mu}[n]$ is computed such 
that the resulting resulting sound pressure $\tilde{p}_\zz^{(m)}[n, \mu]$ best matches the target sound pressure 
$\tilde{t}^{(m)}[n, \mu]$. 

Note that in this approach only newest loudspeaker coefficients $\tilde{x}^{(l)}_{\zz,\mu}$ are being controlled. 
Thus, the previous coefficients $-\infty \leq k \leq \mu - 1$ are held fixed.

The block-based optimization problem can be found by simply replacing all quantities in the previously derived optimization problem
with their block-based counterparts.
The problem is given as follows:
\begin{align}
    \argmin{\tilde{x}_{\za,\mu}^{(l)}[n],\,\tilde{x}_{\zb,\mu}^{(l)}[n]\,\forall\,l}{
       &\sum_{m\in A} \norm[2][2]{\tilde{p}_{\za}^{(m)}[n,\mu] - \tilde{t}_\mu^{(m)}[n,\mu]} +
       \sum_{m\in A} \norm[2][2]{\tilde{p}_{\zb}^{(m)}[n,\mu]} + \\
       &\sum_{m\in B} \norm[2][2]{\tilde{p}_{\zb}^{(m)}[n,\mu] - \tilde{t}_\mu^{(m)}[n,\mu]} + 
       \sum_{m\in B} \norm[2][2]{\tilde{p}_{\za}^{(m)}[n,\mu]}
    }
\end{align}
Note that this problem implicitly contains the target sound pressure and resulting sound pressure of the previous blocks $-\infty \leq k \leq \mu - 1$ due to
the aforementioned recursive definitions.
As a result, the history of what has been transmitted by the loudspeaker previously is included in the optimization.

The problem above is solved recursively for all loudspeaker input signal coefficients $\tilde{x}_{\za,\mu}^{(l)}[n]$ and $\tilde{x}_{\zb,\mu}^{(l)}[n]$.
The final loudspeaker input signals $x_\zz^{(l)}[n]$  can then be found by means of \autoref{}.

\subsection{Block-Based Frequency-Domain Pressure-Matching}
This section will adjust the block-based data model equivalent frequency domain formulation in order to propose a block-based frequency-domain
pressure-matching algorithm.
This is done by first introducing a transformation relating the time and frequency domain quantities.

A suitable transform is the discrete Fourier transform (DFT).
However, it is important to take a number of precautions before applying the DFT directly.
As shown in \autoref{} the computation of the sound pressures used in the optimization problem introduced previously involves taking the 
linear convolution of the loudspeaker input signals with the room impulse responses.

Time domain circular convolution can be computed in the frequency domain through the Hadamard product.
Time domain circular convolution coincides with time domain linear convolution only if the two operands are zero-padded sufficiently.
To be specific, both operands need be zero-padded to the length of the resulting linear convolution.

As such, the frequency domain transform requires this zero padding to be built in.
The convolutions described in the previous chapter are between the window coefficients of size $N_w$ 
and the room impulse responses of size $N_h$.
Thus, the both must be zero padded to convolution length $N_w + N_h - 1$ before going to the frequency domain.

Let $x[n]$ and $X[k]$ denote the time- and frequency-domain representations of an arbitrary sequence.  
A suitable transform is given by the following $N_w + N_h - 1$ point DFT:
\begin{equation}
    X[k] = \sum_{n=0}^{N_w + N_h - 2} x[n]\exp\left(\frac{-j2\pi k n}{N_w + N_h - 1}\right)
\end{equation}

Converting the previously introduced block-based pressure matching to a frequency domain equivalent version
essentially involves converting the sound pressures $\tilde{p}_{\zz}^{(m)}[n,\mu]$ and $\tilde{t}^{(m)}[n,\mu]$ 
to their frequency domain counterparts, which are denoted by $\tilde{P}_{\zz,\mu}^{(m)}[k]$ and $\tilde{T}_\mu^{(m)}[k]$ respectively.

This results in the following expressions.
\begin{align}
    \tilde{T}^{(m)}[k, \mu] &= \tilde{T}^{(m)}[k, \mu - 1] + \sum_{l=0}^{N_L}H^{(l,m)}[k]\tilde{S}_{\zz,\mu}[k] \\
    \tilde{P}_\zz^{(m)}[k, \mu] &= \tilde{P}_\zz^{(m)}[k, \mu - 1] 
        + \sum_{l=0}^{N_L}H^{(l,m)}[k]\tilde{X}^{(l)}_{\zz,\mu}[k]  
\end{align}
Here, $H^{(l,m)}[k]\in\Complex{N_w + N_h - 1}$ is the transformed version of the room impulse responses.

Furthermore, $\tilde{S}_{\zz,\mu}[k]\in\Complex{N_w + N_h - 1}$ and 
$\tilde{X}^{(l)}_{\zz,\mu}[k]\in\Complex{N_w + N_h - 1}$ are the frequency domain versions of
the desired playback signal and the loudspeaker input signal, which are defined as follows:
\begin{align}
    \tilde{S}_{\zz,\mu}[k] &= \sum_{n=0}^{N_w + N_h - 2} \tilde{s}_{\zz,\mu}[n]w[n - \mu H]
        \exp\left(\frac{-j2\pi k n}{N_w + N_h - 1}\right) \\
    \tilde{X}^{(l)}_{\zz,\mu}[k] &= \sum_{n=0}^{N_w + N_h - 2} \tilde{x}^{(l)}_{\zz,\mu}[n]w[n - \mu H]
        \exp\left(\frac{-j2\pi k n}{N_w + N_h - 1}\right)
\end{align}
Note that the window is implicitly included in the transformed quantities.
This is done for ease of notation.

Using the previously derived quantities, it is possible express the frequency domain version of the block-based pressure matching approach
as follows:
\begin{align}
    \argmin{\tilde{x}_{\za,\mu}^{(l)}[n],\,\tilde{x}_{\zb,\mu}^{(l)}[n]\,\forall\,l}{
       &\sum_{m\in A} \norm[2][2]{\tilde{P}_{\za}^{(m)}[k,\mu] - \tilde{T}^{(m)}[k,\mu]} +
        \sum_{m\in A} \norm[2][2]{\tilde{P}_{\zb}^{(m)}[k,\mu]} + \\
       &\sum_{m\in B} \norm[2][2]{\tilde{P}_{\zb}^{(m)}[k,\mu] - \tilde{T}^{(m)}[k,\mu]} + 
        \sum_{m\in B} \norm[2][2]{\tilde{P}_{\za}^{(m)}[k,\mu]}
    }
\end{align}
Note how the optimization is still performed over the time domain signal.
This was done to constrain the loudspeaker input signal coefficient to size $N_w$, 
as that is an assumption made by the frame-based processing.

In principal, this introduces more complexity than solving directly over the frequency domain loudspeaker input coefficient
$\tilde{X}^{(l)}_{\zz,\mu}[k]$.
This however introduces issues as it requires the truncation of the time-domain version to $N_w$ samples, which 
was found to introduce artifacts.
