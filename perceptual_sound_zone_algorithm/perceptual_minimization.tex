Previously, \autoref{ch:sound_zone:approach_selection} noted that pressure matching approach can be formulated using the detectability.
In this section, rather than optimizing the sum reproduction errors $\text{RE}^{(m)}_\zz$ and leakage errors $\text{LE}^{(m)}_\zz$, it is proposed to instead use
the reproduction error detectability $\text{RED}^{(m)}_\zz$ and the leakage error detectability $\text{LED}^{(m)}_\zz$ respectively.
These can be understood to be perceptual alternatives to $\text{RE}^{(m)}_\zz$ and $\text{LE}^{(m)}_\zz$.

In \autoref{ch:sound_zone:approach_selection} the definition of these quantities is given using full-length input sequences.
As noted, this is inaccurate as the detectability operate on short-time scale, and is only done this way in order to clearly convey the concept. 

Henceforth, using the concepts introduced in \autoref{ch:perceptual_sound_zone:stft},
let $\text{RED}^{(m)}_\zz[\mu]$ and $\text{LED}^{(m)}_\zz[\mu]$ denote the reproduction error detectability and  
leakage error detectability for block $\mu$ in control point $m$. 
These be understood as the detectability of the error introduced in due to the current block~$\mu$.  

In order to define these error detectability quantities, recall that detectability is defined as follows: 
\begin{equation}
    D(x[n],\,\varepsilon[n]) = \norm[2][2]{W_x[k]\mathcal{E}[k]} 
\end{equation}
Using this definition alongside the short-time frequency domain definitions given in \autoref{ch:perceptual_sound_zone:stft}, 
the definition of the reproduction error detectability $\text{RED}^{(m)}_\zz$ 
and the leakage error detectability $\text{LED}^{(m)}_\zz$ can be given as follows:
\begin{align}
    &\begin{aligned}
        \text{RED}^{(m)}_\zz[\mu] &= D(\tilde{t}^{(m)}[n, \mu],\,\tilde{p}_\zz^{(m)}[n, \mu] - \tilde{t}^{(m)}[n, \mu]) \\
                       &= \norm[2][2]{W_{\tilde{t}^{(m)}[\mu]}[k]
                            \left(\tilde{P}_\zz^{(m)}[k, \mu] - \tilde{T}^{(m)}[k, \mu]\right)}
    \end{aligned}\\
    &\begin{aligned}
        \text{LED}^{(m)}_\zz[\mu] &= D(\tilde{t}^{(m)}[n, \mu],\,\tilde{p}_\zz^{(m)}[n, \mu]) \\
                       &= \norm[2][2]{W_{\tilde{t}^{(m)}[\mu]}[k]
                            \left(\tilde{P}_\zz^{(m)}[k, \mu]\right)}
    \end{aligned}
\end{align}
Here, $W_{\tilde{t}^{(m)}[\mu]}[k]$ can be understood as the perceptual weighting informed by the masking properties of the
frequency domain target $\tilde{t}^{(m)}[n, \mu]$.

What follows is the proposal of two perceptual sound zone algorithms using the proposed error detectabilities.

\subsection{Proposed Unconstrained Perceptual Pressure Matching Algorithm}
\label{ch:perceptual_sound_zone:perceptual_minimization:unconstrained}
This section proposes an algorithm which minimizes the detectability of the total error.
This is similar to the pressure matching approach introduced in \autoref{ch:sound_zone:approaches}, 
in which the total error is minimized.

Consider the following optimization problem:
\begin{equation}
    \begin{aligned}
    \argmin{\tilde{x}_\za^{(l)}[n,\mu],\,\tilde{x}_\zb^{(l)}[n,\mu]\,\forall\,l}{
       & \sum_{m\in A}\text{RED}_\za^{(m)}[\mu] + \sum_{m\in B}\text{LED}_\za^{(m)}[\mu] +\\
       & \sum_{m\in B}\text{RED}_\zb^{(m)}[\mu] + \sum_{m\in A}\text{LED}_\zb^{(m)}[\mu]
    }
    \end{aligned}
\end{equation}
The total detectability of the reproduction errors and the leakage errors is minimized by optimizing over the 
block-based representations of the loudspeaker input signals $\tilde{x}_\za^{(l)}[n,\mu]$ and $\tilde{x}_\zb^{(l)}[n,\mu]$.

\subsection{Proposed Constrained Perceptual Pressure Matching Algorithm}
\label{ch:perceptual_sound_zone:perceptual_minimization:constrained}
The previously discussed approach minimizes over the total detectability.
In this section, a perceptual sound zone algorithm is proposed that introduces constraints the problem. 

The motivation for this approach is that was found that the Par distortion detectability has a consistent perceptual interpretation.
For example, as mentioned in \autoref{ch:perceptual:implementation}, a detectability of 1 will consistently imply ``just noticeable''~\cite{van2005perceptual}.

This makes choosing constraints for detectability easier than typical non-perceptual pressure matching approaches.
These approaches typically directly constrain the sound pressure energy,
for which it is difficult to determine constraints as the sound pressure energy does not have a consistent perceptual interpretation.

This motivates the proposal of a perceptually constrained sound pressure approach.
In this approach, the reproduction error detectability will be constrained, while the leakage error detectability will be minimized in the cost function.
To this end, the following optimization problem is defined:
\begin{equation}
    \begin{aligned}
    \argmin{\tilde{x}_\za^{(l)}[n,\mu],\,\tilde{x}_\zb^{(l)}[n,\mu]\,\forall\,l}{
       & \sum_{m\in B}\text{LED}_\za^{(m)}[\mu] + \sum_{m\in A}\text{LED}_\zb^{(m)}[\mu]}\\
        \subjectto {& \text{RED}_\za^{(m)}[\mu] \leq D_0 \quad\forall\, m\in A\\
                    & \text{RED}_\zb^{(m)}[\mu] \leq D_0 \quad\forall\, m\in B}
    \end{aligned}
\end{equation}
Here, $D_0$ is the maximum allowed detectability of the reproduction error.

The optimization problem trades-off between reproduction error and leakage: 
relaxing the constraint on the detectability of the reproduction error should result in more mistakes in the reproduced sound pressure.,
this relaxation should however also allow for a greater minimization of the leakage detectability.

It should be noted that constraining the leakage error detectability is also a valid choice.
To the knowledge of the authors there is no precedent in literature in which a reproduction error is constrained.
As such, constraining the reproduction error detectability is found to be a more interesting exploration.

Constraining the detectability of the leakage error is found to be promising future work.
