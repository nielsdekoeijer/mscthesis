In \autoref{ch:perceptual} the Par detectability $D(x[n],\varepsilon[n])$ was introduced as the most promising perceptual model.
Recall that the Par detectability quantifies how detectable a disturbance signal $\varepsilon[n]$ is in presence of masking signal $x[n]$.

It was shown in \autoref{ch:perceptual:implementation} that the Par detectability can be expressed as follows: 
\begin{align}
    D(x[n],\varepsilon[n]) &= \norm[2][2]{W_{x}[k]\mathcal{E}[k]} 
\end{align}

In \autoref{ch:sound_zone}, it was found that a pressure matching sound zone approach is most suitable for integration with the Par detectability
due to their mathematical similarities.
As shown, the detectability is defined as the squared L2 norm of the disturbance weighted by diagonal matrix $\mat{W}_{x}$ determined by the masking signal.
Pressure matching minimizes the squared L2 sound pressure error between the target sound pressure and the sound pressure attained by the loudspeaker.

Recall that the pressure matching cost function is given as follows.
\begin{align}
    \argmin{\tilde{x}_{\za,\mu}^{(l)}[n],\,\tilde{x}_{\zb,\mu}^{(l)}[n]\,\forall\,l}{
       &\sum_{m\in A} \norm[2][2]{\tilde{P}_{\za}^{(m)}[k,\mu] - \tilde{T}^{(m)}[k,\mu]} +
        \sum_{m\in A} \norm[2][2]{\tilde{P}_{\zb}^{(m)}[k,\mu]} + \\
       &\sum_{m\in B} \norm[2][2]{\tilde{p}_{\zb}^{(m)}[k,\mu] - \tilde{T}^{(m)}[k,\mu]} + 
        \sum_{m\in B} \norm[2][2]{\tilde{P}_{\za}^{(m)}[k,\mu]}
    }
\end{align}
Using the definition of detectability given above, consider the following:
\begin{equation}
    D(\tilde{t}^{(m)}[n,\mu],\,\tilde{p}_{\zz}^{(m)}[n,\mu] - \tilde{t}^{(m)}[n,\mu]) = 
        \norm[2][2]{\mat{W}_{\tilde{t}^{(m)}[\mu]}\left(\tilde{P}_{\za}^{(m)}[k,\mu] - \tilde{T}^{(m)}[k,\mu]\right)} 
\end{equation}

Essentially, the approach is to replace the norms in the equation above by the perceptually weighted equivalent.
For each norm, we consider the masking effects determined by the target sound pressure for the respective point $m$. 
This results in the following cost function:
\begin{align}
    \argmin{\tilde{x}_{\za,\mu}^{(l)}[n],\,\tilde{x}_{\zb,\mu}^{(l)}[n]\,\forall\,l}{
       &\sum_{m\in A} \norm[2][2]{
        \vec{w}\left(\tilde{T}^{(m)}[k,\mu]\right)\left[\tilde{P}_{\za}^{(m)}[k,\mu] - \tilde{T}^{(m)}[k,\mu]\right]} + \\
       &\sum_{m\in A} \norm[2][2]{\vec{w}\left(\tilde{T}^{(m)}[k,\mu]\right)
        \left[\tilde{P}_{\zb}^{(m)}[k,\mu]\right]} + \\
       &\sum_{m\in B} \norm[2][2]{\vec{w}\left(\tilde{T}^{(m)}[k,\mu]\right)
        \left[\tilde{P}_{\zb}^{(m)}[k,\mu] - \tilde{T}^{(m)}[k,\mu]\right]} + \\
       &\sum_{m\in B} \norm[2][2]{\vec{w}\left(\tilde{T}^{(m)}[k,\mu]\right)
        \left[\tilde{P}_{\za}^{(m)}[k,\mu]\right]}
    }
\end{align}
The cost function behaves the same way, except all terms are now psycho-acoustically weighted.
The weighting vectors are chosen based on the target sound pressure in the relevant control point $m$.
This essentially exploits the psycho-acoustical masking properties of the target signal.
