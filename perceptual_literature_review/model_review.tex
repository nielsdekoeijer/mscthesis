This section will document a literature review into perceptual models of the human auditory system.
As stated in the introduction, this literature review is performed to determine which perceptual models are currently 
available in the state of the art.
This review will then be later used in \autoref{ch:perceptual:selection} to select a perceptual model most 
suitable for integration into a sound zone algorithm.

For this literature review, the goal was to document the perceptual models that were either promising for 
the integration into algorithms or for evaluating the quality of the output of algorithms.
These are models that attach some ``score'' or ``rating'' to the perceptual quality of input signals.
These ratings can then be used in algorithms to obtain an optimal rating through optimization, or to determine
the quality of results from later algorithms.

As such, the focus of the literature review is not on the latest findings in psycho-acoustics, or models 
that accurately emulate the behavior of the human ear, such as the Dau model.
Instead, two categories of perceptual models are considered.

First, ``Objective Measures'', which are discussed in \autoref{ch:perceptual:review:objective}, which attempt to predict
the perceptual quality ratings found in listening tests. 
And ``Audio Coding'' models, discussed in \autoref{ch:perceptual:review:audio_coding}, which are used to quantify how
perceptually audible the artifacts of compression in audio are.

\subsection{Objective Measures}
\label{ch:perceptual:review:objective}
In order to determine the perceived quality of audio one approach is to use listening tests.
These are tests in which subjects are asked to rate a property (or properties) of a set of audio stimuli.
One example where these tests are performed is for the evaluation of listening aids, where they are used determine 
the speech intelligibility~\cite{taal2011algorithm}.
Other examples include determining which loudspeaker has the best perceived sound quality.

Performing listening tests is however often cumbersome due to the large amount of human labour involved.
This motivates the use of objective quality measures, which attempt to predict the outcomes of these listening tests.
This is very useful for algorithm developers for example, as they can get an indication of how well they are doing
without having to perform a cumbersome test.
Note however that a objective quality measure does not replace a listening test: it can only be used to give an 
indication.

The objective measures that will be considered take a reference and degraded audio stimuli as inputs.
Most models then convert the input audio stimuli to a so-called internal representation, which models how the 
human auditory system perceives the stimuli.
Various features are then derived from the internal representation of the stimuli.
An estimator for teh quality of the stimuli is then created by fitting the features to the results of listening tests.

These objective quality measures are promising for integration into sound zone algorithms as they summarize the 
quality of a signal into a single value, which can be potentially optimized for. 
It stands to reason that if an objective quality measure correlates with audio quality, optimizing over such a measure
could improve sound zone algorithms.

As such, this section will explore various objective measures.
This will be done by considering various classes different objective measures, namely ``Objective Speech Quality Measures'',
``Objective Speech Intelligibility Measures'', and ``Objective Audio Quality Measures''. 

\subsubsection{Objective Speech Quality Measures}
There have been a number of attempts to create objective measures to quantify the quality of speech.
In this section three objective speech quality measures will be discussed.
Namely the Perceptual Evaluation of Speech Quality (PESQ)~\cite{rix2001perceptual} measure,
Perceptual Objective Listening Quality Assessment (POLQA)~\cite{beerends2013perceptual}. measure, and 
Virtual Speech Quality Objective Listener (ViSQOL)~\cite{hines2012visqol,chinen2020visqol} measure.

\begin{itemize}
    \item 
PESQ is a metric which attempts to determine the perceived quality of speech.
It was standardized by the International Telecommunication Union (ITU-T) in 2001.
PESQ is computed by first applying an auditory transform that maps the reference and degraded speech into a 
time-frequency representation of the perceived loudness.
From this internal representation, so-called symmetric and asymmetric disturbances are determined 
between the time-frequency bins of the reference and degraded speech. 
A non-linear average over the frequency bins is then taken to obtain the average disturbance per time bin.
These averaged disturbances are then mapped to the outcomes of listening test outcomes through linear 
combination~\cite{rix2001perceptual}.
    \item
POLQA is speech quality metric which was standardized by the International Telecommunication Union (ITU-T) in 2011. 
It is meant to be the successor of PESQ, with the intention of having more accurate predictions on a 
wider range of distortions.
POLQA works with a similar internal representation to PESQ, but computes distortion in a different way 
as to be capable of handing global temporal compression and expansions~\cite{beerends2013perceptual}.
    \item
ViSQOL is a metric developed in 2012 in a collaboration between Trinity College and Google.
ViSQOL uses a different internal representation than PESQ and POLQA as it uses the Neurogram Similarly Index Measure (NSIM)
to make its predictions.
Neurograms contain the neural firing activity of the auditory nerve in time-frequency bins, and NSIM determines how similar
the firing patterns of two neurograms are.
This similarity is then related to the outcomes of listening tests through a laplacian fit~\cite{hines2012visqol}.
\end{itemize}
In general, PESQ, POLQA and ViSQOL require many steps to compute and were found difficult to optimize for due to many
non-differentiable such as clipping and conditional branches within the algorithms.
Some attempts have been made however to reformulate PESQ in order to make it more tractable for optimization 
by approximating the disturbances by other functions~\cite{kim2019end}.

\subsubsection{Objective Speech Intelligibility}
Intelligibility of speech is defined as the percentage of words identified correctly given a degraded speech signal.
Objective speech intelligibility metrics seek to predict this percentage.
In this section, two of these metrics will be discussed.
Namely, the Short Time Objective Intelligibility (STOI)~\cite{taal2011algorithm} measure and the 
Speech Intelligibility In Bits (SIIB)~\cite{van2017instrumental} measure.

\begin{itemize}
    \item 
STOI was proposed by Taal et al. in 2011 as a speech intelligibility metric that could make accurate predictions 
for a speech signals that were distorted through a distortion that can be modeled as a time-frequency weighting.

It computes the internal representation of the reference and degraded speech signals by converting them into $1/3$ 
octave bands, and then segmented into short time frames.
The average correlation coefficient between the segments of the internal representation fo the reference and degraded 
segments is then computed, and averaged over all time segments and frequency bands to determine the intelligibility.
~\cite{taal2011algorithm}
    \item 
SIIB was introduced by Van Kuyk et al. in 2017 as a speech intelligibility metric that could be motivated through
information theory.
As such, the intelligibility metric is given in bits

The idea behind SIIB is that the intelligibility of speech is related to the information shared between 
intended and degraded speech.
As such, SIIB is computed through the mutual information rate between a clean speech signal and the speech signal
received by a listener.

In order to compute the mutual information rate, the paper models the transmission of an intended message from 
speaker to listener as a communication channel.
Among other aspects, this transmission channel includes a model of the human auditory system.
~\cite{van2017instrumental}
\end{itemize}

Both STOI and SIIB are difficult to optimize for directly. 

In STOI, the removal of silent regions and the clipping operator are non-differentiable operations.
Furthermore, the computation of the correlation coefficient is a non-convex function of the degraded speech.

SIIB is in general non-convex and non-differentiable as it uses the Karhunen-Lo\`eve transform and a 
K-nearest neighbor estimator to compute the mutual information.
However, if the communication channel is approximated as gaussian, the mutual information can be computed in closed form,
and SIIB becomes a differentiable measure.

\subsubsection{Objective Audio Quality Measures}
The previous objective quality metrics are both intended for evaluating speech.
In this section, a number of objective quality metrics will be discussed that are designed instead 
for perceived audio quality.
Namely, the Perceptual Evaluation of Audio Quality (PEAQ)~\cite{thiede2000peaq} 
and ViSQOLAudio~\cite{hines2015visqolaudio}.
The latter is an adapted version of the ViSQOL speech quality measures.

\begin{itemize}
    \item 
PEAQ is a audio quality metric standardized by the International Telecommunication Union (ITU-T).
PEAQ estimates a quality grade by first computing an internal representation  of
the reference and degraded audio signals.
This results in a time-frequency representation of the input stimuli from which a number of perceptually relevant feature,
referred to by PEAQ as Model Output Variables (MOVs), are extracted.
An example of these MOVs are the loudness of the noise or the bandwidth of the input stimuli.
These MOVs are then mapped to the final audio quality grade through a neural network~\cite{thiede2000peaq}.
    \item 
In 2015, it was found that with some adjustments ViSQOL could be used to determine audio quality, which resulted in a 
new metric ViSQOLAudio.
Among the adjustments were the removal of the voice activity detector included in ViSQOL and the use of a larger bandwidth
to cover the entire spectrum of hearing from 50 Hz to 20000 Hz, rather than just the bandwidth of speech.
\end{itemize}

PEAQ, and ViSQOLAudio are both difficult to optimize for.
A number of the MOVs computed in PEAQ, such as the partial noise loudness, are non-differentiable.
As ViSQOLAudio is similar to ViSQOL with some small adjustments, it is similarly difficult to optimize for.

\subsubsection{Distraction Model}
One especially promising objective measure is the distraction proposed by Francombe et al. in 2015~\cite{francombe2015model}.
This measure was designed with the application of sound zones in mind.

The distraction was determined to be the keyword that best describes the perceptual experience of 
interfering audio programs.
This was determined through an elicitation study performed also performed by Francombe et al. 
in 2014~\cite{francombe2014elicitation}.
This prompted the creation of the model.

To create the model, a listening test was performed where the participants were subjected to audio-on-audio interference.
The subjects were played a target audio stimuli they were instructed to focus listening to.
At the same time, an interferer audio stimuli was played to distract the participant from the target.
The participants were given a scale between 0 and 100 on which they were asked to rate how distracting the interference
was when listening to the target program, where a 100 was maximally distracted.

The target-interferer pairs and ratings resulted in a dataset.
This dataset was then used to fit a model which predicted the distraction given novel a target-interferer pair.
The model consisted of taking a linear combination of 5 features which could be computed through the audio files of the 
target and the interferer.

Computing said features could however not be performed in real time.
The reason for this was that as the original distraction model is too computationally complex~\cite{ramo2017real}.
To this end, in 2017, R\"am\"o et al. proposed a version of the distraction model that could be run in real-time.
This was done by approximating the features of the original distraction model by computationally less complex alternatives.
The resulting real-time distraction model was found to be less precise, but could be run in 0.04\% of the time of the 
original distraction model.

On face value, the distraction model is promising to optimize over.
However, while easy to compute, the real-time distraction model by R\"am\"o et al. is non-differentiable as the model uses
piecewise functions and non-convex due to taking the logarithm of the square of the input signals.
In addition to this, the model also performs operations that are difficult to express mathematically, such as counting 
the number of short-time blocks that exceed a certain threshold.

\subsection{Perceptual Models from Audio Coding }
\label{ch:perceptual:review:audio_coding}
The second class of perceptual models that will be considered are the perceptual models used in audio coding.
Audio coding algorithms attempt to find an low-bitrate representation of an audio input signal, as a form of compression.
This process is usually lossy, as reducing the bitrate introduces errors.
These errors can be a detriment to the listening experience.

As such, most audio coding algorithms use a perceptual model to quantify how disturbing the distortions are.
The perceptual model is used to introduce encoding errors in such a way that the audio output
signal is perceptually indistinguishable from the audio input signal~\cite{taal2012low}.
The perceptual model typically takes form of a distortion function which determines how
audible the difference between a reference input audio signal and a distorted output audio signal is.
This function is used to encode an input audio signal such that it has minimal distortion for a
specified bitrate.

The perceptual models used in audio coding are promising for integration into a sound zone algorithm, as they are 
often mathematically tractable.
As stated, these perceptual models typically take the form of some sort of distortion function that quantifies
how perceptually disturbing the introduced artifacts are. 
One approach could be to define sound zone algorithms that minimize a distortion function for example.

\subsubsection{ISO MPEG Models}
The ISO/IEC 11172-3 standard specifies a coded representation for audio files~\cite{ISO11172-3}, 
and a decoder for said representation.
An encoder said representation is not part of the standard.
This is done deliberately, to allow for future improvements to the encoder, without having to change the standard~\cite{pan1995tutorial}.

The standard does however provide a number of examples of possible encoders, with increasing complexity.
Alongside these example encoders, two psycho-acoustical models are included for use during the encoding process. 

The psycho-acoustical models work by subdividing the input audio signal into different frequency bands, 
modeling the frequency bands in the human auditory system.
The model then determines how much quantization noise can be added separately per band without the noise becoming audible.
As such, the model assumes that the distortion signal is noise-like~\cite{van2005perceptual}, which is usually
the case for quantization noise for audio coders.

The output of the psycho-acoustical model is thus the amount of noise that can be added per band.
In the case of audio coding, this can then be used to control quantization noise.
Note that this perceptual model does not come in the form of the earlier described distortion function.
This technique has however been used for various signal processing purposes, 
such as audio watermarking~\cite{taal2012low}, as such examples exist from which optimization schemes could be inspired.

\subsubsection{Par Detectability}
In 2005, van der Par et al. proposed a novel perceptual model designed for use in audio coding~\cite{van2005perceptual}.
The model defines a distortion measure which determines the ``detectability'' of a distortion signal 
in presence of a masking signal.
That is to say, the function quantifies the degree to which a human is to detect a distortion signal.
For audio coding purposes, this distortion signal is error introduced due to the audio compression.

The proposed method differentiates itself from the previously discussed ISO MPEG models in three ways.

Firstly, the paper uses newer findings from psycho-acoustic literature, namely spectral integration.
In spectral integration, the masking effects from neighboring bands are taken into account when computing the 
masking effects.
The psycho-acoustical models defined in the ISO MPEG standard does not do this as it effectively 
works independently per band~\cite{taal2012low}.

Secondly, it assumes that the distortion signal is sinusoidal, rather than noise-like.
As such, it is more effective in hiding sinusoidal distortion.

Thirdly and finally, the perceptual model is described as a distortion function which quantifies how 
detectable a disturbance stimuli is.
The proposed distortion measure can be expressed as an L2-norm.
This mathematical tractability makes for easy integration into existing least-square problems.
As such, the Par model has been used in many signal processing applications, 
examples ranging from speech enhancement to removing perceptually irrelevant sinusoidal 
components~\cite{balazs2009time, taal2013optimal}.

\subsubsection{Taal Detectability}
A paper from 2012 by Taal et al. proposed a novel perceptual model \cite{taal2012low} which also introduce the
detectability of a distortion signal in presence of a masking signal.
In a way, paper proposes an alternative definition to the detectability defined in the Par model.

In contrast to the Par detectability, the Taal detectability measure takes temporal characteristics of a signal into account.
The inclusion of temporal information allows for the suppression of ``pre-echoes'', which is an artifact that 
the Par model suffers from. 
The ``pre-echoes'' artifacts arises from the assumption that the masking effects of the masking signal are stationary across 
time. 
As a result, audio coding algorithms may assume that audio content is masked while it is not, which results quantization
noise not being masked.

In contrast to other temporal perceptual models, the Taal Detectability has a relatively low computational complexity.
In addition to this, it can also be expressed as an L2-norm, which makes it a good candidate for optimization.
The computational demand was however shown to be higher than the Par Detectability~\cite{taal2012low}, especially for larger number of input samples.
