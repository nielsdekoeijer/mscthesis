In \autoref{ch:perceptual:selection}, it was determined that the Par detectability was the 
most suited model for integration with sound zone algorithm of all perceptual models considered in 
the literature review given in \autoref{ch:perceptual:review}.
In this section, the perceptual model is considered in greater detail in order to give the reader a greater
understanding of the implementation and behavior of the model.

This section is organized as follows.
In order to optimize over the model, a suitable expression for it must be found.

\subsection{Introduction to Par Detectability}
\label{ch:perceptual:implementation:intuition}
In this section, a high-level description of the Par detectability will be given.
This is done to give the reader an understanding of the model before going into greater detail.
The Par detectability defines a function $D(x[n],\varepsilon[n])$.
Here, $x[n]\in\Real{N_x}$ is the sound pressure of the masking signal, 
and $\varepsilon[n]\in\Real{N_x}$ is the sound pressure of disturbance signal.
Note how the model specifically uses sound pressure.
This will be discussed further in \autoref{ch:perceptual:implementation:calibration}.

For the model to be accurate, the signals $x[n]$ and $\varepsilon[n]$ should be short-time signals.
The paper uses a signal length $N_x$ corresponding to between 20 to 200 milliseconds.    
This is important, as the model assumes that the psycho-acoustical properties of $x[n]$ 
and $\varepsilon[n]$ are stationary.  

It is assumed that a human is listening to both the masking signal and the disturbance signal at the same time.
The detectability can then be understood as the probability that a human listener can detect the disturbance signal 
$\varepsilon[n]$ in presence of the masking signal $x[n]$~\cite{van2005perceptual}.
The signal $x[n]\in\Real{N}$ is referred to as the masking signal because it masks the disturbance signal 
$\varepsilon[n]$ to a degree.

The metric is normalized in such a way that the detectability $D(x[n],\varepsilon[n])$ is equal to 1 when the 
disturbance signal $\varepsilon[n]$ is just noticeable in presence of masking signal $x[n]$.
The detectability $D(x[n],\varepsilon[n])$ can also attain a value larger than $1$.
The larger values of the detectability correspond with an increased perceived presence of the
disturbance signal $\varepsilon[n]$.

\subsection{Computation Details of the Par Detectability}
\label{ch:perceptual:implementation:computation}
This section will explore calculating the Par detectability.
The first thing to note about the Par detectability is that it operates in the frequency domain. 
To this end, let $X[k]$ and $\mathcal{E}[k]$ denote the frequency domain representations of the masking signal $x[n]$ and 
the disturbance signal $\varepsilon[n]$ respectively.

The Par detectability starts by computing an internal representation of the input signals $X[k]$ and $\mathcal{E}[k]$.
This internal representation models how the input signals appear in the human auditory system.

The Par detectability models this filtering in two steps.
The first step models how parts of the ear filter the incoming sound with an outer- and middle-ear filter $H_\text{om}[k]$. 
Next, a $4^\text{th}$ order Gammatone filter bank is applied, modeling the filtering of the 
basilar membrane inside the ear~\cite{van2005perceptual}.
Note that the exact same filtering is applied to both $X[k]$ and $\mathcal{E}[k]$.

The Gammatone filter bank consists of $N_g$ filters, which will denoted by $\Gamma_i[k]$, for $1 \leq i \leq N_g$. 
To model the frequency-place transform that occurs in the the basilar membrane, 
the filters in the filter bank $\Gamma_i[k]$ have a bandwidth given by the equivalent 
rectangular bandwidth (ERB) and center frequencies given by the corresponding equivalent rectangular bandwidth number
scale (ERBS).
A possible expression for the gammatone filters $\Gamma_i[k]$ are given by the original paper~\cite{van2005perceptual}. 

After filtering, the power per Gammatone filter tap is computed.
Let $M_i$ and $S_i$ denote the output power of the $i^\textt{th}$ filter tap of the for $X[k]$ and 
$\mathcal{E}[k]$ respectively.
The relationship between the input quantities and the output power of the filter taps can be given as follows:
\begin{align}
    M_i &= \frac{1}{N_x}\sum_{k=0}^{N_x}\left|H_\text{om}[k]\right|^2\left|\Gamma_i[k]\right|^2\left|X[k]\right|^2 \\
    S_i &= \frac{1}{N_x}\sum_{k=0}^{N_x}\left|H_\text{om}[k]\right|^2\left|\Gamma_i[k]\right|^2\left|\mathcal{E}[k]\right|^2 
\end{align}
The output powers can then be used to define the within-channel detectability $D_i$ for $1 \leq i \leq N_g$.
This can be thought of the detectability per filter tap.
The within-channel detectability is defined as follows:
\begin{align}
    D_i = \frac{N_xS_i}{N_xM_i + C_a}
\end{align}
Here, $C_a$ is a calibration constant that ensures that the absolute threshold of hearing is predicted correctly.
This can be understood by considering the case where no masking signal $x[n]$ is present, 
in which case $M_i = 0$ for all $i$.
If not for the calibration constant $C_a$, the detectability of any non-zero disturbance $\varepsilon[n]$ would be infinite.

The total detectability $D(x[n],\varepsilon[n])$ can then be computed as the scaled sum of all within channel detectabilities.
It is defined as follows:
\begin{align}
    D(x[n],\varepsilon[n]) &= C_s L_\text{eff}\sum_{i=0}^{N_g} D_i \\
                        &= C_s L_\text{eff}\sum_{i=0}^{N_g} 
                        \frac{\sum_{k=0}^{N_x}\left|H_\text{om}[k]\right|^2\left|
                            \Gamma_i[k]\right|^2\left|\mathcal{E}[k]\right|^2}
                        {\sum_{k=0}^{N_x}\left|H_\text{om}[k]\right|^2\left|
                            \Gamma_i[k]\right|^2\left|X[k]\right|^2 + C_a}
\end{align}
Here, $C_s$ is a calibration constant chosen such that a just noticeable disturbance signal results in a 
detectability of $D(x[n],\varepsilon[n]) = 1$. 
The constant $L_\text{eff}$ is the integration time of the human auditory system.
It is chosen equal to the segment length of $x[n]$ and $\varepsilon[n]$ in milliseconds.  

Consider the behavior of the expression of the detectability $D(x[n],\varepsilon[n])$ above.
Imagine that the spectrum of the masking signal is much larger than the disturbance signal, 
i.e. $X[k] \gg \mathcal{E}[k]$ for all frequency bins $k$.
In this case, the detectability of $\varepsilon[n]$ will be small due to the masking of the masking signal $x[n]$ or
inaudible due to the threshold of hearing determined by $C_a$.

Conversely consider the case that the spectrum of the masking signal is much smaller than the disturbance signal,
i.e. $X[k] \ll \mathcal{E}[k]$ for all frequency bins $k$.
In this case, the resulting detectability is determined greatly by the calibration coefficient $C_a$. 
If the total energy of the filtered disturbance signal $S_i \gg C_a$ for all $i$,
the detectability becomes large, as the disturbance signal is large relative to the threshold of hearing.
Alternatively, if $S_i \ll C_a$ for all $i$, the disturbance signal is inaudible due to the threshold of hearing and 
the detectability will be low accordingly.

This concludes the analysis of the Par model.
What follows is the discussion of the calibration of the Par model in \autoref{ch:perceptual:implementation:calibration},
where calibration constants $C_a$ and $C_s$ are determined.
Afterwards, the model above will be given as a least-squares formulation in order to ease the integration
into optimization in \autoref{ch:perceptual:implementation:least_squares}.

\subsection{Calibration of the the Par Detectability}
\label{ch:perceptual:implementation:calibration}
This section will describe the calibration of the Par Detectability.
As mentioned, a just noticeable disturbance signal must result in a detectability $D(x[n],\varepsilon[n])$ of $1$.
In order to do so, calibration constants $C_a$ and $C_s$ are chosen in a certain way.

This can be done in different ways, but the paper~\cite{van2005perceptual} 

\subsection{Least-Squares Formulation of the Par Detectability}
\label{ch:perceptual:implementation:least_squares}
