In \autoref{ch:perceptual:selection}, it was determined that the Par detectability measure is 
the most suited model for the perceptual sound zone algorithm of all perceptual models considered in 
the literature review given in \autoref{ch:perceptual:review}.
In this section, in order to give the reader a greater understanding of the model, 
the Par detectability measure is considered in greater detail.

This section is organized as follows.
First, \autoref{ch:perceptual:implementation:intuition} gives a high-level description of the Par detectability measure, providing 
an intuitive understanding and introducing some of the notation that is used.
Next, the steps to computing the detectability are described in \autoref{ch:perceptual:implementation:computation}.
Finally, \autoref{ch:perceptual:implementation:least_squares} rewrites the detectability as a squared L2-norm.

\subsection{High-Level Description of the Par Detectability Measure}
\label{ch:perceptual:implementation:intuition}
In this section, a high-level description of the Par detectability measure is given.
This is done to give the reader a basic understanding of the model before going into greater detail.

The Par detectability maps two input sequences to a positive real value, i.e. $D: (\Real{N_x}, \Real{N_x})\mapsto \Real{+}$.
The two input sequences are the masking signal $x[n]\in\Real{N_x}$ and the disturbance signal $\varepsilon[n]\in\Real{N_x}$.
The detectability of these two sequences is denoted as $D(x[n], \varepsilon[n])$. 

Imagine a human that is listening to both the masking signal $x[n]$ and the disturbance signal $\varepsilon[n]$ at the same time.
The detectability $D(x[n], \varepsilon[n])$ can be understood as how easily a human listener can detect the disturbance signal 
$\varepsilon[n]$ in presence of the masking signal $x[n]$.
The signal $x[n]$ is referred to as the masking signal because its masking properties are model to determine how well it masks the disturbance signal 
$\varepsilon[n]$.

For this interpretation to be accurate, the signals $x[n]$ and $\varepsilon[n]$ should be short-time signals.
The paper uses a signal length of to 20 to 200 milliseconds.    
This is important, as the model assumes that the psycho-acoustical properties of $x[n]$ 
and $\varepsilon[n]$ are stationary.  

The measure is normalized in such a way that the detectability $D(x[n],\varepsilon[n])$ is equal to 1 when the 
disturbance signal $\varepsilon[n]$ is ``just noticeable'' in presence of masking signal $x[n]$.
That is to say: if the detectability is 1, the disturbance is on the verge of being noticeable and not noticeable.

The detectability $D(x[n],\varepsilon[n])$ can also attain a value larger than $1$.
The larger values of the detectability correspond with an increased perceived presence of the
disturbance signal $\varepsilon[n]$.

\subsection{Computation Details of the Par Detectability}
\label{ch:perceptual:implementation:computation}
This section explores calculating the Par detectability.
The first thing to note about the Par detectability is that it is computed using the frequency domain representations of its inputs~\cite{van2005perceptual}. 
To this end, let $X[k]$ and $\mathcal{E}[k]$ denote the frequency domain representations of the masking signal $x[n]$ and 
the disturbance signal $\varepsilon[n]$ respectively.

After determining the frequency domain representations, 
the Par detectability computes an internal representation of the input signals $X[k]$ and $\mathcal{E}[k]$.
This internal representation models how the input stimuli appear to the human auditory system.
For the Par detectability measure, this is modeled by filtering the input stimuli.

Two subsequent filters are applied.
The first filter models how parts of the ear filter the incoming sound with an outer- and middle-ear filter $H_\text{om}[k]$. 
Next, a $4^\text{th}$ order Gammatone filter bank is applied, modeling the frequency-place transform that occurs in 
basilar membrane inside of the ear~\cite{van2005perceptual}.

The Gammatone filter bank consists of $N_g$ filters.
The frequency domain representation of each individual filter is denoted by $\Gamma_i[k]$, for $1 \leq i \leq N_g$. 
The filters in the filter bank $\Gamma_i[k]$ have a bandwidth given by the equivalent 
rectangular bandwidth (ERB) and center frequencies given by the corresponding equivalent rectangular bandwidth number
scale (ERBS).
Expressions for the gammatone filters $\Gamma_i[k]$ are provided by the original paper~\cite{van2005perceptual}.

After filtering, the power per Gammatone filter tap is computed.
Let $M_i$ and $S_i$ denote the output power of the $i^\text{th}$ filter tap for the masking signal $X[k]$ and 
the disturbance signal $\mathcal{E}[k]$ respectively.
This output power can be understood as the amount of power perceived per frequency band of the human ear. 
The relationship between the input quantities and the output power of the filter taps can be given as follows:
\begin{align}
    M_i &= \frac{1}{N_x}\sum_{k=0}^{N_x-1}\left|H_\text{om}[k]\right|^2\left|\Gamma_i[k]\right|^2\left|X[k]\right|^2 \\
    S_i &= \frac{1}{N_x}\sum_{k=0}^{N_x-1}\left|H_\text{om}[k]\right|^2\left|\Gamma_i[k]\right|^2\left|\mathcal{E}[k]\right|^2 
\end{align}
The output powers can then be used to define the within-channel detectability $D_i$ per filter tap $i$.
This can be thought of the detectability per frequency band of the human ear, and is defined as follows:
\begin{align}
    D_i = \frac{N_xS_i}{N_xM_i + C_a}
\end{align}
Here, $C_a$ is a calibration constant that ensures that the absolute threshold of hearing is predicted correctly.
This can be understood by considering the case where no masking signal $x[n]$ is present, 
in which case $M_i = 0$ for all $i$.
If not for the calibration constant $C_a$, the detectability of any non-zero disturbance $\varepsilon[n]$ would be infinite.
In order to take the frequency-dependence of the threshold of hearing into account, the previously described outer- and middle ear filters are defined as the
inverse of the threshold of hearing~\cite{van2005perceptual}.

The total detectability $D(x[n],\varepsilon[n])$ can then be computed as the scaled sum of all within channel detectabilities.
It is defined as follows:
\begin{align}
    D(x[n],\varepsilon[n]) &= C_s L_\text{eff}\sum_{i=0}^{N_g} D_i \\
                        &= C_s L_\text{eff}\sum_{i=0}^{N_g} 
                        \frac{\sum_{k=0}^{N_x-1}\left|H_\text{om}[k]\right|^2\left|
                            \Gamma_i[k]\right|^2\left|\mathcal{E}[k]\right|^2}
                        {\sum_{k=0}^{N_x-1}\left|H_\text{om}[k]\right|^2\left|
                            \Gamma_i[k]\right|^2\left|X[k]\right|^2 + C_a}
    \label{eq:perceptual:implementation:computation:detectability}
\end{align}
Here, $C_s$ is a calibration constant chosen such that a just noticeable disturbance signal results in a 
detectability of $D(x[n],\varepsilon[n]) = 1$. 
The constant $L_\text{eff}$ is the integration time of the human auditory system.
It is chosen equal to the segment length of $x[n]$ and $\varepsilon[n]$ in milliseconds.  

In order to further understand detectability, consider the behavior of the expression of the detectability $D(x[n],\varepsilon[n])$ above.
Imagine that the spectrum of the masking signal is much larger than the disturbance signal, 
i.e. $X[k] \gg \mathcal{E}[k]$ for all frequency bins $k$.
In this case, the detectability of $\varepsilon[n]$ will be small due to the masking of the masking signal $x[n]$ or
due to the threshold of hearing (determined by the calibration constant $C_a$).

Conversely consider the case that the spectrum of the masking signal is much smaller than the disturbance signal,
i.e. $X[k] \ll \mathcal{E}[k]$ for all frequency bins $k$.
In this case, the resulting detectability is determined greatly by the calibration coefficient $C_a$: 
\begin{itemize}
    \item If the total energy of the filtered disturbance signal is much larger than the calibration constant 
$S_i \gg C_a$ for all $i$, the detectability becomes large.
This models the case that the disturbance signal is large relative to the threshold of hearing.
    \item Alternatively, if $S_i \ll C_a$ for all $i$, the disturbance signal is inaudible due to the threshold of hearing and 
the detectability will be low accordingly.
\end{itemize}

This concludes the analysis of the Par model.
The determination of the calibration constants $C_a$ and $C_s$ is discussed in \autoref{ch:perceptual:implementation:calibration}.


\subsection{Least-Squares Formulation of the Par Detectability}
\label{ch:perceptual:implementation:least_squares}
This section will rewrite the previously introduced detectability into a least-squares representation. 
This representation is more mathematically tractable than \autoref{eq:perceptual:implementation:computation:detectability} and thus 
will allow for easier integration into existing sound zone algorithms.

To obtain this expression, the sum of squares will be expressed as a L2-norm.
Consider the following rewrite of the detectability given in \autoref{eq:perceptual:implementation:computation:detectability}: 
\begin{align*}
    D(x[n],\varepsilon[n]) &= C_s L_\text{eff}\sum_{i=0}^{N_g}
                        \frac{\sum_{k=0}^{N_x-1}\left|H_\text{om}[k]\right|^2\left|
                            \Gamma_i[k]\right|^2\left|\mathcal{E}[k]\right|^2}
                        {\sum_{k=0}^{N_x-1}\left|H_\text{om}[k]\right|^2\left|
                            \Gamma_i[k]\right|^2\left|X[k]\right|^2 + C_a} \\
                           &= \sum_{i=0}^{N_g}
                           \left(\frac{C_s L_\text{eff}}{\norm[2][2]{H_\text{om}[k]\Gamma_i[k]X[k]} + C_a}\right)
                        \sum_{k=0}^{N_x-1}\left|H_\text{om}[k]\right|^2\left|
                        \Gamma_i[k]\right|^2\left|\mathcal{E}[k]\right|^2 \\
                           &= \sum_{k=0}^{N_x-1}\left(\sum_{i=0}^{N_g}\frac{C_s L_\text{eff}\left|\Gamma_i[k]\right|^2}{\norm[2][2]{H_\text{om}[k]\Gamma_i[k]X[k]} + C_a}\right)
                        \left|H_\text{om}[k]\right|^2\left|\mathcal{E}[k]\right|^2 \\
                           &= \sum_{k=0}^{N_x-1}\left|W_x[k]\right|^2\left|\mathcal{E}[k]\right|^2 \\
                           &= \norm[2][2]{W_x[k]\mathcal{E}[k]} 
\end{align*}
The rewrite above introduced perceptual weighting $W_x[k]\in\Real{N_x}$ describing the masking effects of the masking signal $x[n]$. 
The perceptual weighting $W_x[k]$ is defined as follows: 
\begin{equation}
    W_x[k] = \left(\sqrt{\sum_{i=0}^{N_g}\frac{C_s L_\text{eff}\left|\Gamma_i[k]\right|^2}{\norm[2][2]{H_\text{om}[k]\Gamma_i[k]X[k]} + C_a}}\right)
                        \left|H_\text{om}[k]\right|
\end{equation}
Note from this formulation that the perceptual weighting is only a function of the masking signal $x[n]$.

Note also that the resulting detectability $D(x[n],\varepsilon[n])$ is a convex function of the disturbance signal $\varepsilon[n]$. 
The frequency-domain representation $\mathcal{E}[k]$ is related to the time-domain representation $\varepsilon[n]$ through the DFT, which is a linear operator.
The perceptual weighting of $\mathcal{E}[k]$ performed by $W[k]$ is also a linear operation.
As such, $W[k]\mathcal{E}[k]$ is an affine function of $\varepsilon[n]$.
