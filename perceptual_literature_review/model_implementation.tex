In \autoref{ch:perceptual:selection}, it was determined that the Par detectability was the perceptual model
most suited for integration in a sound zone algorithm from the perceptual models considered in 
the literature review given in \autoref{ch:perceptual:review}.
In this section, the perceptual model is considered in greater detail.

This section is organized as follows.
In order to optimize over the model, a suitable expression for it must be found.

\subsection{Introduction to Par Detectability}
\label{ch:perceptual:implementation:intuition}
In this section, a high-level description of the Par detectability will be given.
This is done to give the reader an understanding of the model before going into greater detail.
The Par detectability defines a function $D(x[n],\varepsilon[n])$.
Here, $x[n]\in\Real{N_x}$ is the masking signal, and $\varepsilon[n]\in\Real{N_x}$ is the disturbance signal.

For the model to be accurate, the signals $x[n]$ and $\varepsilon[n]$ should be short-time signals.
The paper uses a signal length $N_x$ corresponding to between 20 to 200 milliseconds.    
This is important, as the model assumes that the psycho-acoustical properties of $x[n]$ and $\varepsilon[n]$ are stationary.  
Note that the Taal detectability does not make this assumption~\cite{taal2012low}.

It is assumed that a human is listening to both the masking signal and the disturbance signal at the same time.
The detectability can then be understood as the probability that a human listener can detect the disturbance signal 
$\varepsilon[n]$ in presence of the masking signal $x[n]$~\cite{van2005perceptual}.
The signal $x[n]\in\Real{N}$ is referred to as the masking signal because it masks the disturbance signal 
$\varepsilon[n]$ to a degree.

The metric is normalized in such a way that the detectability $D(x[n],\varepsilon[n])$ is equal to 1 when the 
disturbance signal $\varepsilon[n]$ is just noticeable in presence of masking signal $x[n]$.
This can be understood as the probability of detecting $\varepsilon[n]$ in presence of $x[n]$ 
being equal to $1$, or a $100\%$.

The detectability $D(x[n],\varepsilon[n])$ can also attain a value larger than $1$, which clashes with the 
interpretation of detectability as a probability.
In fact, a detectability of larger than $1$ will often be used in this work.
The larger values of the detectability correspond with a more noticeable disturbance signal $\varepsilon[n]$ 
in presence of masking signal $x[n]$. 

\subsection{Computation Details of the Par Detectability}
\label{ch:perceptual:implementation:computation}
This section will explore the Par detectability.
Here, it will be shown how human auditory system is modeled and how the detectability is computed.

The first thing to note about the Par detectability is that it operates in the frequency domain version 
To this end, let $X[k]$ and $\mathcal{E}[k]$ denote the frequency domain representations of the masking signal $x[n]$ and 
the disturbance signal $\varepsilon[n]$ respectively.

The Par detectability starts by computing an internal representation of the input signals $X[k]$ and $\mathcal{E}[k]$.
This internal representation models the filtering that occurs in the human auditory system.
Note that the exact same filtering is applied to both $X[k]$ and $\mathcal{E}[k]$.

The Par detectability models this filtering in two steps.
The first step models how parts of the ear filter the incoming sound with an outer- and middle-ear filter $H_\text{om}[k]$. 
Next, a $4^\text{th}$ order Gammatone filter bank is applied, modeling the filtering of the 
basilar membrane inside the ear~\cite{van2005perceptual}.
The Gammatone filter bank consists of $N_g$ filters, which are denoted by $\Gamma_i[k]$ 

After filtering, the power per Gammatone filter tap is computed.
Let $M_i$ and $S_i$ denote the output power of the $i^\textt{th}$ filter tap of the for $X[k]$ and 
$\mathcal{E}[k]$ respectively.
The relationship between the input quantities and the output power of the filter taps can be given as follows:
\begin{align}
    M_i &= \frac{1}{N_x}\sum_{k=0}^{N_x}\left|H_\text{om}[k]\right|^2\left|\Gamma_i[k]\right|^2\left|X[k]\right|^2 \\
    S_i &= \frac{1}{N_x}\sum_{k=0}^{N_x}\left|H_\text{om}[k]\right|^2\left|\Gamma_i[k]\right|^2\left|\mathcal{E}[k]\right|^2 
\end{align}
The output powers can then be used to define the within-channel detectability $D_i$.
This can be thought of the detectability per filter tap.
The within-channel detectability is defined as follows:
\begin{align}
    D_i = \frac{N_xS_i}{N_xM_i + C_a}
\end{align}
Here, $C_a$ is a calibration constant that models the absolute threshold of hearing.
This can be understood by considering the case where $M_i = 0$.

The total detectability $D(x[n],\epsilon[n])$ can then be computed by combining the within channel detectabilities.
It is defined as follows:
\begin{align}
    D(x[n],\epsilon[n]) &= C_s L_\text{eff}\sum_{i=0}^{N_g} D_i \\
                        &= C_s L_\text{eff}\sum_{i=0}^{N_g} 
                        \frac{\sum_{k=0}^{N_x}\left|H_\text{om}[k]\right|^2\left|
                            \Gamma_i[k]\right|^2\left|\mathcal{E}[k]\right|^2}
                        {\sum_{k=0}^{N_x}\left|H_\text{om}[k]\right|^2\left|
                            \Gamma_i[k]\right|^2\left|X[k]\right|^2 + C_a}
\end{align}
Here, $C_s$ is a calibration constant chosen such that a just noticeable disturbance signal results in a 
detectability of $D(x[n],\epsilon[n]) = 1$. 
The constant $L_\text{eff}$ is the integration time of the human auditory system.
It can be chosen equal to the length of $x[n]$ and $\epsilon[n]$ in milliseconds.  

\subsection{Calibration of the the Par Detectability}
\label{ch:perceptual:implementation:calibration}
This section will describe the calibration of the Par Detectability.
As mentioned, a just noticeable disturbance signal must result in a detectability $D(x[n],\epsilon[n])$ of $1$.
In order to do so, calibration constants $C_a$ and $C_s$ are chosen in a certain way.

This can be done in different ways, but the paper~\cite{van2005perceptual} 
