In the field of psycho-acoustics significant research has been done in characterizing the auditory 
perception and time-frequency
analysis capabilities of the human ear~\cite{painter2000perceptual}.
From this understanding, a number of perceptual models have been proposed which aim to model the 
perception of auditory stimuli by humans~\cite{van2005perceptual}.

Perceptual models are employed for various purposes.
Objective audio quality measures, for example, are perceptual models which aim to predict the 
perceived quality of audio~\cite{torcoli2021objective}.
In another example, perceptual audio coding uses models of auditory perception to minimize the perceived artifacts introduced when
performing the compression of audio~\cite{herre2019psychoacoustic}.

In general, many perceptual models operate on an time-frequency internal-ear representation of the input stimuli which is obtained by applying a 
analysis filter bank. 
Among other effects, the filtering performed by the human ear is often taken 
into account at this stage~\cite{van2005perceptual, taal2012low}.
This representation is then used to determine its perceptually relevant aspects 
of the input stimuli~\cite{herre2019psychoacoustic}.

One aspect often used in perceptual models are the various auditory masking properties of the 
input stimuli~\cite{herre2019psychoacoustic}.
In general, auditory masking refers to the effects one sound has on the perception of other sounds~\cite{painter2000perceptual}.
In simultaneous masking for example, one loud tone may overpower a tone of a similar frequency, rendering the 
latter tone inaudible~\cite{painter2000perceptual}.

Another aspect that is often used is the ``threshold of hearing'', which determines the minimum sound pressure level 
that can be perceived by a human~\cite{herre2019psychoacoustic}.
Combining this principal with the masking properties, one can define the ``masking threshold'' of an input stimuli.
This threshold determines the sound pressure level required for other stimuli to be audible to a human observer in presence 
of the input stimuli~\cite{painter2000perceptual}, and is often used in perceptual 
models~\cite{van2005perceptual, taal2012low}. 
For example, the threshold of hearing is used in perceptual audio coding to 
make the coding artifacts inaudible~\cite{herre2019psychoacoustic}.

This chapter will motivate the use of the ``Par distortion detectability'' as the perceptual model used in 
perceptual sound zone algorithm proposed in \autoref{ch:perceptual_sound_zone}.
In doing so, a number of other perceptual measures are discussed in detail, some of which are used 
in \autoref{ch:results} to evaluate the performance of the proposed algorithm.

The structure of this chapter is given as follows.
\begin{itemize}
    \item This chapter begins with \autoref{ch:perceptual:review} which documents a review of 
        possible candidate perceptual models from literature for use in the perceptual sound zone algorithm. 
    \item Next, \autoref{ch:perceptual:selection} motivates the selection of one of the reviewed candidates, 
        namely the ``Par distortion detectability'', as the perceptual model for use in the 
        proposed perceptual sound zone algorithm.
    \item Finally, the implementation and behavior of the ``Par distortion detectability'' is discussed in more detail in 
        \autoref{ch:perceptual:implementation}.
\end{itemize}
