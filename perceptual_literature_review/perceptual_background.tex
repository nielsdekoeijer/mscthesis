In the field of psycho-acoustics significant research has been done in characterizing the auditory perception and time-frequency
analysis capabilities of the human ear~\cite{painter2000perceptual}.
From this understanding, a number of perceptual models have been proposed which aim to model perception of auditory stimuli~\cite{van2005perceptual}.

Perceptual models are proposed for various purposes.
Objective audio quality measures, for example, are perceptual models which aim to predict the perceived quality of audio~\cite{torcoli2021objective}.
In another example, perceptual audio coding uses models of auditory perception to minimize the perceived artifacts when 
performing compression of audio~\cite{herre2019psychoacoustic}.

In general, many perceptual models apply an analysis filter bank to obtain a time-frequency representation of the stimuli.
Often, the of the human human ear is also taken into account at this stage~\cite{van2005perceptual, taal2012low}.
This representation is used to determine its perceptually relevant aspects~\cite{herre2019psychoacoustic}.

One aspect often used in perceptual models are are the auditory masking properties of the input stimuli~\cite{herre2019psychoacoustic}.
In general, auditory masking refers the effects one sound has on the perception of other sounds~\cite{painter2000perceptual}.
In simultaneous masking for example, one loud tone may overpower a tone of a similar frequency, rendering it inaudible~\cite{painter2000perceptual}.

Another aspect that is often used is the ``threshold of hearing'', which determines the minimum sound pressure level 
that can be perceived by a human~\cite{herre2019psychoacoustic}.

Using these principals, one can define the ``masking threshold'' of an input stimuli.
This determines the sound pressure level required for other stimuli to be audible to a human observer in presence 
of the input stimuli~\cite{painter2000perceptual}. 
These principals are used in perceptual audio coding to make the encoding noise inaudible~\cite{herre2019psychoacoustic}.

\subsection*{Chapter Structure}
The goal of this chapter is to find a suitable perceptual model for the creation of a perceptual sound zone algorithm.
This chapter is structured as follows.

\begin{itemize}
    \item This chapter begins with a literature review into perceptual models is given in \autoref{ch:perceptual:review}.
The purpose of this review is to document candidates for the perceptual model that will be used in the perceptual sound zone algorithm.
In addition to this, the reviewed models could also serve as potential candidates for use in the evaluation 
of the perceptual sound zone algorithm that will be proposed in \autoref{ch:perceptual_sound_zone}.
    \item To perform the selection of a perceptual model from the candidates discussed in the literature review, 
criteria reflecting desirable properties for the model for use in sound zones are be defined in \autoref{ch:perceptual:selection}. 
The criteria are then used to select a perceptual model.
    \item Afterwards, the selected perceptual model is discussed in more detail in \autoref{ch:perceptual:implementation} by stating implementation 
details and describing its behavior.
\end{itemize}
