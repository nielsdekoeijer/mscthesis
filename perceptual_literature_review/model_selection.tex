In \autoref{ch:perceptual:review} a literature review of various models of the human auditory system was provided.
This section will determine criteria with which one of the discussed perceptual models can be selected for use
in integration into a sound zone algorithm.
The other perceptual models will still be of use however during the evaluation of the results of the perceptual
sound zone algorithm that will be derived in \autoref{ch:perceptual_sound_zone}.

The structure of this section is as follows.
First, \autoref{ch:perceptual:selection:criteria} will discuss the criteria that will inform the decision.
Next, in \autoref{ch:perceptual:selection:selection} said criteria will be used to select a suitable perceptual models 
and reflect on this choice.

\subsection{Criteria for Selecting Perceptual Model}
\label{ch:perceptual:selection:criteria}
This section will define desirable criteria for the perceptual model for integration with a sound zone algorithm.
Two criteria can be distinguished.
\begin{enumerate}
    \item \textbf{Mathematical Tractability:}\\
        It is a desirable property for the perceptual model to be easy to include in optimization problems mathematically.
        As will be shown in \autoref{ch:sound_zone}, many sound zone algorithms are posed as (convex) optimization 
        problems.

        Therefore if a perceptual model is too complicated to be integrated mathematically into optimization problems are
        undesirable.
        For example, if the computation of a perceptual model involves conditional branching (i.e. if-statements), their
        integration into an existing optimization problem may be difficult.

        Furthermore, as most sound zone algorithms can be posed as a convex optimization problem, it is also desirable that
        the perceptual models preserve this convexity.
        Convexity is a desirable property for optimization, as it guarantees that the optimization problem has a single, 
        global optimal value, rather than many locally optimal values~\cite{boyd2004convex}. 

    \item \textbf{Computational Overhead:}\\
        It is desirable for the inclusion of the perceptual model to add minimal computational load 
        to the sound zone algorithm.
        As such, the additional overhead of the perceptual model should not increased the run time of the sound zone 
        algorithm by many orders of magnitude.
\end{enumerate}

\subsection{Selection of Perceptual Model}
\label{ch:perceptual:selection:selection}
This section will use the criteria defined by \autoref{ch:perceptual:selection:criteria} in order to select a perceptual
model.
In the literature review given in \autoref{ch:perceptual:review}, two classes of perceptual model were considered:
perceptual models from audio coding and objective audio measures.

All objective audio measures were found to be mathematically untractable, as all models are both non-differentiable and 
non-convex functions of their input signals.
As such, they are difficult to integrate into existing sound zone algorithms.
However, as the objective audio measures predict the outcomes of listening tests, they are uniquely suited for evaluation
of perceptual sound zone algorithm that is to be proposed in \autoref{ch:perceptual_sound_zone}.

All models from audio coding are mathematically tractable and of low computational overhead.
The ISO MPEG models were found to be less promising than the Par and Taal detectability, as it does not immediately 
define a cost function which can be optimized over:
the models only determine how much noise can be added per band.

As such, the decision is between the Par and Taal detectability.
The Taal detectability takes into account temporal properties of the input signal in it's perceptual model.
This is beneficial, as it will lead to a more accurate description of the masking properties of the input signals.
However, it has been shown to be at the cost of computational complexity.

As such, the Par detectability will be selected for use in the sound zone algorithm.
Note however that surface through inspection Taal and Par detectability seem sufficiently similar that it is likely
possible to use the Taal detectability in place of the Par detectability in the algorithms
proposed in \autoref{ch:perceptual}. 
It is also interesting to note that the Taal detectability is computed in the time domain, whereas the 
Par detectability is computed in the frequency domain.
Exploring the possibilities of the Taal detectability will however be left to future work and not further explored in this
work.
