In \autoref{ch:perceptual:review} a literature review of various models of the human auditory system was given.
This section will detail the selection of one of these perceptual models for use in a perceptual sound zone algorithm.

The structure of this section is as follows.
First, \autoref{ch:perceptual:selection:criteria} will discuss the criteria that will be used to select the most suitable perceptual model for use in 
a perceptual sound zone algorithm.
Next, these criteria will be used to evaluate how promising the candidate models are.
This will be done by first considering the objective audio measures in \autoref{ch:perceptual:selection:objective_audio_measures}, 
followed by the perceptual models from audio coding in \autoref{ch:perceptual:selection:audio_coding}.

\subsection{Criteria for Selecting Perceptual Model}
\label{ch:perceptual:selection:criteria}
This section will define desirable criteria for the perceptual model for integration with a sound zone algorithm.

The main criteria is mathematically tractability.
That is to say, it desirable for the perceptual model to be easy to include mathematically in optimization problems.

As will be shown in \autoref{ch:sound_zone}, many sound zone algorithms are posed as optimization problems.
Therefore, it is undesirable if perceptual model is too complicated to be integrated mathematically into the existing optimization problems.
For example, if the computation of a perceptual model involves heavy conditional branching (i.e. if-statements), their
integration into an existing sound zone optimization problem may be challenging.

Furthermore, most sound zone algorithms can be posed as a convex optimization problem.
Convexity is a desirable property for an optimization problem, as it guarantees that the optimization problem has a single, 
global optimal value, rather than many locally optimal values~\cite{boyd2004convex}. 
In addition to this, there are many efficient solvers available for convex optimization problems.
Therefore, it is also desirable that the perceptual models preserve this convexity.

\subsection{Considering Objective Audio Measures}
\label{ch:perceptual:selection:objective_audio_measures}
All objective audio measures were found to be mathematically untractable. 
Every model showed a degree of non-differentiability and non-convexity in their computation.
For some models, such as the distraction model, the complete computation could not be expressed easily with mathematical expressions.
As such, they are difficult to integrate into convex optimization problems and will not be used in the perceptual sound zone algorithm.
However, as the objective audio measures predict the outcomes of listening tests, they will be used for evaluation.

\subsection{Considering Perceptual Models from Audio Coding}
\label{ch:perceptual:selection:audio_coding}
In contrast to the objective audio measures, 
all models from audio coding are mathematically tractable.

The ISO MPEG models were found to be less promising than the Par and Taal detectability, as it does not immediately 
define a cost function which can be optimized over:
the models only determine how much noise can be added per band.

As such, the decision is between the Par and Taal detectability.
The Taal detectability takes into account temporal properties of the input signal in it's perceptual model.
This is beneficial, as it will lead to a more accurate description of the masking properties of the input signals.
However, it has been shown to be at the cost of computational complexity.

Surface level inspection has shown Taal and Par detectability seem sufficiently similar that it is likely
possible to use the Taal detectability in place of the Par detectability in the algorithms
proposed in \autoref{ch:perceptual}. 
The Taal detectability is evaluated in the time domain rather than the frequency domain, which is the case for the Par model. 
Exploring the possibilities of the Taal detectability will however be left to future work and not further explored in this work.

As such, the Par detectability will be selected for use in the sound zone algorithm.
