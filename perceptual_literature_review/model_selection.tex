From the perceptual models discussed in the literature review given in \autoref{ch:perceptual:review}, the Par distortion detectability is selected for use in the proposed perceptual sound zone framework, as it 
is found to be the most tractable for optimization.
This section seeks to motivate this.

In \autoref{ch:sound_zone} it is shown that sound zone algorithms are typically posed as optimization problems. 
The goal of optimization problems is typically to minimize or maximize a cost function, which is done by leveraging the (sub)differential of the function.

Furthermore, many approaches are posed as convex optimization problems.
Convex optimization is a sub-class of optimization problems that guarantee that the optimizer is globally unique~\cite{boyd2004convex}. 
As such, one does not have to deal with many sub-optimal local optima. 
In addition to this, there are many efficient solvers available for convex optimization problems.

As such, perceptual models which contain conditional branching or complex, non-convex operations which cannot readily 
be integrated into cost functions are less promising.

To this end, all the objective audio measures discussed in \autoref{ch:perceptual:review:objective} 
are ruled out for use in the perceptual sound algorithm. 
As discussed, all models showed a degree of non-differentiability and non-convexity in their computation.
They are challenging to integrate into convex optimization problems and are therefore not used in the proposed perceptual sound zone algorithm.
They are, however, used in the evaluation of the proposed perceptual sound zone algorithm.

From the three remaining perceptual models from audio coding, 
the perceptual models proposed by the ISO MPEG standard are found to be the least promising.
As stated in \autoref{ch:perceptual:review:audio_coding}, this is because these models do not define a cost function that can be optimized over: instead, only the noise that can be added per auditory band is determined.

As such, the decision is between the Par and Taal distortion detectability, which are both expressed using a  squared L2-norm, which is a convex function~\cite{boyd2004convex}.

In contrast to the Par model, the Taal detectability takes into account the temporal properties of the input signal.
This is beneficial, as it will lead to a more accurate description of the masking properties of the input signals.
However, it has been shown to be at the cost of computational complexity.
The Taal detectability has been shown to take at least two times as long to compute as the Par detectability, with this disparity seemingly growing as a function of input signal length~\cite{taal2012low}.

In addition to this, the Taal model operates on time-domain versions of the input stimuli, whereas the Par model operates in the frequency-domain representations~\cite{van2005perceptual, taal2012low}.
Frequency-domain sound zone approaches are typically less demanding computationally than time-domain approaches~\cite{vindrola2019personal}.

As a lower computational complexity is desirable, the Par distortion detectability is used in the proposed perceptual sound zone algorithm.
Exploring the possibilities of using the Taal detectability in a perceptual sound zone algorithm is found to be promising but is left to future work and not further explored.
