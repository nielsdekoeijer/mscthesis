In \autoref{ch:perceptual:review} a literature review of various models of the human auditory system was given.
This section will detail the selection of one of these perceptual models for use in a perceptual sound zone algorithm.
The other perceptual models will still be of use however during the evaluation of the results of the perceptual
sound zone algorithm that will be derived in \autoref{ch:perceptual_sound_zone}.

The structure of this section is as follows.
First, \autoref{ch:perceptual:selection:criteria} will discuss the criteria that will be used to select the most suitable perceptual model for use in 
a perceptual sound zone algorithm.
The selection from the reviewed models will then be performed in \autoref{ch:perceptual:selection:selection} using said criteria.

\subsection{Criteria for Selecting Perceptual Model}
\label{ch:perceptual:selection:criteria}
\todo{Rewrite this.}
This section will define desirable criteria for the perceptual model for integration with a sound zone algorithm.

Two criteria can be distinguished:
\begin{enumerate}
    \item \textbf{Mathematical Tractability:}\\
        It is a desirable property for the perceptual model to be easy to include in optimization problems mathematically.

        As will be shown in \autoref{ch:sound_zone}, many sound zone algorithms are posed as optimization problems.
        Therefore, it is undesirable if perceptual model is too complicated to be integrated mathematically into optimization problems.
        For example, if the computation of a perceptual model involves conditional branching (i.e. if-statements), their
        integration into an existing optimization problem may be difficult.

        Furthermore, as most sound zone algorithms can be posed as a convex optimization problem.
        Convexity is a desirable property for an optimization problem, as it guarantees that the optimization problem has a single, 
        global optimal value, rather than many locally optimal values~\cite{boyd2004convex}. 
        Therefore, it is also desirable that the perceptual models preserve this convexity.

    \item \textbf{Computational Overhead:}\\
        It is desirable for the inclusion of the perceptual model to add as little computational load to the sound zone algorithm as possible.
\end{enumerate}

\subsection{Selection of Perceptual Model}
\label{ch:perceptual:selection:selection}
\todo{Rewrite this.}
This section will use the criteria defined by \autoref{ch:perceptual:selection:criteria} in order to select a perceptual
model.
In the literature review given in \autoref{ch:perceptual:review}, two classes of perceptual model were considered:
perceptual models from audio coding and objective audio measures.

All objective audio measures were found to be mathematically untractable. 
Every model showed a degree of non-differentiability and non-convexity in their computation.
For some models, such as the distraction model, the complete computation could not be expressed easily with mathematical expressions.
As such, they are difficult to integrate into convex optimization problems and will not be used in the perceptual sound zone algorithm.
However, as the objective audio measures predict the outcomes of listening tests, they will be used for evaluation.

All models from audio coding are mathematically tractable and of low computational overhead.
The ISO MPEG models were found to be less promising than the Par and Taal detectability, as it does not immediately 
define a cost function which can be optimized over:
the models only determine how much noise can be added per band.

As such, the decision is between the Par and Taal detectability.
The Taal detectability takes into account temporal properties of the input signal in it's perceptual model.
This is beneficial, as it will lead to a more accurate description of the masking properties of the input signals.
However, it has been shown to be at the cost of computational complexity.

As such, the Par detectability will be selected for use in the sound zone algorithm.
Note however that surface through inspection Taal and Par detectability seem sufficiently similar that it is likely
possible to use the Taal detectability in place of the Par detectability in the algorithms
proposed in \autoref{ch:perceptual}. 
It is also interesting to note that the Taal detectability is computed in the time domain, whereas the 
Par detectability is computed in the frequency domain.
Exploring the possibilities of the Taal detectability will however be left to future work and not further explored in this
work.
