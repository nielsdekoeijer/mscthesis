In \autoref{ch:perceptual:review} a literature review of various models of the human auditory system were given.
In this section, one perceptual model will be selected for use in a sound-zone algorithm.

The other perceptual models will still be of use however during the evaluation of the results of the perceptual
sound zone algorithm.

The approach to select a perceptual model is as follows.
First, \autoref{ch:perceptual:selection:criteria} will discuss the criteria that will inform the decision.
Next, \autoref{ch:perceptual:selection:selection} will use these criteria to evaluate the perceptual models
discussed in the literature review in \autoref{ch:perceptual:review}.
From this evaluation the most suited perceptual model will be selected for use in a perceptual sound zone 
algorithm.

\subsection{Criteria for Selecting Perceptual Model}
\label{ch:perceptual:selection:criteria}
This section will define desirable criteria for the perceptual model for integration with a sound zone algorithm.
These criteria will be used to select a perceptual model from the models discussed in the literature review.
\begin{enumerate}
    \item \textbf{Mathematical Tractability.}\\
        It is a desirable property for the perceptual model to be easy to include in optimization problems.
        For example, perceptual models that can are differentiable are therefore favoured over optimization problems
        that are non-differentiable.
    \item \textbf{Computational Overhead:}\\
        It is desirable for the inclusion of the perceptual model to add minimal computational load 
        to the sound zone algorithm.
        The additional overhead of the perceptual model should not increased the run time of the sound zone algorithm by many orders of magnitude.
\end{enumerate}

\subsection{Selection of Perceptual Model}
\label{ch:perceptual:selection:selection}
This section will use the criteria defined by \autoref{ch:perceptual:selection:criteria} in order to select a perceptual
model.
In the literature review given in \autoref{ch:perceptual:review}, two classes of perceptual model were considered:
perceptual models from audio coding and objective audio measures.

All objective audio measures were found to be mathematically untractable, as all models are both non-differentiable and 
non-convex functions of their input signals.
As such, they are difficult to integrate into existing sound zone algorithms.
However, as the objective audio measures predict the outcomes of listening tests, they are uniquely suited for evaluation
of the results.

All models from audio coding are mathematically tractable and of low computational overhead.
The ISO MPEG models were found to be less promising than the Par and Taal detectability as it does not immediately 
define a cost function which can be optimized over.
The ISO MPEG models only determines how much noise can be added per frequency band, but does not provide an immediate 
means of using that information in optimization.

As such, the decision is between the Par and Taal detectability.
As discussed, the Taal detectability takes into account more perceptual.
This is however this has been shown to be at the cost of computational complexity.
As such, the Par detectability will be selected for use in the sound zone algorithm.


