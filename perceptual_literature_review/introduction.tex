The goal of this chapter is to find a suitable perceptual algorithm for integration with a sound zone algorithm.

To this end, this chapter is structured as follows.
The chapter will begin in \autoref{ch:perceptual:background} with some background information on the perceptual aspects of
the human auditory system.
This is done to ensure that the reader is well informed on the perceptual aspects discussed in later parts of the thesis.

Next, with the necessary psycho-acoustical background in place, a literature review into state of the art perceptual models
is performed in \autoref{ch:perceptual:review}.
The purpose of this review is to document candidates for perceptual models that are best suited for integration into 
sound zone algorithms.
In addition to this, the reviewed models could also serve as potential candidates for use in the evaluation of the 
results of the perceptual sound zone algorithm, which will be done in \autoref{ch:perceptual_sound_zone}.

To aid in the selection of a perceptual model from the candidates found by the literature review, criteria are be defined
in \autoref{ch:perceptual:selection}. 
These criteria reflect desirable properties for a perceptual model to have for integration in a sound zone algorithm.
The criteria are then in the same section to make a selection.

Next, in order to give the reader more a greater understanding, the selected perceptual model is discussed in more detail in
\autoref{ch:perceptual:implementation} by stating implementation details and describing its behavior.
The chapter will then wrap up with some conclusions in \autoref{ch:perceptual:conclusion}.
