\section{Multi-Zone Pressure Matching Solution Approach}
\label{sec:pressure_matching}
The ``Pressure Matching'' (PM) is widely used in literature to solve the sound zone problem.
In this section, a ``Multi-Zone Pressure Matching'' (MZ-PM) algorithm will be derived.
The motivation for discussing it is that it will be used as the foundation on which the perceptual sound zone algorithm will be built, 
as it was found that perceptual information was easily intergratable into the pressure matching framework.
\todo{Expand this motivation with a couple more arguments...?}

In the typical PM approach, the resulting loudspeaker input signals $x^{(l)}[n]$ are determined for just a single zone.
If the solution for multiple zones is desired, than multiple PM problems must be solved and their resulting loudspeaker input signals combined. 
In the MZ-PM approach, the loudspeaker input signals are instead determined for jointly for all zones.

In a two zone approach, the loudspeaker input signals are decomposed into two parts as follows:
\begin{equation}
    x^{(l)}[n] = x_\za^{(l)}[n] + x_\zb^{(l)}[n]
\end{equation}
Here, $x_\za^{(l)}[n]$ and $x_\zb^{(l)}[n]$ are the parts of the loudspeaker input signal responsible for reproducing the target sound pressure 
in zone $\za$ and $\zb$ respectively.
Now, it is possible to consider the sound pressure that arises due to the separate loudspeaker input signals:
\begin{align}
    p_\za^{(m)}[n] &= \sum_{l=0}^{N_L} \left(h^{(l,m)} \ast x_\za^{(l)}\right)[n] \\
    p_\zb^{(m)}[n] &= \sum_{l=0}^{N_L} \left(h^{(l,m)} \ast x_\zb^{(l)}\right)[n]
\end{align}
Here, $p_\za^{(m)}[n]$ and $p_\zb^{(m)}[n]$ can be understood to be the pressure that arises due to 
playing loudspeaker input signals $x_\za^{(l)}[n]$ and $x_\zb^{(l)}[n]$ respectively. 

The idea in this approach is to chose $x_\za^{(l)}[n]$ and such that the resulting pressure $p_\za^{(m)}[n]$ attains the target sound pressure $t^{(m)}[n]$ in all $m \in A$.   
At the same time however, $p_\za^{(m)}[n]$ should not attain any sound pressure in all $m \in B$.
Any sound pressure resulting from $x_\za^{(l)}[n]$ in zone $\zb$ is essentially leakage or cross-talk between zones. 
Similar arguments can be given for $x_\zb^{(l)}[n]$: it should reproduce the target sound pressure for $m \in B$ but no sound pressure for $m \in A$. 

In the MZ-PM approach, the loudspeaker weights $x_\za^{(l)}[n]$ and $x_\zb^{(l)}[n]$ that achieve this goal are found by 
minimizing the difference between the intended pressure and the realized pressure as follows:

\begin{align}
    \argmin{x_\za^{(l)}[n],\,x_\zb^{(l)}[n]\,\forall\,l}{
       &\sum_{m\in A} \norm[2][2]{p_\za^{(m)}[n] - t^{(m)}[n]} +
        \sum_{m\in A} \norm[2][2]{p_\zb^{(m)}[n]} + \\
       &\sum_{m\in B} \norm[2][2]{p_\zb^{(m)}[n] - t^{(m)}[n]} + 
        \sum_{m\in B} \norm[2][2]{p_\za^{(m)}[n]}
    }
\end{align}

Here, the first two terms can be understood as the reproduction error and the leakage for zone $\za$.
Similarly, the last two terms are the reproduction error and leakage for zone $\zb$. 
Typically, this approach results in trade-off between minimizing the reproduction errors and leakages. 
Some pressure matching approaches attempt to control this trade-off by introducing weights for the different error terms, or constraints.

The problem can be solved in the time and the frequency domain.
In frequency domain approaches, the convolutions become inner products, which typically results in a lower computational complexity.

The algorithm above will form the basis of the perceptual algorithms to be introduced in the following sections.
