\section{Block-Based Multi-Zone Pressure-Matching}
In the preceding section it is assumed that the desired playback signals $s_\za[n]$ and $s_\zb[n]$ were known in their entirety.
In practice however, this is not a valid assumption as a user can change the desired playback content in real-time.
This is the case for example when a user changes the song they are playing on their system.

In reality, the sound zone system can only have knowledge the most recent samples and all previous samples.
In order to deal with this limitation, one option is to buffer a large number of incoming samples and apply the existing MZ-PM approach.
However, this would introduce significant latency to the system.

Instead, a block-based approach can be used where the incoming samples of the desired playback signals are used in real-time as they become available.
The system buffers a block of $H$ incoming samples samples, and then solves the sound zone problem for the newest block.
This results in a latency of at least $H$, which could be acceptable assuming $H$ is chosen sufficiently small. 

In addition to the benefit of block-based processing, the block-based approach is also practical for the integration of the perceptual model.
The perceptual model is designed to operate on short time segments in the order of 20 to 200 milliseconds.
Block-based processing would allow the algorithm to operate on segments of this time scale.

For these reasons, this section will adapt existing Multi-Zone Pressure Matching approach introduced in \autoref{sec:data_model:pressure_matching} 
to accommodate for block based processing.

\subsection{Mathematical Block Model}
For the block-processing based sound zone approach, the incoming samples of the desired playback signals for both zones $s_\za[m]$ and $s_\zb[m]$ are buffered into blocks.
As such, the sound zone system only has knowledge of the most recent block $\mu$ .
The relation between the global time index $n$ and block index $\mu$ is given as follows:  
\begin{equation}
      \mu = \lfloor n / H \rfloor
\end{equation}
Thus at a time $n$, up to and including the $\mu^\text{th}$ blocks of desired playback signals $s_\za[m]$ and $s_\zb[m]$ are known.

As the desired playback signals $s_\za[m]$ and $s_\zb[m]$ are revealed in a block-wise fashion, the sound zone system cannot compute the entirety of 
loudspeaker input signals $x^{(l)}_{\za}[n]$ and $x^{(l)}_{\zb}[n]$.
Instead, one approach is to compute the loudspeaker input signals in the same block-wise fashion, by finding the $H$ newest samples of the loudspeaker input signals 
as the $H$ newest samples of desired playback signals $s_\za[m]$ and $s_\zb[m]$ are revealed to the system.

\subsection{Block Based Multi-Zone Pressure-Matching}
As discussed previously, the Multi-Zone Pressure-Matching (MZ-PM) algorithm attempts to control the loudspeaker input signals 
$x^{(l)}_\za[n]$ and $x^{(l)}_\zb[n]$ such that a specified target sound pressure $t^{(m)}[n]$ is attained at all control points $m$.
Here, target sound pressure $t^{(m)}[n]$ is determined by the sound pressure that arises due to the 
desired playback signals $s_\za[n]$ and $s_\zb[n]$.

The introduction of the block-based approach limits the knowledge of $s_\za[n]$ and $s_\zb[n]$ up to and including the most recent block $\mu$.
As such, the desired playback signals are only known up for $0 \leq n \leq \mu H$, assuming causal desired playback signals.

This limitation has implications for the computation of the loudspeaker input signals and the target sound pressure.
Neither quantities can be computed in their entirety due to the limitation in knowledge of the desired playback signals.
Because the desired playback signals are revealed to the system in blocks of size $H$, the system will instead compute the loudspeaker input signals and target sound pressure
at the same rate.

As such, after a new block of desired playback signals is revealed, 
a new block of loudspeaker input signals will be computed to attain such that a new block of target sound pressure is best attained.
Adapting the existing MZ-PM algorithm to operate on a block-by-block basis is the topic of this section.

\subsubsection*{Defining Block-Based Loudspeaker Input Signals}
First, consider the implications of the block based processing on the loudspeaker input signals.
The goal is to compute $x^{(l)}_{\zz}[n]$ in a blocks of size $H$.
To do so, consider the segmentation of the sound pressure that is realized due to loudspeaker input signal $x^{(l)}_{\zz}[n]$:
\begin{align}
    p_\zz^{(m)}[n] &= \sum_{l=0}^{N_L - 1} \left(h^{(l,m)} \ast x^{(l)}_\zz\right)[n] \\
               &= \sum_{l=0}^{N_L - 1} \sum_{b = n - N_h + 1}^{n} h^{(l,m)}[n - b] x^{(l)}_\zz[b] \\
               &= \sum_{l=0}^{N_L - 1} \sum_{b = n - N_h + 1}^{n} h^{(l,m)}[n - b] x^{(l)}_\zz[b] \sum_{k=-\infty}^{\infty} w[b - kH]\\
               &= \sum_{l=0}^{N_L - 1} \sum_{b = n - N_h + 1}^{n} h^{(l,m)}[n - b] \sum_{k=-\infty}^{\infty} x^{(l)}_\zz[b] w[b - kH]\\
\end{align}
Here, $w[n]\in\Real{N_w}$ is a window that is defined to be non-zero for $-N_w + 1 \leq n \leq 0$, as such it is non-causal.
It is chosen such that satisfies the constant overlap add (COLA) condition for a hop size $H$ , which is given as follows:
\begin{equation}
    \sum_{k=-\infty}^{\infty} w[n - kH] = 1\quad\forall\,n
\end{equation}
The interpretation of the rewrite of $p_\zz^{(m)}[n]$ above can be understood as a projection of the loudspeaker input signals $x^{(l)}_\zz[n]$ onto a basis overlapping windows $w[n]$.
The hop-size is chosen to be equal to the block size $H$, as such the overlap is equal to $N_w - H$. 
This forms individual frames $x^{(l)}_{\zz}[n]w[n - \mu H]$, which have support $-N_w + 1 + \mu H \leq n \leq \mu H$.
Due to the properties of the COLA condition, the individual frames can be recombined to form the complete loudspeaker input signal:
\begin{equation}
    x^{(l)}_\zz[n] = \sum_{k=-\infty}^{\infty} x^{(l)}_\zz[n] w[n - kH]
\end{equation}
This model can be used order to compute the complete loudspeaker input signal $x^{(l)}_\zz[n]$ block-by-block by solving the sound zone problem per frame $x^{(l)}_\zz[n] w[n - kH]$.
When the $\mu^\text{th}$ block of $s_\zz[n]$ is revealed, this allows for an update to target sound pressure (how this is done is discussed later).
The idea is then to compute the $x^{(l)}_\zz[n] w[n - \mu H]$ frame of $x^{(l)}_\zz[n]$ such that said target pressure is best attained.

In order to do so, let $x^{(l)}_{\zz,\mu}[n]\in\Real{N_w}$ represent the content of the $\mu^\text{th}$ frame.  

When block $\mu$ of $s_\zz[n]$ is revealed, $x^{(l)}_{\zz,\mu}[n]$ can be computed such that the target pressure defined by the new desired playback content is best attained.
The $\mu^\text{th}$ frame can then be added in an overlap-add like fashion to compute the loudspeaker input signal in real-time.

Before deriving how the loudspeaker input signal frame content $x^{(l)}_{\zz,\mu}[n]$ can be computed 
the block-wise computation the target sound pressure must be discussed.
This is the topic of the next paragraph.

\subsection{Block-Based Target Pressure Computation}
As mentioned, the playback signals $s_\za[n]$ and $s_\zb[n]$ are revealed in blocks of size $H$ in the block based framework.
As such, for block $\mu$, the knowledge of the desired playback signals is limited, which results in a limited knowledge in the target sound pressure $t^{(m)}[n]$.

The target sound pressure can be segmented in a way analogous to the segmentation of the loudspeaker input signals:
\begin{align}
    t^{(m)}[n] &= \sum_{l=0}^{N_L - 1} \left(h^{(l,m)} \ast s_\zz\right)[n] \\
               &= \sum_{l=0}^{N_L - 1} \sum_{b = n - N_h + 1}^{n} h^{(l,m)}[n - b] s_\zz[b] \\
               &= \sum_{l=0}^{N_L - 1} \sum_{b = n - N_h + 1}^{n} h^{(l,m)}[n - b] s_\zz[b] \sum_{k=-\infty}^{\infty} w[b - kH] \\
               &= \sum_{l=0}^{N_L - 1} \sum_{b = n - N_h + 1}^{n} h^{(l,m)}[n - b] \sum_{\mu =-\infty}^{\infty} s_\zz[b] w[b - \mu H] \\
\end{align}
In the rewrite above, the desired playback signal $s_\zz[n]$ is projected onto a basis spanned by windows $w[n]$ of size $N_w$. 
The equation above essentially sums the contributions individual contributions of windowed blocks.
Note also that the windowed blocks need not be overlapping in general, i.e. one possible choice of window is the rectangular window $H$ with $N_w = H$.

This allows for a formulation of $t_\mu^{(m)}[n]$ in which we only consider the contribution up to and including the $\mu^\text{th}$ windowed block.  
Such a formulation is given as follows:
\begin{align}
    t_\mu^{(m)}[n] &= \sum_{l=0}^{N_L - 1} \sum_{b = n - N_h + 1}^{n} h^{(l,m)}[n - b] \sum_{k=-\infty}^{\mu} s_{\zz}[b] w[b - \mu H]\\
                   &= \sum_{l=0}^{N_L - 1} \sum_{b = n - N_h + 1}^{n} h^{(l,m)}[n - b] s_{\zz}[b]\left(w[b - \mu H] + \sum_{k=-\infty}^{\mu - 1} w[b - \mu H] \right)\\
                   &= \sum_{l=0}^{N_L - 1} \sum_{b = n - N_h + 1}^{n} h^{(l,m)}[n - b] s_{\zz}[b] w[b - \mu H] + t_{\mu - 1}^{(m,l)}[n]
\end{align}
As can be seen, $t_\mu^{(m)}[n]$ is expressed as the contribution of the current windowed blocks $\mu$ and the contribution of all previous blocks $-\infty \leq k \leq \mu - 1$.
The computation can be performed recursively:
to compute $t_\mu^{(m)}[n]$, we compute the convolution of the current windowed block $s_{\zz,\mu}[n]w[n - \mu H]$ with the RIRs, 
and then add the history of previous blocks.

Thus, $t_\mu^{(m)}[n]$ can be understood to be the target pressure given blocks up to $\mu$. 
As new blocks are revealed, the target target sound pressure can be updated.
Note that this definition converges to the ``true'' target sound pressure:
\begin{equation}
    t_\infty^{(m)}[n] = t^{(m)}[n]  
\end{equation}
One interpretation of was done so far is that the target sound pressure can be performed by
breaking the convolution of the desired playback signal $s_{\zz}[n]$ with the room impulse responses $h^{(l,m)}[n]$
into a sum of convolutions of windowed blocks of the desired playback signal.
In doing so, the target sound pressure can be computed in real-time as new samples of $s_{\zz}[n]$ come available.

The target sound pressure $t_\mu^{(m)}[n]$ will be used in the computation of $x^{(l)}_{\zz,\mu}[n]$.
The idea here is to choose $x^{(l)}_{\zz,\mu}[n]$ such that the resulting sound pressure best matches $t_\mu^{(m)}[n]$.

As $t_\mu^{(m)}[n]$ is shown to converge to the true sound pressure, the sum of the all contributions of $x^{(l)}_{\zz,\mu}[n]$ is thus chosen to best attain it. 
How this is done exactly is discussed in the next section.

\subsubsection*{Derivation of Block-Based Multi-Zone Pressure-Matching}
After translating the loudspeaker input signals and the target sound pressure into their block-wise counterparts, 
the Block-Based Multi-Zone Pressure-Matching (BB-MZ-PM) algorithm can be stated.

To begin, let the sound pressure realized after playing the first $\mu$ loudspeaker input signal frames $x^{(l)}_{\zz,\mu}[n]$ be denoted by $p_{\zz,\mu}^{(m)}[n]$,
which can be expressed as follows:
\begin{align}
    p_{\zz,\mu}^{(m)}[n] &= \sum_{l=0}^{N_L - 1} \sum_{b = n - N_h + 1}^{n} h^{(l,m)}[n - b] \sum_{k=-\infty}^{\mu} x^{(l)}_{\zz,k}[b] w[b - kH] \\
                   % &= \sum_{l=0}^{N_L - 1} \sum_{b = n - N_h + 1}^{n} h^{(l,m)}[n - b] \left(x^{(l)}_{\zz,\mu}[b]w[b - \mu H] 
                        % + \sum_{k=-\infty}^{\mu - 1}x^{(l)}_{\zz,k}[b]w[b - kH] \right)\\
                   &= \sum_{l=0}^{N_L - 1} \sum_{b = n - N_h + 1}^{n} h^{(l,m)}[n - b]  x^{(l)}_{\zz,\mu}[b]w[b - \mu H] + p_{\zz,\mu-1}^{(m)}[n]
\end{align}
Note that $p_{\zz,\mu}^{(m)}[n]$ can be computed recursively just as was the case with the target sound pressure.
The sound pressure at a given moment is equal to the sound pressure due to the most recent loudspeaker input signal frame $k = \mu$ and the previous loudspeaker input frames $-\infty \leq k \leq \mu - 1$.

The solution approach: computing the loudspeaker input signal frames $x_\mu^{(l)}[n]$ such that the resulting sound pressure $p_{\zz,\mu}^{(m)}[n]$ best attains $t_\mu^{(l)}[n]$.
As such, the optimization is over one frame at a time, all frames $k \leq \mu - 1$ are assumed constant. 
Note however that $x_\mu^{(l)}[n]$ can only influence samples $\mu H - N_w + 1$ due to its finite support.

The final optimization problem can be thus stated as follows:
\begin{align}
    \argmin{x_{\za,\mu}^{(l)}[n],\,x_{\zb,\mu}^{(l)}[n]\,\forall\,l}{
       &\sum_{m\in A} \norm[2][2]{p_{\za,\mu}^{(m)}[n] - t_\mu^{(m)}[n]} +
        \sum_{m\in A} \norm[2][2]{p_{\zb,\mu}^{(m)}[n]} + \\
       &\sum_{m\in B} \norm[2][2]{p_{\zb,\mu}^{(m)}[n] - t_\mu^{(m)}[n]} + 
        \sum_{m\in B} \norm[2][2]{p_{\za,\mu}^{(m)}[n]}
    }
\end{align}

The problem above is solved recursively for loudspeaker input signal frames $x_{\za,\mu}^{(l)}[n]$ and $x_{\zb,\mu}^{(l)}[n]$ as new samples $s_\za[n]$ and $s_\zb[n]$ are revealed.
The final loudspeaker input signals can then be found in real-time as follows: 
\begin{equation}
    x^{(l)}_\zz[n] = \sum_{k=-\infty}^{\mu} x^{(l)}_{\zz,\mu}[n] w[n - kH]\quad \forall\,n \leq \mu H - N_w + H
\end{equation}
The expression above is only valid up to $n \leq \mu H - N_w + H$ due to missing overlapping frames.
The resulting $x^{(l)}_\zz[n]$ can then be played in real-time as the loudspeaker input signal frames $x^{(l)}_{\zz,\mu}[n]$ are being computed.
