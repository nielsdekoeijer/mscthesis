\section{Frame-Based Processing Framework}
In the preceding section it is assumed that the desired playback signals $s_\za[n]$ and $s_\zb[n]$ were known in their entirety.
In practice however, this is not a valid assumption as a user can change the desired playback content in real-time.
In this section, the Multi-Zone Pressure Matching approach introduced in \autoref{sec:data_model:pressure_matching} will be adjusted 
in order to accommodate the real-time processing constraint.
\todo{This really needs better motivation...?}

As mentioned, it cannot be assumed that the desired playback signals $s_\za[n]$ and $s_\zb[n]$ are known for all $n$ 
if the system is to accommodate real-time playback.  
Instead, at time $m$ it is assumed that $s_\za[n]$ and $s_\zb[n]$ are only known from $\infty \leq n \leq m$, 
i.e. the most recent and all previous input samples.

The loudspeaker input signals $x^{(l)}[n]$ must thus be found in real-time.
In order to do so, a block-processing based approach is taken where $x^{(l)}[n]$ is computed in blocks of hop size $H$. 
When $H$ new samples of $s_\za[n]$ and $s_\zb[n]$ are revealed to the system, $H$ new samples of $x^{(l)}[n]$ can be computed.
As such, this approach will introduce a delay of at least $H$, as $H$ new samples of the desired playback signals are required before computation can begin.
\todo{This really needs better motivation...?}

The processing occurs in steps of $H$ samples.
Let $\mu = \lfloor n / H \rfloor$ index is the current step at a time $n$.
Thus at a time $n$, up to and including the $\mu^\text{th}$ block of desired playback signals is known.

Adapting the previously derived data model to the chosen block-based approach will be the topic of this section.

\subsection{Implications for Computing Target Pressure}
As the playback signals are revealed in blocks of size $H$, it makes sense to update the target sound pressure $t^{(m)}[n]$ in blocks of $H$.  
Consider the following rewrite of the target signal for the desired sound pressure of zone $\za$:
\begin{align}
    t^{(m)}[n] &= \sum_{l=0}^{N_L - 1} \left(h^{(l,m)} \ast s_\za\right)[n] \\
               &= \sum_{l=0}^{N_L - 1} t^{(l,m)}[n]
\end{align}
Here, $t^{(m,l)}[n]$ is the contribution of the $l^\text{th}$ loudspeaker to the target sound pressure at reproduction point $m$.
Now, consider the following rewrite:
\begin{align}
     t^{(m,l)}[n]   &= \sum_{b = n - N_h + 1}^{n} h^{(l,m)}[n - b] s_\za[b] \\
                    &= \sum_{b = n - N_h + 1}^{n} h^{(l,m)}[n - b] s_\za[b] 
                        \sum_{k=-\infty}^{\infty} w[b - kH] \\
                    &= \sum_{b = n - N_h + 1}^{n} h^{(l,m)}[n - b]  
                        \sum_{k=-\infty}^{\infty} s_\za[b] w[b - kH] \\
                    &= \sum_{b = n - N_h + 1}^{n} h^{(l,m)}[n - b]  
                        \sum_{k=-\infty}^{\infty} s_{\za,k}[b]  
\end{align}
Here, $w[n]\in\Real{N_w}$ is a window that satisfies the COLA condition for a hop size $H$.
The window is defined to be non-zero for $-N_w + 1 \leq n \leq 0$, as such it is non-causal.
Furthermore, the windows are overlapping, thus $N_w > H$. 

In the rewrite above, the desired playback signal $s_\za[n]$ is projected onto a basis of overlapping frames of size $N_w$.
The projection results in a sum of frames $s_{\za, k}[n]$. 
The support of $s_{\za, k}[n]$ is defined by the shifted window that is used to synthesize it, i.e. it is non-zero for $-N_w + 1 + kH \leq n \leq kH$.
As the COLA condition is met for the chosen window, the sum over all frames reconstructs $s_\za[n]$ perfectly.

At a time $n = \mu H$, the frames up to $k = \mu$ can be computed.
Let $t_\mu^{(m,l)}[n]$ represent the target using frames up to $k = \mu$:  
\begin{align}
    t_\mu^{(m,l)}[n] &= \sum_{b = n - N_h + 1}^{n} h^{(l,m)}[n - b] 
                        \sum_{k=-\infty}^{\mu} s_{\za,k}[b] \\
                     &= \sum_{b = n - N_h + 1}^{n} h^{(l,m)}[n - b] 
                        s_{\za,\mu}[b] + \sum_{b = n - N_h + 1}^{n} h^{(l,m)}[n - b] 
                        \sum_{k=-\infty}^{\mu - 1} s_{\za, k}[b] \\
                     &= \sum_{b = n - N_h + 1}^{n} h^{(l,m)}[n - b] 
                        s_{\za, \mu}[b] + t_{\mu - 1}^{(m,l)}[n]
\end{align}
As can be seen, $t_\mu^{(m,l)}[n]$ can expressed as the contribution of the current frames and the contribution of all previous frames.
In addition, note how the computation can be performed recursively.
To compute $t_\mu^{(m,l)}[n]$, we compute the convolution of the current frame $s_{\za, \mu}[n]$ with the RIRs, 
and then add the history of previous frames.

Thus, $t_\mu^{(m,l)}[n]$ can be considered an estimation of the target given frames up to $\mu$. 
As new frames are revealed, the target estimation can be updated.
Note that this definition converges to the ``true'' target estimation: $t_\infty^{(m,l)}[n] = t^{(m,l)}[n]$. 

Essentially, this approach allows for the real-time computation for the target signal.

\subsection{Implications for Computing Loudspeaker Inputs}
Just as the target is comuted as new frames are revealed, the loudspeaker input should also be computed this way.
\todo{I need to motivate this?}
Consider the following rewrite of the realized sound pressure $p^{(l)}[n]$:
\begin{align}
    p^{(m)}[n] &= \sum_{l=0}^{N_L - 1} \left(h^{(l,m)} \ast x^{(l)}\right)[n] \\
               &= \sum_{l=0}^{N_L - 1} p^{(l,m)}[n]
\end{align}
$p^{(m,l)}[n]$ is the contribution of the $l^\text{th}$ loudspeaker to the realized sound pressure at reproduction point $m$.
Consider the following:
\begin{align}
    p^{(m,l)}[n]   &= \sum_{b = n - N_h + 1}^{n} h^{(l,m)}[n - b] x^{(l)}[b] \\
                    &= \sum_{b = n - N_h + 1}^{n} h^{(l,m)}[n - b] x^{(l)}[b] 
                        \sum_{k=-\infty}^{\infty} w[b - kH] \\
                    &= \sum_{b = n - N_h + 1}^{n} h^{(l,m)}[n - b]  
                        \sum_{k=-\infty}^{\infty} x^{(l)}[b] w[b - kH] \\
                    &= \sum_{b = n - N_h + 1}^{n} h^{(l,m)}[n - b]  
                        \sum_{k=-\infty}^{\infty} x_k^{(l)}[b]  
\end{align}
Analogous to the derivation for the target sound pressure $t^{(m,l)}[n]$, the loudspeaker input signal is project onto a basis consisting of windows $w[n]$.
This results in frames $x_k^{(l)}[n]$. 
Just as with the target sound pressure, let $p_\mu^{(m,l)}[n]$ represent the realized sound pressure using frames up to $k = \mu$: 
\begin{align}
    p_\mu^{(m,l)}[n] &= \sum_{b = n - N_h + 1}^{n} h^{(l,m)}[n - b] 
                        \sum_{k=-\infty}^{\mu} x_k^{(l)}[b] \\
                     &= \sum_{b = n - N_h + 1}^{n} h^{(l,m)}[n - b] 
                        x_\mu^{(l)}[b] + \sum_{b = n - N_h + 1}^{n} h^{(l,m)}[n - b] 
                        \sum_{k=-\infty}^{\mu - 1} x_k^{(l)}[b] \\
                     &= \sum_{b = n - N_h + 1}^{n} h^{(l,m)}[n - b] 
                        x_\mu^{(l)}[b] + p_{\mu - 1}^{(m,l)}[n]
\end{align}
As such, the realized pressure can be computed recursively just like the target sound pressure.
Using this data model also allows for the recursive computation of loudspeaker frames $x_\mu^{(l)}[n]$.
The loudspeaker input signals must be chosen such that sound pressure realized by them must approximate the target sound pressure.

This reveals one possible solution approach: computing the loudspeaker frames $x_\mu^{(l)}[n]$ such that the target sound pressure $t_\mu^{(l)}[n]$ is attained.
\todo{Expand on this. Why does this approach make sense? There are probably other ways of doing it aswell...}

\subsection{Implications for the Multi-Zone Pressure Matching Algorithm}
\todo{Expand...}
In the previous sections, frame-based approaches for the target sound pressure and the realized sound pressure were derived.
This allows us to rewrite the problem as follows:

\begin{align}
    \argmin{x_{\za,\mu}^{(l)}[n],\,x_{\zb,\mu}^{(l)}[n]\,\forall\,l}{
       &\sum_{m\in A} \norm[2][2]{p_{\za,\mu}^{(m)}[n] - t_\mu^{(m)}[n]} +
        \sum_{m\in A} \norm[2][2]{p_{\zb,\mu}^{(m)}[n]} + \\
       &\sum_{m\in B} \norm[2][2]{p_{\zb,\mu}^{(m)}[n] - t_\mu^{(m)}[n]} + 
        \sum_{m\in B} \norm[2][2]{p_{\za,\mu}^{(m)}[n]}
    }
\end{align}

The problem above is solved recursively for each frame $\mu$. 
The loudspeaker input signals $x_\mu^{(l)}$ can then be recombined as follows:

\begin{equation}
    x^{(l)} = \sum_{k=-\infty}^{\infty} x_k^{(l)}
\end{equation}
