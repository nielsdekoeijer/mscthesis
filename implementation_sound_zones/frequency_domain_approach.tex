\section{Frequency Domain Block Based Multi-Zone Pressure Matching}
In the previous section, the Block-Based Multi-Zone Pressure-Matching (BB-MZ-PM) algorithm was derived.
When deriving this algorithm it was stated that it's advantages are twofold.

Firstly, one advantage of using this algorithm over its non block-based counterpart is that it can work in real-time.

Secondly, the block-based approach works on a variable time-scale determine by the block size $H$.
As a result, it can operate on short time-scales.
This is useful, as the perceptual model that we wish to integrate operates on short time-scales of the order of 20 to 200 milliseconds.

There is however an additional adjustment that needs to be made before the perceptual model can be integrated. 
Currently, the BB-MZ-PM algorithm operates in the time domain, whereas the perceptual model operates in the frequency domain.

For this reason, this section will convert the existing time domain BB-MZ-PM algorithm to an equivalent frequency domain formulation.
By equivalent it is meant that the algorithms has the same resulting loudspeaker input signals $x_\zz^{(l)}$.

In order to relate the frequency domain and the time domain, a natural choice is are discrete fourier transform (DFT) and inverse discrete fourier transform (IDFT).

\subsection{Quantities the Frequency Domain}
In this section, the quantities used in the Block-Based Multi-Zone Pressure-Matching approach will be converted to their frequency domain counterparts.

Essentially, this involves converting the sound pressures $p_{\zz,\mu}^{(m)}[n]$ and $t_\mu^{(m)}[n]$ to their frequency domain versions given by 
$\hat{p}_{\zz,\mu}^{(k)}[n]$ and $\hat{t}_\mu^{(k)}[n]$ respectively.

One approach is to evaluate the sound pressures $p_{\zz,\mu}^{(m)}[n]$ and $t_\mu^{(m)}[n]$ first in the time domain, and then take the DFT.
However, the computation of the time-domain versions involves convolutions with the room impulse responses.
Another approach is to instead and compute the time-domain convolutions in frequency domain.

Here, use can be made of a property of the DFT: inner product in the frequency domain corresponds to a circular convolution in the time domain.
Circular convolution can be made to correspond to linear convolution by means of zero-padding to convolution length.

As such, imagine the convolution between a RIR $h\in\Real{N_h}$ and a loudspeaker input signal $x\in\Real{N_x}$.
Here, the convolution length is equal to $N_c = N_h + N_x - 1$.
The following holds:
\begin{align}
    \sum_{n=0}^{N_c - 1} \left(h \ast x\right)[n]\exp{\left(\omega kn\right)}  &= \left[\sum_{n=0}^{N_c - 1} h[n]\exp{\left(\omega kn\right)}\right] \circ \left[\sum_{n=0}^{N_c - 1} x[n]\exp{\left(\omega kn\right)}\right] \\
    &= \hat{h}[k] \circ \hat{x}[k]
\end{align}
Here, $\omega = 2\pi / N_c$. 
The expression above relates the DFT of the linear convolution between RIR and loudspeaker input signal to the inner product between the DFT of the RIR and loudspeaker input signals
The DFT as defined here implicitly zero-pads the input signals to convolution length $N_c$.  

This property can be used to find the frequency domian version of the pressure $p_{\zz,\mu}^{(m)}[n]$ as follows: 
\todo{Spontaneously introduced some funky notation... Also, this may not make sense. Consider the time shift required for the DFT.}
\begin{align}
    \hat{p}_{\zz,\mu}^{(m)}[k] &= \left\{\sum_{n=0}^{N_c - 1}\left[\sum_{l=0}^{N_L - 1} \left(h^{(l,m)} \ast x^{(l)}_{\zz,\mu}w_\mu\right)[n]\right]\exp{\left(\omega kn\right)}\right\} + \hat{p}_{\zz,\mu-1}^{(m)}[k] \\
                               &= \left\{\sum_{l=0}^{N_L - 1} \hat{h}^{(l,m)}[k] \circ \left[\sum_{n=0}^{N_c - 1} \left(x^{(l)}_{\zz,\mu}[n - \mu H]w_\mu[n]\right)\exp{\left(\omega kn\right)}\right]\right\} + \hat{p}_{\zz,\mu-1}^{(m)}[k] 
\end{align}
\todo{Similar arguments for the target pressure}

\subsection{Proposed Frequency Domain Approach}
\todo{
    Essentially, we zero-pad and go to the frequency domain for all quantities.
    Then all convolutions become inner products.
    Zero-padding is done to convolution length.
    We optimize still over the time domain signal.
}
\begin{align}
    \argmin{x_{\za,\mu}^{(l)}[n],\,x_{\zb,\mu}^{(l)}[n]\,\forall\,l}{
       &\sum_{m\in A} \norm[2][2]{\hat{p}_{\za,\mu}^{(m)}[k] - \hat{t}_\mu^{(m)}[k]} +
        \sum_{m\in A} \norm[2][2]{\hat{p}_{\zb,\mu}^{(m)}[k]} + \\
       &\sum_{m\in B} \norm[2][2]{\hat{p}_{\zb,\mu}^{(m)}[k] - \hat{t}_\mu^{(m)}[k]} + 
        \sum_{m\in B} \norm[2][2]{\hat{p}_{\za,\mu}^{(m)}[k]}
    }\\
    \subjectto{
       &\hat{p}_{\za,\mu}^{(m)}[k] = \text{Windowing, Zero-pad, DFT of } x_{\za,\mu}^{(l)}[n]\quad\forall\ m\in\za \\
       &\hat{p}_{\zb,\mu}^{(m)}[k] = \text{Windowing, Zero-pad, DFT of } x_{\zb,\mu}^{(l)}[n]\quad\forall\ m\in\zb 
   }
\end{align}


