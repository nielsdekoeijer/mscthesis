In order to build a perceptual sound zone algorithm, we review literature for perceptual models to find a suitable perceptual model.
\begin{itemize}
    \item I will discuss the criteria which will determine which perceptual model is chosen.
        \begin{itemize}
            \item Complexity
            \item Feasibility to optimize
        \end{itemize}
    \item I will discuss my literature review into perceptual models to find a model that best fits the criteria.
        \begin{itemize}
            \item Dau Model
            \item Detectibility Models, i.e. Par and Taal
            \item Distraction Model
            \item Audio quality models, PEAQ, VISQOL
            \item Speech Intelligibility Based, i.e. SIIB and STOI
        \end{itemize}
    \item I will discuss and motivate the chosen sound zone approach (pressure matching) and the chosen perceptual model (detectibility).
        This is done by means of summarizing the findings, and then reflecting on the criteria.
        From this, I will conclude that the \textbf{Par Detectibility} is best suited.
    \item I will discuss what perceptual models will be used for evaluation.
        From this I will conclude that PEAQ / VISQOL are useful for quality evaluation, and that the distraction model is useful for leakage evaluation.
\end{itemize}
